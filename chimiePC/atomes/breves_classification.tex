% Niveau :      PCSI *
% Discipline :  Chimie Orga I
% Mots clés :   Spectrométrie UV-visible, Réactions acidobasiques

\begin{exercise}{Classification périodique\quad---\quad Le zirconium}{1}{PCSI}
{Atomistique,Classification périodique, Structure électronique}{bermu}

\begin{questions}
    \question En justifiant, donner la position du zirconium (Zr $Z=40$) dans la classification périodique.
    \question Donner le numéro atomique du hafnium Hf qui se situe dans la même colonne mais sur la période suivante.
\end{questions}
\end{exercise}

\vfill

\begin{exercise}{Classification périodique\quad---\quad Le molybdène}{1}{PCSI}
{Atomistique,Classification périodique, Structure électronique}{bermu}

\begin{questions}
    \question En justifiant, donner la position du molybdène (Mo $Z=42$) dans la classification périodique.
    \question Proposer un remplissage alternatif de la couche de valence du molybdène et justifier qu'il possède des propriétés magnétiques.
\end{questions}
\end{exercise}

\vfill

\begin{exercise}{Classification périodique\quad---\quad Le molybdène}{1}{PCSI}
{Atomistique,Classification périodique, Structure électronique}{bermu}    

\begin{questions}
    \question En justifiant, donner la position du manganèse (Mn $Z=25$) dans la classification périodique.
    \question  Proposer des degrés d'oxydation et les ions susceptibles d'être observés pour le manganèse.
\end{questions}
\end{exercise}

\vfill 
~