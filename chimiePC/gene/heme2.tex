% Niveau :      PCSI *
% Discipline :  Chimie Orga I
% Mots clés :   Spectrométrie UV-visible, Réactions acidobasiques

\begin{exercise}{Influence du pH sur l'hémoglobine (2)}{2}{PCSI}
{Chimie générale,Réactions acidobasiques,Réactions de complexation}{bermu}

L'hémoglobine est une protéine globulaire qui permet de transporter le dioxygène $\mathrm{O_2}$ dans le sang des poumons vers les autres organes. Elle est constituée de 4 complexes du fer, appelés hèmes et notés $\mathrm{HmNH_2}$ qui se lient aux molécules de $\mathrm{O_2}$ dissoutes dans le sang.

Nous nous intéressons dans cet exercice à l'influence du pH sur la ligation du $\mathrm{O_2}$ sur les hèmes.

\begin{questions}
\questioncours Analogies et différences entre les réactions acido-basiques avec des polyacides AH$_n$ et les réactions de complexation avec des complexes ML$_n$. \\ On détaillera en particulier les considérations de stabilité (diagrammes de prédominance / distribution, $K_a$ / p$K_a$ et $K_d$ / p$K_d$).

\begin{EnvUplevel}
    Le mécanisme de complexation de l'hème avec le O$_2$ est le suivant :
    \begin{center}\schemestart[][-6]
        \chemfig{HmNH_2}
        \arrow{<=>[O$_{2\text{ (aq)}}$][$K_d$]}[0,1.5]
        \chemfig{O_2-HmNH_2}
        \arrow{<=>[H$^+_\text{ (aq)}$][?]}[-90,1.5]
        \chemfig{O_2-HmNH_3^+}
        \arrow{<=>[O$_{2\text{ (aq)}}$][$\gamma K_d$]}[-180,1.5]
        \chemfig{HmNH_3^+}
        \arrow{<=>[\rotatebox{180}{$K_a$}][\rotatebox{180}{H$^+_\text{ (aq)}$}]}[90,1.5]
    \schemestop\chemnameinit{}
    \end{center}
    
    $K_a = 2,6 \times 10^{-8}$ étant la constante d'acidité du groupe amine de l'hème, \\
    $K_d = 4,9 \times 10^{-6}$ la constante de dissociation de l'hème et du O$_2$, \\
    $\gamma = 0,73$ un facteur sans dimension.
\end{EnvUplevel}

\question Donner la relation entre le $K_d$ de l'hème et les concentrations des espèces en solution.

\question Donner la relation entre $K_a$, le pH et les concentrations des espèces en solutions.

\question Expliciter la constante de la réaction \chemfig{O_2-HmNH_2} $\leftrightharpoons$ \chemfig{O_2-HmNH_3^+}. \\
Quelle est la signification de $\gamma$ ?

\question Le pH favorise-t-il donc le transport du dioxygène dans le sang ?

\uplevel{On s'intéresse à la constante effective de dissociation $K_\text{eff}$ de l'hème qui prend en compte les formes acides et basiques de l'hème
$$K_\text{eff} = \mathrm{[O_2]} \times \dfrac{\Sigma \text{ espèces non liées}}{\Sigma \text{ espèces liées}}$$}

\question Expliciter l'expression de $K_\text{eff}$ en fonction des concentrations des espèces en solution puis en fonction uniquement du pH, $K_a$, $K_d$ et $\gamma$.

\uplevel{On appelle saturation S$_{\text{O}_2}$ la proportion de hèmes liés au dioxygène.}

\question Quel est le sens biologique de S$_{\text{O}_2}$ ?

\question Sachant que la concentration en $[\text{O}_2]$ est proportionnelle à la pression partielle $P_{\text{O}_2}$,\\
\hfill $[\text{O}_2] = \kappa_\textsc{h} P_{\text{O}_2},$ \hfill ~\\
montrer que S$_{\text{O}_2}$ peut s'écrire
$$\text{S}_{\text{O}_2} = \dfrac{P_{\text{O}_2}/P_{50}}{1 + P_{\text{O}_2}/P_{50}},$$
et expliciter l'expression et la signification de $P_{50}$.

\question{\sffamily Question ouverte :} Modéliser l'effet de ligands concurrents (CO$_2$, CN$^-$...) sur la S$_{\text{O}_2}$ et donner l'expression corrigée de la loi établie à la question précédente.

\end{questions}

\plusloin
Dash, R.K. \emph{et al, Ann Biomed Eng} \textbf{32}, 1676--1693 (2004).

\end{exercise}

\begin{solution}
\begin{questions}
    \questioncours
    \begin{itemize}
        \item Analogies : ces réactions de dissociation sont de manière générale
        $$\mathrm{XL_\textit{n} \leftrightharpoons X L_\textit{n-1} + L},$$
        et dont la constante est
        $$K_d = \mathrm{\dfrac{[XL_\textit{n-1}][L]}{[X L_\textit{n}]}} \quad \Longleftrightarrow \text{p}K_d = -\log K_d = \text{pL} + \log\mathrm{\dfrac{[XL_\textit{n-1}]}{[X L_\textit{n}]}}.$$
        \begin{center}\begin{tabular}{c|cc}
             & \textbf{Réactions acido-basiques} & \textbf{Réactions de coordination} \\ \hline\hline
            L & H$^+$ & Ligand \\
            XL$_n$ & Acide & Complexe \\
            XL$_{n-1}$ & Base & Complexe \\
        \end{tabular}\end{center}  
        
        \item La principale différence entre les réactions acide-base et de complexation est que les complexes peuvent dismuter ce qui n'est pas le cas pour les ampholytes car on a toujours
        $$\text{p}K_{a1} > \text{p}K_{a2} > \ldots,$$
        ce qui n'est pas le cas pour les complexes.
    \end{itemize}
    
    \question \hfill $K_d = \mathrm{\dfrac{[HmNH_2][O_2]}{[O_2\cdot HmNH_2]}}.$ \hfill ~
    
    \question Et de même \hfill $K_a = \mathrm{\dfrac{[HmNH_2][H^+]}{[HmNH_3^+]}}.$ \hfill ~
    
    \question La constante de la réaction $\mathrm{HmNH_2 \longrightarrow O_2\cdot HmNH_3^+}$ étant $\gamma K_a K_d$, la constante de la réaction $\mathrm{HmNH_3^+ \longrightarrow O_2\cdot HmNH_3^+}$ est $\gamma K_a$. \\
    $1/\gamma$ est donc le facteur indiquant à quel point la coordination du dioxygène O$_2$ par l'hème est moins efficace lorsque celle-ci est protonée.
    
    \question Et comme $\gamma < 1$, une augmentation de $\mathrm{[H^+]}$ / diminution de pH défavorise la coordination de O$_2$ par l'hème (souvent, H$^+$ défavorise les réactions de coordination).
    
    \question $K_\text{eff} = \mathrm{[O_2] \times \dfrac{[HmNH_2] + [HmNH_3^+]}{[O_2\cdot HmNH_2] + [O_2\cdot HmNH_3^+]}}
    = K_d \times \mathrm{\dfrac{[O_2\cdot HmNH_2] + \gamma [O_2\cdot HmNH_3^+]}{[O_2\cdot HmNH_2] + [O_2\cdot HmNH_3^+]}}$
    $$\text{d'où} \qquad K_\text{eff} = K_d \dfrac{1 + \dfrac{[H^+]}{K_a}}{1 + \dfrac{1}{\gamma}\,\dfrac{[H^+]}{K_a}}$$
    
    \question La S$_{\text{O}_2}$ représente la fraction d'hémoglobine qui transporte effectivement du O$_2$. C'est le marqueur biologique de l'efficacité de la respiration.
    
    \question $$\mathrm{S_{O_2}} = \dfrac{\mathrm{[O_2]}/K_\text{eff}}{1 + \mathrm{[O_2]}/K_\text{eff}} = \dfrac{P_{\text{O}_2}/P_{50}}{1 + P_{\text{O}_2}/P_{50}},$$
    avec $P_{50} = K_\text{eff} / \kappa_\textsc{h}$, la pression partielle en O$_2$ nécessaire à avoir une S$_{\text{O}_2}$ à 50\%.
    
    \question L'hème peut également complexer le CO$_2$ et CN$^-$ et va donc faire diminuer la S$_{\text{O}_2}$.
    
    Notons de manière générale L le ligand qui interfère et $K_\textsc{l}$ sa constante de dissociation, alors S$_{\text{O}_2}$ devient :
    $$\mathrm{S_{O_2}} = \dfrac{P_{\text{O}_2}/P_{50}}{1 + P_{\text{O}_2}/P_{50} + \mathrm{\dfrac{[L\cdot HmNH_2]}{[HmNH_2]}}} = \dfrac{P_{\text{O}_2}/P_{50}}{1 + P_{\text{O}_2}/P_{50} + \dfrac{a_\textsc{l}}{K_\textsc{l}}},$$
    $a_\textsc{l}$ étant l'activité de L en solution, $\mathrm{[CN^-]}$ pour le cynaure ou $P_\mathrm{CO_2}/\kappa_\mathrm{CO_2}$ pour le CO$_2$.
    
    On peut ensuite décliner un modèle analogue à celui de O$_2$ pour l'influence du pH sur la coordination de L.
\end{questions}
\end{solution}