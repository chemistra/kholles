% Niveau :      PCSI *
% Discipline :  Géné
% Mots clés :   Diagrammes E-pH

\begin{exercise}{Diagrammes de Pourbaix}{2}{PCSI}
{Chimie générale, Réactions acidobasiques, Oxydoréduction, Diagrammes E-pH}{bermu}

\paragraph{Consignes}~\\
\textsl{Vous tracerez les diagrammes ...}

\paragraph{Convention de tracé :}
\begin{itemize}
    \item les espèces en solution ont pour activité $c_\text{tr}/c^\circ = 10^{-5}$,
    \item les espèces gazeuses ont pour activité $p_\text{tr}/p^\circ = 10^{-2}$,
    \item les espèces solides ont pour activité $1$.
\end{itemize}

\begin{questions}
\questioncours Tracer le diagramme $E$-pH de l'eau.

\end{questions}

\plusloin
Dash, R.K. \emph{et al, Ann Biomed Eng} \textbf{32}, 1676--1693 (2004).

\end{exercise}

\begin{solution}
\begin{questions}
    \questioncours
    \begin{itemize}
        \item Analogies : ces réactions de dissociation sont de manière générale
        $$\mathrm{XL_\textit{n} \leftrightharpoons X L_\textit{n-1} + L},$$
        et dont la constante est
        $$K_d = \mathrm{\dfrac{[XL_\textit{n-1}][L]}{[X L_\textit{n}]}} \quad \Longleftrightarrow \text{p}K_d = -\log K_d = \text{pL} + \log\mathrm{\dfrac{[XL_\textit{n-1}]}{[X L_\textit{n}]}}.$$
        \begin{center}\begin{tabular}{c|cc}
             & \textbf{Réactions acido-basiques} & \textbf{Réactions de coordination} \\ \hline\hline
            L & H$^+$ & Ligand \\
            XL$_n$ & Acide & Complexe \\
            XL$_{n-1}$ & Base & Complexe \\
        \end{tabular}\end{center}  
        
        \item La principale différence entre les réactions acide-base et de complexation est que les complexes peuvent dismuter ce qui n'est pas le cas pour les ampholytes car on a toujours
        $$\text{p}K_{a1} > \text{p}K_{a2} > \ldots,$$
        ce qui n'est pas le cas pour les complexes.
    \end{itemize}
    
    \question \hfill $K_d = \mathrm{\dfrac{[HmNH_2][O_2]}{[O_2\cdot HmNH_2]}}.$ \hfill ~
    
    \question Et de même \hfill $K_a = \mathrm{\dfrac{[HmNH_2][H^+]}{[HmNH_3^+]}}.$ \hfill ~
    
    \question La constante de la réaction $\mathrm{HmNH_2 \longrightarrow O_2\cdot HmNH_3^+}$ étant $\gamma K_a K_d$, la constante de la réaction $\mathrm{HmNH_3^+ \longrightarrow O_2\cdot HmNH_3^+}$ est $\gamma K_a$. \\
    $1/\gamma$ est donc le facteur indiquant à quel point la coordination du dioxygène O$_2$ par l'hème est moins efficace lorsque celle-ci est protonée.
    
    \question Et comme $\gamma < 1$, une augmentation de $\mathrm{[H^+]}$ / diminution de pH défavorise la coordination de O$_2$ par l'hème (souvent, H$^+$ défavorise les réactions de coordination).
    
    \question $K_\text{eff} = \mathrm{[O_2] \times \dfrac{[HmNH_2] + [HmNH_3^+]}{[O_2\cdot HmNH_2] + [O_2\cdot HmNH_3^+]}}
    = K_d \times \mathrm{\dfrac{[O_2\cdot HmNH_2] + \gamma [O_2\cdot HmNH_3^+]}{[O_2\cdot HmNH_2] + [O_2\cdot HmNH_3^+]}}$
    $$\text{d'où} \qquad K_\text{eff} = K_d \dfrac{1 + \dfrac{[H^+]}{K_a}}{1 + \dfrac{1}{\gamma}\,\dfrac{[H^+]}{K_a}}$$
    
    \question La S$_{\text{O}_2}$ représente la fraction d'hémoglobine qui transporte effectivement du O$_2$. C'est le marqueur biologique de l'efficacité de la respiration.
    
    \question $$\mathrm{S_{O_2}} = \dfrac{\mathrm{[O_2]}/K_\text{eff}}{1 + \mathrm{[O_2]}/K_\text{eff}} = \dfrac{P_{\text{O}_2}/P_{50}}{1 + P_{\text{O}_2}/P_{50}},$$
    avec $P_{50} = K_\text{eff} / \kappa_\textsc{h}$, la pression partielle en O$_2$ nécessaire à avoir une S$_{\text{O}_2}$ à 50\%.
    
    \question L'hème peut également complexer le CO$_2$ et CN$^-$ et va donc faire diminuer la S$_{\text{O}_2}$.
    
    Notons de manière générale L le ligand qui interfère et $K_\textsc{l}$ sa constante de dissociation, alors S$_{\text{O}_2}$ devient :
    $$\mathrm{S_{O_2}} = \dfrac{P_{\text{O}_2}/P_{50}}{1 + P_{\text{O}_2}/P_{50} + \mathrm{\dfrac{[L\cdot HmNH_2]}{[HmNH_2]}}} = \dfrac{P_{\text{O}_2}/P_{50}}{1 + P_{\text{O}_2}/P_{50} + \dfrac{a_\textsc{l}}{K_\textsc{l}}},$$
    $a_\textsc{l}$ étant l'activité de L en solution, $\mathrm{[CN^-]}$ pour le cynaure ou $P_\mathrm{CO_2}/\kappa_\mathrm{CO_2}$ pour le CO$_2$.
    
    On peut ensuite décliner un modèle analogue à celui de O$_2$ pour l'influence du pH sur la coordination de L.
\end{questions}
\end{solution}