% Niveau :      PCSI *
% Discipline :  Chimie Orga
% Mots clés :   Stéréochimie

\begin{exercise}{Horloge chimique}{3}{PCSI}
{Transformationn de la matière,Cinétique,Arrhénius}{bermu}

\begin{questions}
\questioncours Approximation de l’étape cinétiquement déterminante. \\ On illustrera par un exemple.

\begin{EnvUplevel}
     Hans Heinrich Landolt découvre en 1886 une réaction qu'il qualifie d'horloge chimique.
     
     Partant d'une solution contenant des ions iodure (I$^-$, $\mathrm{[I]_i} = 10$ mmol) et thiosulfate (S$_2$O$_3^{2-}$, $\mathrm{[S_2O_3^{2-}]_i} = 10$ mmol) et de l'amidon (noté L, $\mathrm{[L]_i} = 100$ mmol), on ajoute de l'eau oxygénée (H$_2$O$_2$) acidifiée (H$^+$). Les variations de volumes sont négligées.
     
     On observe que la solution, limpide initialement, devient violette au bout de $\tau = 5$ secondes.
     
     Il se produit trois réactions :
     \begin{enumerate}[~\quad(1)]
         \item $\mathrm{3 I^- + H_2O_2 + 2 H^+ \quad\longrightarrow\quad I_3^- + 2 H_2O}$, lente, $K^\circ_1 = 10^{30}$ ;
         \item entre I$_3^-$ et S$_2$O$_3^{2-}$ qui produit I$^-$ et S$_4$O$_6^{2-}$, rapide, $K^\circ_2 = 10^{50}$ ;
         \item $\mathrm{I_3^- + L \quad\rightleftharpoons\quad [I_3L]^-}$, qui colore la solution en violet, rapide, $K^\circ_3 = 10^{20}$.
     \end{enumerate}
\end{EnvUplevel}

\question Initialement, à $t=0$, seule la réaction (1) se produit.
\begin{parts}
    \part Que veux-t-on dire par 'la réaction est totale et lente' ? Donner des critères quantifiant ces deux affirmations.
    \part Sachant que la concentration en eau oxygénée et en ions H$^+$ est très grande (on précisera devant quoi), donner une expression simplifiée de la loi de vitesse de la réaction 1. On pourra considérer qu'elle est d'ordre effectif 1.
    \part Donner la concentration de $\mathrm{[I_3^-]_1}(t)$ en fonction du temps, et des données du problème, compte tenu uniquement de la réaction (1). Estimer  la constante de vitesse de réaction effective $k^\text{eff}_1$ ainsi que la vitesse de réaction initiale $v^\text{i}_1$ à l'aide des données du problème.
\end{parts}

\question On considère qu'il s'est initialement formé par la réaction (1) une petite quantité $\varepsilon$ de I$_3^-$. \\
Il est immédiatement consommé par les réactions (2) et (3). 
\begin{parts}
    \part Équilibrer l'équation de réaction de (2).
    \part En précisant les hypothèses utilisées et en les justifiant par des calculs rigoureux, donner l'état final du système à l'instant $t$.
    
    \textsfbf{Aide :} considérer le rapport des avancements volumiques $\xi_2^\text{i,éq}/\xi_3^\text{i,éq}$ à l'instant initial.
    
    \part Comment évolue donc le système initialement ? Donner l'évolution (totale) de $\mathrm{[I^-]}(t)$, $\mathrm{[I_3^-]}(t)$ et $\mathrm{[S_2O_3^{2-}]}(t)$ en fonction des données du problème et de $v_1^\text{i}$.
\end{parts}

\question Le système évolue ainsi jusqu'à l'instant $\tau$.
\begin{parts}
    \part Identifier les hypothèses mises en défaut à cet instant.
    
    \part Quelle est donc la valeur de $k_1$ en fonction de $\tau$ ?
    
    \part Comment modéliser le système pour $t > \tau$ ?
    
\end{parts}


\end{questions}

\end{exercise}