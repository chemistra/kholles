% Niveau :      PCSI *
% Discipline :  Chimie Orgam
% Mots clés :   Ballistique, Mécanique du point, PFD, Chute libre

\begin{exercise}{Régiosélectivité des organométalliques}{2}{PCSI}
{Chimie organique I, AN, organométallique}{bermu}

\begin{questions}
\questioncours Structure et propriétés chimiques des organomagnésiens mixtes.

\begin{EnvUplevel}
On étudie la réaction de différents composés organométalliques sur la pent-3-èn-2-one \textbf{A} :
\begin{center}
    \schemestart
        \textbf{A}
        \arrow(--[xshift=-3.5em]){->[MeMgBr][Et$_2$O, reflux]}[,2.1]
        \mbox{}%
        \arrow{->[H$^+$][H$_2$O, $0^\circ$C]}[,1.9]
        \chemname{\chemfig{C_6H_{12}O}}{\textbf{B}}
    \schemestop\chemnameinit{}
    \end{center}
\end{EnvUplevel}

\question Donner la structure de \textbf{B} et le mécanisme de la réaction. \\
On précisera la structure de l'intermédiaire réactionnel formé avant l'hydrolyse finale ?

\question Commenter les conditions expérimentales. Quelles précautions expérimentales faut-il prendre lors de cette synthèse ?

\question Donner rapidement le mode de formation de MeMgBr.

\question Quel sont les deux effets de l'hydrolyse finale ?

\uplevel{Lors de cette réaction, on observe la présence d'un produit supplémentaire \textbf{C} isomère de chaîne de \textbf{B}.}

\question Identifier le second site électrophile de \textbf{A} et donner la structure de \textbf{C}.

\question Donner le mécanisme de la première étape de la formation de \textbf{C}.

\begin{EnvUplevel}
Il exsite d'autres composés organometalliques qui ont un comportement analogue à celui des organomagnésiens dont les propriétés varient en fonction du métal M employé lors de la réaction. Ainsi, lors de la réaction précédente, la proportion \textbf{B} : \textbf{C} varie comme l'indique la table ci-dessous.

\begin{table}[H]
    \centering
    \begin{tabular}{rlccr}
        Organométallique & CH$_3$--M & \'Electroneg.$^\ast$ & \% ion. C--M & \textbf{B} : \textbf{C} \\ \hline\hline
        Organopotassique      & CH$_3$--K    & 0,82 & 53 & $\sim 1:0$~ \\
        Organosodique         & CH$_3$--Na   & 0,93 & 48 & $99:1$~ \\
        Organolithien         & CH$_3$--Li   & 0,98 & 46 & $98:2$~ \\
        Organomagnésien mixte & CH$_3$--MgBr & 1,55 & 32 & $86:14$ \\
        Organozincique mixte  & CH$_3$--ZnBr & 1,65 & 32 & $72:28$ \\
        Organocuprate lithié  & CH$_3$--CuLi & 1,90 & 10 & $\sim 0:1$~ \\ \hline
    \end{tabular}
    \begin{flushleft}
        \small $^\ast$ \'Electronégativité de M au sens de Pauling, à comparer avec celle du carbone : 2,55.
    \end{flushleft}
    \caption{Propriétés et sélectivité de différents composés organométalliques.}
\end{table}
\end{EnvUplevel}

\question Interpréter à l'aide des données ci-dessus l'évolution du caractère ionique de la liaison carbone--métal (C--M) des différents composés organométalliques.

\question Commenter la sélectivité de ces composés. Les organomagnésiens sont-ils sélectifs ?


\end{questions}
\end{exercise}