% Niveau :      PCSI *
% Discipline :  Chimie Orga
% Mots clés :   Infrarouge

\begin{exercise}{Spectroscopies d'absorption}{0}{PCSI}
{Chimie organique I,Spectroscopie}{bermu}

\begin{questions}
\question Rappelez brièvement le principe physique des spectroscopies d'absorption et d'émission.

\uplevel{Par la suite, nous allons évoquer différents types de spectroscopies. On les placera sur un diagramme ayant pour axe la fréquence / la longueur d'onde / l'énergie des transitions employées.}

\question On considère tout d'abord une molécule monoatomique (\emph{e.g.} He).
\begin{parts}
    \part Quelles transitions énergétiques y a-t-il pour une telle molécule ?
    \part Placer sur le diagramme l'ordre de grandeur des fréquences en jeu.
    \part Quelle loi connaissez-vous concernant les niveaux d'énergie de ces transitions ?
    \part Donnez un ou des exemple$\cdot$s de spectroscopie$\cdot$s mettant en jeu ces transitions.
    \part Quel$\cdot$s objet$\cdot$s du quotidien met$\cdot$tent en jeu de telles transitions ?
\end{parts}
\question On considère à présent une molécule polyatomique (\emph{e.g.} $\mathrm{NO_2}$). Quels nouveaux types de transitions faut-il considérer ? Reprendre les étapes de la question précédente pour chaque type de transition.

\end{questions}

\end{exercise}