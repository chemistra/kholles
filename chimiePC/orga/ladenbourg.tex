% Niveau :      PCSI *
% Discipline :  Chimie Orga
% Mots clés :   Stéréochimie

\begin{exercise}{Benzène de Ladenburg}{2}{PCSI}
{Chimie organique I,Stéréochimie}{jbb}

\begin{questions}
\questioncours Répondre aux questions suivantes en redéfinissant les notions abordées et en les illustrant par un (des) exemple(s) pertinent(s).
\hspace*{-1ex}\begin{tabbing}
    \textsf{\bfseries a) } Une molécule qui possède (ne possède pas) un axe de symétrie \hspace{3ex} \= est-elle (a)chirale ? \\
    \textsf{\bfseries b) } Une molécule qui possède (ne possède pas) un plan de symétrie \> est-elle (a)chirale ? \\
    \textsf{\bfseries c) } Une molécule qui possède (ne possède pas) un centre de symétrie \> est-elle (a)chirale ? \\
    \textsf{\bfseries d) } Quel est le lien entre les stéréodescripteurs L,D et R,S ?
\end{tabbing}


\begin{EnvUplevel}
    Au XIX\textsuperscript{ème} siècle, la structure du benzène était un mystère pour les chimistes. Seule la formule brute $\mathrm{C_6H_6}$ était connue. En 1874, Ladenburg a proposé de présenter la structure du benzène sous forme d'un prisme droit à base triangulaire dont chaque sommet était un atome de carbone lié à un atome d'hydrogène.
\end{EnvUplevel}

\question Représenter le benzène de Ladenburg sous forme semi-développée.

\question Les benzènes disubstitués $\mathrm{C_6H_4X_2}$, pour lequel on a remplacé deux liasons C--H par une liaison C--X, existent sous forme de trois isomères isolés expérimentalement.
\begin{parts}
    \part Montrer que la structure proposée est compatible avec ce fait et donner les isomères en question.
    \part Comment peut-on isoler expérimentalement ces isomères ?
\end{parts}

\question En 1876, Van't Hoff a critiqué la structure de Ladenburg en indiquant qu'elle implique la présence d'énantiomères, ce qui est en contradiction avec les observations expérimentales effectuées jusqu'alors.
\begin{parts}
    \part Montrer que pour la structure proposée, les isomères du benzène disubstitué $\mathrm{C_6H_4X_2}$ donnent lieu à la formation d'énantiomères.
    \part Combien y a-t-il d'isomères (au sens large) au total ? \part Pourquoi certains isomères ne donnent pas lieu à des énantiomères ?
    \part Comment peut-on expérimentalement séparer ces isomères ?
    \part Conclure quant à l'objection de Van't Hoff.
\end{parts}

\question Reprendre la question précédente avec un benzène disubstitué de deux groupes différents, \\
X et Y : $\mathrm{C_6H_4XY}$.

\question Rappeler la vraie structure du benzène (formule de Kékulé) et vérifier pour cette formule que toutes les observations expérimentales sont compatibles.
\paragraph{Rappel :} le benzène est un cyclohexane dont les liaisons sont par alternance simples et doubles.

\end{questions}

\plusloin Pouvez-vous penser à d'autres structures que celles de Kékulé et Ladenbourg ?

\end{exercise}