% Niveau :      PCSI *
% Discipline :  Chimie Orgam
% Mots clés :   Ballistique, Mécanique du point, PFD, Chute libre

\begin{exercise}{Réarrangement de Wagner--Meerwein}{3}{PCSI}
{Chimie organique I, SN, E, Réarrangement, Activation}{bermu}

\begin{questions}
\questioncours Intérêt synthétiques et méthodes d'activation électrophile des alcools.

\begin{EnvUplevel}
    L'activation électrophile des alcools en milieu acide peut être suivie d'un mécanisme de d'élimination monomoléculaire E1 en deux étapes :
    \begin{itemize}
        \item la première analogue à celle d'une $\mathrm{S_N1}$,
        \item la seconde analogue à une E2.
    \end{itemize}
\end{EnvUplevel}

\question \`A l'aide de vos connaissances et de la question précédente proposer un mécanisme d'une E1 sur le 2-méthylpropan-2-ol dans la l'acide sulfurique concentré. \\
On pretera bien attention au fait qu'on se trouve en milieu acide.

\question Quel est le critère cinétique ou thermodynamique qui détermine si une E1 se fait spontanément ?

\begin{EnvUplevel}
    On s'intéresse à présent à la réaction des alcools dérivés du bornane
    \begin{center}
        \chemfig{?[a](!\Cnumo{6}2)-[:10]?[b,1,over line](!\Cnumo{-3}1)(-[:-140](!\Cnumo{:-140}1'{}))-[:-10](!\Cnumo{0}6)-[:50](!\Cnum5)-[:170](!\Cnumo{1}4)(-[:120,1.1]?[b](!\Cnumo{3}7)(-[1](!\Cnum7''{}))-[5](!\Cnumo{6}7'{}))-[:-170,.55]-[:-170,.2,,,draw=none]-[:-170,.25]?[a](!\Cnumo{6}3)}
    \end{center}
    dans l'acide sulfurique concentré.
\end{EnvUplevel}

%\question Combien y a-t-il de monoalcools dérivés du bornane (en prenannt en compte les stéréoisomères) ? Ne pas les représenter.

\question Justifier les résultats expérimentaux suivants :
\begin{parts}
    \part L'alcool substitué sur le carbone 4 ne réagit pas du tout.
    \part Les alcools substitués sur les carbones 1', 7' et 7'' réagissent peu.
\end{parts}

\question Quel est le produit formé par la réaction de l'alcool substitué sur le carbone 2 (ou 6) ?

\begin{EnvUplevel}
Le produit formé par cette réaction est en fait le camphène :
    \begin{center}
        \chemfig{?[a]-[:10]?[b,1,over line]-[:-10](=[-2])-[:50](-[-3])(-[:10])-[:170](-[:120,1.1]?[b])-[:-170,.55]-[:-170,.2,,,draw=none]-[:-170,.25]?[a]}
    \end{center}
\end{EnvUplevel}
    \question Proposez un mécanisme pour expliquer cette réaction.\\
    Justifiez la nomenclature de réarrangement 1,2 pour cette réaction.
\end{questions}

\end{exercise}

\begin{solution}
\begin{questions}
\questioncours Les alcools ne sont pas de bons nucléofuges. Pour les transformer en bon groupe partants (activation électrophile), on peut :
    \begin{itemize}
        \item ajouter un H$^+$ dans un acide fort : R-OH + H$^+$ = R-H$_2$O$^+$
        \item remplacer le OH par un OR très encombré, par exemple avec des sulfonyles (tosylate, mésylate), qui ont l'avantage d'être des bases faibles et des mauvais nucléophiles.
    \end{itemize}

\question En milieu acide, il y a tout d'abord une activation électrophile du brome (Q1) : \\[1ex]

\hfill\schemestart
        \chemfig{-[1](-[:-109])(-[-1])-[3]@{br}\lewis{024,Br}}
        \arrow{->[\chemfig{@{h}\lewis{4|,H}^\oplus}][]}
        \chemfig{-[1](-[:-109])(-[-1])-[3]\lewis{24,Br}^\oplus-[0,1.3]H}
        \chemmove[red,-stealth,red,shorten <= 2pt, shorten >= 4pt]{
            \draw(br)..controls +(up:10mm) and +(west:10mm).. (h);}
\schemestop\hfill~

Puis la E1 :\hfill\schemestart
        \chemfig{-[1](-[:-109])(-[-1])-[@{a1}3,,,,red]@{b1}\lewis{24,Br}^\oplus-[0,1.3]H}
        \arrow{->[][$-$ HBr]}
        \chemfig{-[2](-[6,0.2,,,draw=none]@{b2}\chemabove{\lewis{4|,}}{\quad\scriptsize\oplus})(-[4])-[0]-[@{a2}-2,,,,red]H}
        \chemmove[red,-stealth,red,shorten >= 5pt]{
            \draw(a1)..controls +(west:3mm) and +(south west:3mm).. (b1);
            \draw(a2)..controls +(west:3mm) and +(south east:3mm).. (b2);}
        \arrow{->}
        \chemfig{-[2](-[4])=[0]}
    \schemestop\chemnameinit{}\hfill~

\question ~
\begin{parts}
    \part La formation du carbocation en C$_4$ est impossible à cause des contraintes angulaires cycle.
    \part Les carbocations 1' et 7' sont peu stables car primaires.
\end{parts}

\question Si on suit le mécanisme précédent on obtient la structure (peu stable) suivante

\hfill\schemestart
        \chemfig{?[a](-[:170]@{oh}O-[6,.4,,,draw=none]H)-[:10]?[b,1,over line](-[:-140])-[:-10]-[:50]-[:170](-[:120,1.1]?[b](-[1])-[5])-[:-170,.55]-[:-170,.2,,,draw=none]-[:-170,.25]?[a]}
        \arrow{->[\chemfig{@{h}\lewis{4|,H}^\oplus}][]}
        \chemmove[red,-stealth,red,shorten <= 2pt, shorten >= 4pt]{
            \draw(oh)..controls +(south east:20mm) and +(west:10mm).. (h);}
        \chemfig{?[a](-[6,0.2,,,draw=none]@{b1}\chemabove{\lewis{4|,}}{\quad\scriptsize\oplus})-[:10]?[b,1,over line](-[:-140])-[:-10]-[:50]-[:170](-[:120,1.1]?[b](-[1])-[5])-[:-170,.55]-[:-170,.2,,,draw=none]-[:-170,.25]?[a]-[@{a1}:170,,,,red]H}
        \chemmove[red,-stealth,red,shorten >= 3pt]{
            \draw(a1)..controls +(south west:2mm) and +(west:2mm).. (b1);}
        \arrow{->[][$-$ H$^\oplus$]}
        \chemfig{(=[:50]?[a])-[:10]?[b,1,over line](-[:-140])-[:-10]-[:50]-[:170](-[:120,1.1]?[b](-[1])-[5])-[:-170,.55]-[:-170,.2,,,draw=none]?[a]}
\schemestop\hfill~

    \question Il y a une délocatisation électronique (ce qu'on appelle un réarrangement) :
    
\hfill\schemestart
        \chemfig{?[a](-[:170]@{oh}O-[6,.4,,,draw=none]H)-[:10]?[b,1,over line](-[:-140])-[:-10]-[:50]-[:170](-[:120,1.1]?[b](-[1])-[5])-[:-170,.55]-[:-170,.2,,,draw=none]-[:-170,.25]?[a]}
        \arrow{->[\chemfig{@{h}\lewis{4|,H}^\oplus}][]}
        \chemmove[red,-stealth,red,shorten <= 2pt, shorten >= 4pt]{
            \draw(oh)..controls +(south east:20mm) and +(west:10mm).. (h);}
        \chemfig{?[a](-[6,0.2,,,draw=none]@{b1}\chemabove{\lewis{4|,}}{\quad\scriptsize\oplus})-[:10]?[b,1,over line](-[:-140])-[@{a1}:-10,,,,red]-[:50]-[:170](-[:120,1.1]?[b](-[1])-[5])-[:-170,.55]-[:-170,.2,,,draw=none]-[:-170,.25]?[a]}
        \chemmove[red,-stealth,red, shorten >= 3pt]{
            \draw(a1)..controls +(south west:2mm) and +(south:2mm).. (b1);}
        \arrow{->[][réa$^\text{t}$]}
        \chemfig{?[a]?[c]-[:10]?[b,1,over line](-[:-140])(-[0,0.3,,,draw=none]@{b1}\chemabove{\lewis{4|,}}{\quad\scriptsize\oplus})-[:-10,,,,draw=none]?[c]-[:50]-[:170](-[:120,1.1]?[b](-[1])-[5])-[:-170,.55]-[:-170,.2,,,draw=none]-[:-170,.25]?[a]}
\schemestop\hfill~

Formant le carbocation plus stable
        \chemfig{?[a]?[c]-[:10]?[b,1,over line](-[:-140])(-[0,0.3,,,draw=none]@{b1}\chemabove{\lewis{4|,}}{\quad\scriptsize\oplus})-[:-10,,,,draw=none]?[c]-[:50]-[:170](-[:120,1.1]?[b](-[1])-[5])-[:-170,.55]-[:-170,.2,,,draw=none]-[:-170,.25]?[a]} qui est identique à \chemfig{?[a]-[:10]?[b,1,over line]-[:-10](-[0,0.3,,,draw=none]@{b1}\chemabove{\lewis{4|,}}{\quad\scriptsize\oplus})(-[-2])-[:50](-[-3])(-[:10])-[:170](-[:120,1.1]?[b])-[:-170,.55]-[:-170,.2,,,draw=none]-[:-170,.25]?[a]},
  
\hfill\schemestart
        \chemfig{?[a]-[:10]?[b,1,over line]-[:-10](-[0,0.3,,,draw=none]@{b1}\chemabove{\lewis{4|,}}{\quad\scriptsize\oplus})(-[-2]-[@{a1}0,,,,red]H)-[:50](-[-3])(-[:10])-[:170](-[:120,1.1]?[b])-[:-170,.55]-[:-170,.2,,,draw=none]-[:-170,.25]?[a]}
        \arrow{->[][$-$ H$^\oplus$]}
        \chemfig{?[a]-[:10]?[b,1,over line]-[:-10](=[-2])-[:50](-[-3])(-[:10])-[:170](-[:120,1.1]?[b])-[:-170,.55]-[:-170,.2,,,draw=none]-[:-170,.25]?[a]}
        \chemmove[red,-stealth,red, shorten >= 3pt]{
            \draw(a1)..controls +(south west:2mm) and +(south:2mm).. (b1);}
\schemestop\hfill~      

\end{questions}

\end{solution}