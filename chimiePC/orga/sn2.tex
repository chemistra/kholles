% Niveau :      PCSI *
% Discipline :  Chimie Orgam
% Mots clés :   Ballistique, Mécanique du point, PFD, Chute libre

\begin{exercise}{Aspect cinétique des SN}{2}{PCSI}
{Chimie organique I, SN}{bermu}

\begin{questions}
\questioncours Donnez les mécanismes réactionnel de la $\mathrm{SN_1}$ et de la $\mathrm{SN_2}$ et comparez dans un tableau les différentes propriétés associées (controle, sélectivité, nature des réactifs, influence du solvant, de la température, de la concentration ...).

\uplevel{Par la suite, on chercher à tirer sur un projectile $M$ avec la balle $m$.}

\end{questions}
\end{exercise}