% Niveau :      PCSI *
% Discipline :  Chimie Orgam
% Mots clés :   Ballistique, Mécanique du point, PFD, Chute libre

\begin{exercise}{Excès énantiomérique}{1}{PCSI}
{Chimie organique I,Polarimétrie,Excès énantiomérique,Pouvoir optique}{poublanc}
\index{\'Enantiomère}

\begin{questions}
\questioncours Loi de Biot. On fera bien attention de préciser les dépendances implicites des paramètres et les hypothèses de validité de la loi.

\begin{EnvUplevel}
    L'époxydation de Sharpless (prix Nobel de chimie 2001) permet la formation d'époxydes à partir d'alcools allyliques, selon le schéma réactionnel :
    
    Dans cette réaction l'excès énantiométrique est $\mathrm{e.e.} = 94 \%$.
    
    \paragraph{Rappel :}en notant $[\textsfbf{R}]$ et $[\textsfbf{S}]$ les concentrations respectives des énantiomères R et S, on définit l'excès énantiomérique comme
    $$\mathrm{e.e.} = \abs{\dfrac{[\textsfbf{R}] - [\textsfbf{S}]}{[\textsfbf{R}] + [\textsfbf{S}]}}.$$
\end{EnvUplevel}

\question Entre quelles bornes l'excès énantiométrique $\mathrm{e.e.}$ peut varier ? Dans quelles cas les bornes sont atteintes ?
\question Déterminer les pourcentages des composés \textsfbf{R} et \textsfbf{S} dans le mélange de produits ? Commenter.

\begin{EnvUplevel}
    Afin de mesurer l'excès énantiométrique et donc la stéréosélectivité de la aréaction, on utilise la polarimétrie. En notant $\alpha_\text{max} >0$ la valeur absolue du pouvoir rotatoire d'une solution énantiomériquement pure à la concentration $C_1$, on définit le pouvoir optique $\mathrm{p.o.}$ comme le rapport
    $$\mathrm{p.o.} = \abs{\dfrac{\alpha}{\alpha_\text{max}}}.$$
\end{EnvUplevel}

\question Etablir la relation entre l'excès énantiométrique $\mathrm{e.e.}$ et le pouvoir optique $\mathrm{p.o.}$ Quel est le pouvoir optique de cette solution ?


\question Sous quelles conditions la donnée expérimentale du pouvoir optique p.o. permet-elle de remonter à l'excès énantiomérique. On prendra gare au fait que les angles sont définis à $2\pi$ près.

\question Comment généraliser ces résultats avec des diastéréoisomères ?

\end{questions}
\end{exercise}