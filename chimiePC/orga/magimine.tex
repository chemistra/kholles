% Niveau :      PCSI *
% Discipline :  Chimie Orgam
% Mots clés :   Ballistique, Mécanique du point, PFD, Chute libre

\begin{exercise}{Addition des organomagnésiens}{2}{PCSI}
{Chimie organique I, AN, organométallique}{bermu}

\begin{questions}
\questioncours Décrire la réaction du bromure de propylemagnésium \textbf{A} sur la butanone \textbf{B}$_1$ pour donner le produit \textbf{C}$_1$. On précisera :
\begin{itemize}
    \item \textbf{C}$_1$ et sa stéréochimie ;
    \item les conditions expérimentales (et les précautions à prendre) ;
    \item le mécanisme réactionnel.
\end{itemize}

\begin{EnvUplevel}
On se propose d'étudier la réaction de \textbf{A} sur \textbf{B}$_2$, l'imine suivante :
\begin{center}
    \schemestart
        \chemname{\chemfig{-[1](=[3]N-[1])-[-1]}}{\textbf{B}$_2$}
        \arrow(--){->[1. \textbf{A} / Et$_2$O][2. H$^+$ / H$_2$O]}[,2.2]
        \chemname{\chemfig{C_6H_{15}N}}{\textbf{C}$_2$}
    \schemestop\chemnameinit{}
    \end{center}
\end{EnvUplevel}


\question Par analogie avec les questions précédentes, donner la structure de \textbf{C}$_2$ et le mécanisme de la réaction.

\question Même question pour l'addition de \textbf{A} sur la carboglace \textbf{B}$_3$ (CO$_\text{2, (g)}$) : quel$\cdot$s produit$\cdot$s peut-on envisager ?

\question Proposer une voie de synthèse détaillée de l'acide phényle-sulfinique Ph--SO$_2$H.

\question Un cas particulier : pour l'addition du propyne \textbf{B}$_4$ sur \textbf{A}, on obtient un nouvel organomagnésien \textbf{C}$_4$, et un alcène. Donner la structure de \textbf{C}$_4$.

\question Proposer une voie de synthèse pour \chemfig{-~-=[2]} à partir de \textbf{A}, \textbf{B}$_4$ et des réactifs que vous jugerez utiles.

\end{questions}
\end{exercise}