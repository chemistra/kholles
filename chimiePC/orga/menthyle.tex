% Niveau :      PCSI *
% Discipline :  Chimie Orgam
% Mots clés :   Ballistique, Mécanique du point, PFD, Chute libre

\begin{exercise}{Réactivité du chlorure de menthyle}{2}{PCSI}
{Chimie organique I, SN, E}{poublanc,bermu}

\begin{questions}
\questioncours Activation nucléophile des alcools : principe et intérêt synthétique.
%Sélectivité$\cdot$s et spécificité$\cdot$s de la réaction d'élimination sur les halogénoalcanes.

\begin{EnvUplevel}
    On s'intéresse à présent à la réactivité du chlorure de menthyle \textbf{A} et du chlorure de néomenthyle \textbf{B}.
    
        ~\hfill
        \chemfig{>:[-3]*6(---(-[-3,,,,Cram2Sides](-[7])(-[-3,1.2,,,draw=none]{\textbf{A}})-[-1])-(<:[-1,1.2]C\ell)--)}
        \hfill
        \chemfig{>:[-3]*6(---(-[-3,,,,Cram2Sides](-[7])(-[-3,1.2,,,draw=none]{\textbf{B}})-[-1])-(<[-1,1.2]C\ell)--)}
        \hfill~
\end{EnvUplevel}
    \question Quelle est la relation de stéréochimique entre ces deux isomères ?

\begin{EnvUplevel}
    Lors de la réaction de \textbf{A} et \textbf{B} avec l'éthanolate de sodium, on observe la formation de deux produits :
    
    ~\hfill\schemestart[][west]
        \textbf{A}, \textbf{B}
        \arrow{->[NaOEt][EtOH, $70^\circ$C]}[,2.1]
        \textbf{C} + \textbf{D},
    \schemestop\chemnameinit{}\hfill (R\arabic{exercise})
    
    \textbf{C} étant le produit ayant un stéréodescripteur RS, et \textbf{D} celui ayant un stéréodescripteur S.
\end{EnvUplevel}
    \question Quelle est la nature de la réaction précédente ? Donner la structure des produits formés.
    \question Commenter les conditions expérimentales.
\begin{EnvUplevel}
On observe que la vitesse de réaction et la proportion \textbf{C} : \textbf{D} varie selon que le substrat soit \textbf{A} ou \textbf{B}.
\begin{table}[H]
    \centering
    \begin{tabular}{rccc}
        Substrat & $k$ (rel.) & \textbf{C} (\%) & \textbf{D} (\%) \\ \hline\hline
        \textbf{A} & $10^{-2}$ & 99,9 & 0,01 \\
        \textbf{B} & $10^{0}$  & 25,4 & 74,6 \\ \hline
    \end{tabular}
    \caption{Constante de vitesse et proportion des produits obtenus lors de la réaction (R\arabic{exercise}).}
\end{table}
\end{EnvUplevel}
    \question Qualifier ce constat expérimental.
    \question Interpréter ces faits expérimentaux et écrire le mécanisme de la réaction dans chaque cas.
    \paragraph{Aide :} représenter \textbf{A} et \textbf{B} dans leur conformation la plus stable.\vspace{1.5em}
\end{questions}

\end{exercise}

    