\begin{exercise}{Mirages mirages}{2}{Sup}
{Réfraction, Optique géométrique}{bermudez}

\begin{questions}
    \question \textsf{Brève :} On considère un aquarium constitué d'une paroi de verre, d'indice $n_\text{v} = 1,5$ et d'épaisseur $e = 5$ mm, séparant l'air (considéré comme du vide d'indice 1) et l'eau de l'aquarium ($n_\text{e} = 1,33$). \\
    Donner le lien entre l'angle d'incidence côté aquarium et l'angle transmis coté air. Voit-on tout ce qui se passe dans l'aquarium ?
    
    \uplevel{On s'intéresse maintenant à un type de mirage similaire lié aux différences de température. L'indice de l'air $n$ est lié à la température $T$ par la loi de Gladstone qui s'énnonce comme suit :
    $$n-1 = K/T,$$ avec $K = 82$ mK.}
    
    \question \textsf{Culture sciences--physiques :} Expliquer le phénomène de mirage.
    \question On suppose que la température près du sol $T_\text{s}$ (entre $z=0$ et $z=h$) est plus élevée que la température de l'air $T_\text{a}$ ($z>h$). Pour un observateur dont les yeux sont à une hauteur $H > h$ au dessus du sol, déterminer à quelle distance minimale se situe le mirage.
\end{questions}

\end{exercise}
