\begin{exercise}{Miroirs de Lloyd}{2}{Spé}
{Interference, Michelson}{lelay}

On considère un miroir horizontal plan $M$. Un écran est placé orthogonalement à $M$, et une source lumineuse ponctuelle et monochromatique $S$ est placée à un distance $D$ de l'écran et à une distance $\frac{a}2$ du miroir.

\begin{questions}
    \questioncours Différence de marche, interférences (on illustrera avec un exemple du cours)
    \question Faire un schéma
    \question Les interférences sont obtenues par superposition sur l'écran de la lumière issue de $S$ et de la lumière réfléchie par le miroir. Quelles sont les deux sources dont sont issues les interférences ?
    \question Ces sources sont-elles cohérentes ? On rappelle que la réflexion d'une onde électromagnétique sur un miroir s'accompagne d'un déphasage de $\pi$ (est-ce un retard ou une avance de phase ?).
    \question Donner l'éclairement résultant sur l'écran. Quels points communs, quelles différences avec une expérience de trous d'Young ?
    \question Peut-on remplacer la source ponctuelle $S$ par une fente lumineuse allongée dans la direction commune au miroir et au plan sans dégrader la visibilité des franges ?
    \question (Difficile) Que se passe-t-il si on rajoute un second miroir à distance $a$ du premier, de telle sorte que la source se trouve au milieu des deux miroirs ? On répondra d'abord de manière qualitative, en faisant un parallèle judicieux avec le cours.
\end{questions}

\end{exercise} 