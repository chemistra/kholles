\begin{exercise}{Source étendue}{3}{Spé}
{Interference, Young}{bermudez,lelay}

On considère deux trous d'Young situés à distance $D$ d'un écran, éclairés par une source ponctuelle $S$ monochromatique de longueur d'onde $\lambda_0$ située dans le plan médian des trous à une distance $L$ de leur point milieu.

\begin{questions}
    \questioncours Trous d'Young. Décrire la figure d'interférence observée ainsi que la répartition de l'éclairement $E(x)$ sur l'écran.
    \uplevel{On remplace la source $S$ (de puissance $P_0$) par deux sources incohérentes $S_1$ et $S_2$ (chacune de puissance $P_0/2$) situées à une distance $\pm \frac{b}{2}$ de $S$ dans la direction verticale.}
    \question Déterminer l'expression de l'éclairement sur l'écran dans cette nouvelle situation. On mettra en valeur un terme nouveau, la visibilité $V$. Comment faut-il choisir $b$ pour avoir la figure d'interférence la plus nette possible (i.e. la meilleure visibilité ?
    \question On considère le cas d'une source étendue de puissance $P_0$ de taille $B$. Expliquer pourquoi il est équivalent de considérer un continuum de couples de sources infinitésimales incohérentes écartées de $b$, de puissance $\dd{P} = P_0 \frac{\dd{b}}{2B}$ avec $b$ variant entre $0$ et $B$.
    \question Quel est l'éclairement total de l'écran dans ce cas ? Comment choisir $B$ pour conserver la figure d'interférence la plus nette possible ?
    \question En réalité, il est impossible d'obtenir une source purement ponctuelle. Quelle est selon vous la taille maximale acceptable d'une source lumineuse pour réaliser une interférence avec des trous d'Young ? Expliquer pourquoi en pratique on réalise ces interférences en utilisant comme `source' un laser dirigé vers une feuille opaque percée d'un trou micrométrique.
    
    \uplevel{\paragraph{Variantes :} comment sera changé l'éclairement si considère :}
    \question Une source $S$ étendue 'en fréquence'
    \begin{itemize}
        \item émettant à 2 nombres d'onde $\sigma_1$ et $\sigma_2$ tels que $\frac{\sigma_1+\sigma_2}2 = \sigma_0 = \frac1{\lambda_0}$
        \item émettant un continuum de nombres d'onde entre $\sigma_1$ et $\sigma_2$.
    \end{itemize}
    
    \question Une fente d'Young de taille $\delta a$ de l'ordre de $\lambda$.
    
    \question Conclure quant aux différents termes de modulation de l'éclairement qu'on l'on peut observer dans cette expérience.
\end{questions}

\end{exercise}