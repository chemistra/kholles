\begin{exercise}{Atmosphère et couche d'Ozone}{-1}{Spé}
{Thermo}{centrale}

\textsl{Extrait adapté de Wikipédia, l'encyclopédie libre.}

\begin{center}\begin{minipage}{.9\textwidth}
L'ozone est un gaz de formule chimique O$_3$. C'est une molécule triatomique dont un des atomes d'oxygène est lié aux deux autres. L'ozone est une variété allotropique de l'oxygène, mais bien moins stable que le dioxygène O$_2$, en lequel il tend naturellement à se décomposer à température ambiante. La rapidité de la réaction dépend de la température, de l'humidité de l'air ou de la présence de catalyseurs.

L'ozone est naturellement présent dans l'atmosphère terrestre, formant dans la stratosphère une couche d'ozone entre 13 et 40 km d'altitude qui intercepte plus de 97 \% des rayons ultraviolets du Soleil, mais est un polluant dans les basses couches de l'atmosphère (la troposphère) où il est toxique pour les être vivants.

\`A 20 km d'altitude, les densités en O$_2$ et en O$_3$ sont respectivement de $4\times 10^{17}$ cm$^{-3}$ et de $2\times 10^{12}$ cm$^{-3}$.
\end{minipage}\end{center}

\paragraph{Données :}~\\
Mécanisme réactionnel du cycle de Chapman de formation et de décomposition de l'ozone :
\begin{align}
    \mathrm{O_3} & \underset{k_{-1}}{\overset{k_1}{\resizebox{4em}{1.5ex}{~$\rightleftharpoons$~}}} \mathrm{O_2 + O^{\bullet\bullet}} \tag{R1} \ \\
    \mathrm{O_2} & \overset{k_2}{\resizebox{4em}{.8ex}{~$\rightarrow$~}} \mathrm{2O^{\bullet\bullet}} \tag{R2}
    \\
    \mathrm{O_3 + O^{\bullet\bullet}} & \overset{k_3}{\resizebox{4em}{.8ex}{~$\rightarrow$~}} \mathrm{2O_2} \tag{R3}
\end{align}
On considérera que les concentrations des gaz et non leur pression partielle pour effectuer les calculs cinétiques et thermodynamiques chimiques.

On peut considérer que la réaction (R1) est un équilibre rapide et que la concentration en [O$^{\bullet\bullet}$] est négligeable devant les autres.

Données thermodynamiques à $300$ K :
\begin{center}\begin{tabular}{l|cc}
    & O$_2$ & O$_3$ \\ \hline
    $\Delta_\text{f}H^\circ$ ($\mathrm{kJ\cdot mol^{-1}}$) & --- & 142,12  \\
    $S_\text{m}$ ($\mathrm{J\cdot mol^{-1}\cdot K^{-1}}$) & 204,82 & 237,42 \\
    Energie de première ionisation (eV) & 12,07 & 12,43 \\ \hline\hline
\end{tabular}\end{center}

\paragraph{Résolution de problème :}~\\
Nous allons dans cet exercice justifier que l’ozone se forme dans les hautes couches de l’atmosphère.

\begin{questions}
    \question Donner la formule de Lewis de l'Ozone.
    \question\textsf{Modèle cinétique :} On appelle [Ox] = [O$^{\bullet\bullet}$] + [O$_3$].  Interpréter cette quantité.
    \question En utilisant les approximations adéquates, montrer que l'équation cinétique des Ox peut s'écrire
    $$\dv{\mathrm{[Ox]}}{t} = 2 k_2 \mathrm{[O_2]} - k\dfrac{\mathrm{[Ox]^2}}{\mathrm{[O_2]^2}},$$
    avec $k$ dont on précisera l'expression. Commenter ce taux en fonction de la température.
    \question\textsf{Modèle thermodynamique :} Quelle est l'équation globale de la transformation de O$_3$ et O$_2$ ? Dans quel cas peut-on considérer que l'on a atteint l'équilibre thermodynamique ?
    \question Le cas échéant, donner une relation entre [O$_3$], [O$_2$] et la température $T$. En déduire la tendance du O$_3$ a être détruit dans la basse atmosphère et produit dans la haute atmosphère.
\end{questions}

\end{exercise}
