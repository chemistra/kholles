\begin{exercise}{Cuisson des oeufs}{-1}{Spé}
{Thermo}{centrale}

La cuisson des oeufs est un art subtil ancré dans la tradition gastronomique française. De nos jours, les restaurateurs ont des autocuiseurs qui permettent d'avoir une température bien homogène.

Il existe trois principales cuissons des oeufs à l'eau bouillante dans leur coquille 
\begin{itemize}
    \item la cuisson \emph{à la coque}, pour laquelle l'extérieur du blanc est solide, mais pas l'intérieur ni le jaune ;
    \item la cuisson \emph{mollet}, pour laquelle le blanc est solide, mais pas le jaune ;
    \item la cuisson \emph{dure}, pour laquelle le blanc et le jaune sont solides.
\end{itemize}

\`A cela s'ajoute la question sanitaire de l'élimination des salmonelles.

\paragraph{Données :}~\\
L'oeuf peut être considéré en première approximation comme une sphère de rayon $R$ avec :
\begin{itemize}
    \item température de cuisson du blanc : $T_\text{b} = 62^\circ$C ;
    \item température d'élimination des salmonelles : $T_\text{s} = 64^\circ$C pendant 5 minutes ;
    \item température de cuisson du jaune : $T_\text{j} = 68^\circ$C ;
    \item chaleur latente de cuisson des protéines : $L = 1,7$ J$\cdot$kg$^{-1}$ ;
    \item un oeuf d'autruche pèse 1,5 kg environ.
\end{itemize}
Pour les autres données, on pourra considérer que l'oeuf est constitué d'eau liquide.

Pour l'eau :
\begin{itemize}
    \item capacité calorifique de l'eau $c = 4,2$ $\mathrm{J\cdot kg^{-1}\cdot K^{-1}}$ ;
    \item conductivité thermique de l'eau $\lambda = 0,60$ $\mathrm{W\cdot m^{-1}\cdot K^{-1}}$ ;
\end{itemize}

Les réfrigérateurs de la restauration sont généralement maintenus à $T_\text{0} = 4^\circ$C.

\paragraph{Résolution de problème :}~\\
Le restaurateur sort un oeuf du réfrigérateur et le met dans un bain marie maintenu au point d'ébullition de l'eau $T_\text{e}$ (à pression atmosphérique).

\begin{questions}
    \question Représenter qualitativement et à différents instants le profil de température $T(t,r)$ de l'intérieur de l'oeuf.
    \question Par analyse dimensionnelle et avec un bilan énergétique sommaire, évaluer le temps caractéristique $\tau$ de cuisson d'un oeuf dur pour une poule et pour une autruche.
    \question Justifier que l'on puisse négliger l'énergie de solidification des protéines pendant la cuisson.
    \question Retrouver l'équation de la chaleur vérifiée par $T(t,r)$ dans le système de coordonnées adaptées au système. On fera appaitre du temps caractéristique de cuisson $\tau$ dans l'expression et on indiquera les conditions initiales et aux bords vérifiées par $T(t,r)$.
    \question Adimensionner l'équation en introduisant le changement de variable suivant :
        \begin{align*}
            \Theta(s,z) &= \dfrac{T(t,r) - T_\text{e}}{T_\text{0} - T_\text{e}}, &
            z &= r/R, & s = t/T,
        \end{align*}
    \question Trouver l'équation vérifiée par $F(s,z) = z\Theta(s,z)$. Commenter.
%    \question Résoudre cette équation par séparation des variables $F(s,z) = S(s) Z(z)$. Montrer que l'on peut exprimer $\Theta$ sous la forme :
%        $$\Theta(s,z) = \sum_{n=1}^{\infty} \theta_n \dfrac{\sin (n \pi z)}{n \pi z} e^{-n\pi s},$$
%        $\theta_n$ étant des coefficients associés.
%    \question Retrouver les coefficients $\theta_n$ à partir des conditions initiales et donner la solution $T(t,r)$. \\
%        On utilisera que la fonction porte peut s'écrire comme suit :
%        $$\Pi(z) = \left\lbrace \begin{array}{ll}
%            1 & \text{si } -1 < z < 1 \\
%            0 & \text{sinon}
%        \end{array}\right. = -2\sum_{n=1}^{\infty} (-1)^n \dfrac{\sin (n \pi z)}{n \pi z}.$$
%   \question Conclure quant aux différents critères de cuisson de l'oeuf en fonction de $T_\text{e}$ et $T_0$ (considérer $r=0$).
    
\end{questions}

\end{exercise}
