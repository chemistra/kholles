\begin{exercise}{Câble coaxial}{-1}{Spé}
{Electromagnétisme,cable coaxial}{mines-T}

On considère deux câbles coaxiaux de rayons respectifs $R_1$ et $R_2 > R_1$. Un gaz isolant occupe tout l'espace entre les deux câbles. On lance une décharge sur le gaz qui le rend conducteur, de conductivité $\gamma$, en supposant qu'il reste à tout instant neutre. A l'instant $t=0$, la câble intérieur porte la charge $q_0$. A l'instant $t>0$, il porte la charge $q(t)$.

\begin{questions}
    \question Calculer $\vec{E}(t)$.
    \question En déduire $\vec{J}(t)$, et $\vec{B}(t)$.
    \question \'Etablir l'équation différentielle vérifiée par $q(t)$, et la résoudre.
\end{questions}

\end{exercise}