\begin{exercise}{Lunette astronomique}{-1}{Spé}
{Optique géométrique}{lelay}

Galilée souhaite observer les cratères de la lune. Pour ce faire, il voudrait disposer d'une lunette astronomique : pour cela, il lui faut assembler aux extrémités d'un tube deux lentilles convergentes en formant un système afocal, c'est-à-dire que le plan focal image de la première lentille (l'objectif) est confondu avec le plan focal objet de la seconde (l'oculaire).

Galilée souhaite obtenir un grossissement de 9, et le tube dont il dispose fait 1m20 de long. 

\textbf{Quelle est la vergence des lentilles que Galilée doit utiliser, et comment doit-il les agencer ?}

\end{exercise}

\begin{solution}
    $G = f_2/f_1$
    
    $L = f_1 + f_2$
    
    donc $FL = (G+1)f_1$
    
    donc $f_1 = 12$~cm et $f_2 = 108$ cm, $\delta = 1/f$
\end{solution}