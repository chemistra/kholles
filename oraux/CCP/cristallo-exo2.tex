\begin{exercise}{Thermochimie du germanium}{-1}{Spé}
{Thermochimie}{bermudez}

 On s'intéresse désormais à l'équilibre hétérogène à 790 K (en phase gaz et solide) entre GeO$_{\text{ (s)}}$ et H$_{2\text{ (g)}}$ qui produit entre autres du Ge$_\text{(s)}$.
 On dispose initialement de $n$(GeO)$=n_1=1$ mol, $n$(H$_2$)$=n_2=8$ mol, $n$(Ge)$=n_1=1$ mol et $n$(H$_2$O)$=n_1=1$ mol.
 \begin{questions}
    \question Donner l'équation de réaction. De quel type de réaction s'agit-il vis à vis du germanium ?
    \question Déterminer la constante $K^\circ$ de la réaction à $T_0 = 300$ K puis à $T_1 = 790$ K. 
    \question Dans quel sens évolue la réaction ? Déterminer l'état final du système.
\end{questions}

\paragraph{Données}
\begin{center}
\begin{tabular}{rcc}
    \hline
    (à 300 K) & $\mathrm{GeO_{2 (g)} \,/\, Ge_{(s)}}$ & $\mathrm{GeO_{(s)} \,/\, Ge_{(s)}}$ \\
    $E^\circ$ (mV) & $-50$ & $+262$\\ \hline\hline 
\end{tabular}~\\

\medskip


\begin{tabular}{rccccc}
    \hline
    (à 400 K) & Ge$_\text{(s)}$ & $\mathrm{H_2O_{(g)}}$ & $\mathrm{H_{2 (g)}}$ & $\mathrm{GeO_{(g)}}$ & $\mathrm{GeO_{2 (g)}}$ \\
    $S_\text{m}^\circ$ (J$\cdot$mol$^{-1}\cdot$K$^{-1}$) & 168 & 189 & 151 & 111 & 67  \\
    $c_\text{p,m}^\circ$ (J$\cdot$mol$^{-1}\cdot$K$^{-1}$) & 23 & 37 & 29 & 43 & 60\\ \hline\hline 
\end{tabular}\end{center}
\begin{itemize}
    \item $e^\circ = \dfrac{RT}{\scr{F}} \ln10 = 59$ mV à 300 K.
    \item Entropie de vaporisation de l'eau $\Delta_\text{vap}H$(H$_2$O, 300 K)$ = 44$ kJ$\cdot$mol$^{-1}$.
    \item Masse d'un nucléon $m_n = 1,67\times 10^{-27}$ kg.
\end{itemize}

\end{exercise}