% \begin{exercise}{Quantique pas fini}{-1}{Spé}
% {Quantique}{lelay}

% \paragraph{Exercice 1}
% \begin{questions}
%     \question Donner l'expression des niveaux d'énergie $E_n$ d'une particule de masse $m$ dans un puits de potentiel infini de longueur $a$.
%     \question On considère l'atome d'hydrogène, on note $r$ la distance de l'électron au noyau. Calculer $E_n$ sachant que l'électron se trouve dans un puits de potentiel infini de longueur la demi-circonférence de l'atome.
%     \question L'énergie totale de l'électron est $\cal{E}_n$, la somme de $E_n$ et de l'énergie potentielle d'interaction de l'électron avec le noyau. Donner l'expression de $\cal{E}_n$.
%     \question Donner les positions $r_n$ d'équilibre stable de l'électron. Calculer $r_n$ pour $n\in${$1,2,3$}.
%     \question Montrer que $\cal{E}_n$ peut s'écrire sous la forme
%     $$\cal{E}_n=-\dfrac{\text{Ry}}{n^2},$$
%     Ry étant la constante de Rydberg dont on donnera l'expression et la valeur.
%     \question De quelle couleur est la radiation émise par un électron qui passe du niveau d'énergie 3 au 2 ?
% \end{questions}

% \paragraph{Données}
% \begin{itemize}
% \item Masse de l'électron $m = 9,1\times 10^{-31}$ kg ;
%     \item Charge de l'électron $e = 1,6 \times 10^{-19}$ C ;
%     \item Constante de Planck $h = 6,6\times 10^{-34}$ J$\cdot$s ;
%     \item Permittivité du vide $\varepsilon_0 = 8,9 \times 10^{-12}$ H$\cdot$F$^{-1}$.
% \end{itemize}





% \end{exercise}