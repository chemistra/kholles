\begin{exercise}{Lunettes}{-1}{Spé}
{Optique géométrique}{lelay}

M. Griffin et Mme. Dinkley ont besoin de nouvelles lunettes. 

Mme Dinkley est myope et ne peut pas voir net au delà de trois mètres. M. Griffin est hypermétrope et ne voit rien en dessous d'une distance d'un mètre.

\textbf{Comment concevoir des lunettes adaptées pour que chacun de ces deux personnages ait une vue équivalente à celle d'un oeil emmétrope (qui n'a pas besoin de correction) ?}

\end{exercise}

\begin{solution}
On suppose que les lunettes sont accolées et donc les dioptries s'ajoutent.

En notant $a$ la distance cornée-rétine on a la vergence de l'oeil emmetrope accomodant à l'infini : $\delta_e^\infty = 1/a$. Pour l'oeil myope qui voit à une distance $d_m$, on a $\delta_m^\infty = \frac1a + \frac1{d_m}$. Avec la correction $\delta_m^{corr}$, 

$\delta_m^{corr} + \delta_m^\infty = \delta_e^\infty$ i.e. $\delta_m^{corr} = -\frac1{d_m}$ : vergence négative, verres divergents, dioptrie -1/3

Hypermétrope : pour l'oeil emmétrope on voit net à $d_e = 20$ cm : $\delta_e^0 = 1/a + 1/d_e$

l'oeil hypermétrope voit net à $d_h$ : $\delta_h^0 = 1/a + 1/d_h$

$\delta_h^{corr} + \delta_h^0 = \delta_e^0$ i.e. $\delta_h^{corr} = \frac{1}{d_e}-\frac1{d_h}$ : vergence positive, verres convergents, dioptrie 4.
\end{solution}