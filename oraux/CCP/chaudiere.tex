\begin{exercise}{Chaudière}{-1}{Spé}
{Thermo, système ouvert}{lelay, centrale, CCP}

On s’intéresse à un système pour chauffer l’eau. On fait rentrer de l’eau à une température $\theta_\text{e} = 30$~$^\circ$C dans un cylindre de rayon $R_\text{min}= 8$~cm creusé dans un plus grand cylindre de fonte de rayon $R_\text{max} = 40$~cm et de longueur $L = 60$~cm. Une source de chaleur (feu de bois par exemple) fournit une puissance thermique $\phi = 20$~kW à la fonte. L’eau circule avec un débit massique $D_\text{m} = 25$~kg/min. On suppose que la température dans la fonte ne varie que selon le rayon $r$.


\paragraph{Questions :}
\begin{questions}
    \question Déterminer la température $\theta_\text{s}$ de l’eau en sortie
    \question Déterminer la température de la fonte en $R_\text{max}$
    \question Comment évolue le système si l’on arrête simultanément la chauffage ($\phi = 0$) et l’arrivée d’eau ($D_\text{m} = 0$) ?
\end{questions}

Données :
\begin{itemize}
    \item Conductivité thermique de la fonte : 36~W/m/K
    \item Capacité thermique massique de la fonte : 0.45~J/g/K
    \item Masse volumique de la fonte : 7800~kg/m$^3$
    \item Température de fusion de la fonte : 1800~$^\circ$C
    \item Capacité thermique massique de l'eau : 4.18~J/g/K
\end{itemize}

\end{exercise}

\begin{solution}
    
\begin{questions}
    \question ThermoD en système ouvert $D_m c_{eau}(\theta_s-\theta_a) = \phi$. On trouve une augmentation de température de 11 degrés : c'est ok, la température d'une douche
    \question Diffusion dans un cylindre, il faut retrouver l'expression du laplacien en cylindrique $\Delta T = \frac1r \partial_r\qty(\frac1r\partial_rT)$. Il faut se mettre en régime permanent et prendre $T(R_{min}) = \theta_s$ par exemple. On trouve une temp ext de l'ordre de 150 degrés
    \question Il faut dire que tout se stabilise à une température $\theta_f$ et écrire le premier principe entre le moment où on coupe et la fin, puis intégrer pour trouver $\theta_f$.
\end{questions}
\end{solution}