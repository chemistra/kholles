\begin{exercise}{Bilan entropique}{-1}{Spé}
{}{mines}

\paragraph{Question de cours} \textsf{(15 minutes préparation, 10 minutes de passage) \textbf{:}}

Mouvements dans un champ de force centrale conservatif. Énergie potentielle effective.

\paragraph{Exercice} \textsf{(15 minutes préparation, 15 minutes de passage) \textbf{:}}

On considère une tige de masse volumique $\rho$, de section $\Sigma$ et de longueur $L$. La température de chacune de ses extrémités est maintenue constante par des thermostats à la température $T_1$ (en $x=0$) et $T_2$ (en $x=L$). La tige est calorifugée sur les côtés.

\begin{questions}
\question On se place dans le régime stationnaire. Quel est le profil de température stationnaire $T(x)$ ?

\question Déterminer l'entropie créée par unité de temps pour une tranche de longueur $\dd{x}$, puis pour toute la tige.

\question On retire les thermostats et on isole la barre de sorte à ce qu'elle n'échange plus de chaleur avec l'extérieur. Après un certain temps, que devient le profil de température ? Calculer la variation d'entropie de la tige.
\end{questions}

\end{exercise}
