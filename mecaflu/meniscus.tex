% Niveau :      PC
% Discipline :  Mécaflu
%Mots clés :    Tension superficielle

\begin{exercise}{Ménisque et chimie}{3}{Spé}
{Statique des fluides, Tension superficielle, Capillarité}{bermu}

\begin{questions}
    \questioncours Rappeler ce que sont les effets de tension de surface et les modéliser rapidement. On définira le coefficient de tension de surface $\gamma$.
\begin{EnvUplevel}
\paragraph{Rappel :} loi de Laplace \\
\`A l'interface entre deux fluides $a$ et $b$ non miscibles, la tension de surface (dont le coefficient est $\gamma_{ab}$) créée une surpression telle que
$$P_b - P_a = \dfrac{2\gamma_{ab}}{R_c},$$
$R_c$ étant le rayon de courbure de l'interface.

\bigskip

On considère un fluide au repos en contact avec une surface solide verticale. On note $\theta$ l'angle de contact entre le fluide et la paroi.

\end{EnvUplevel}
    \question \`A l'aide de la loi de Laplace, donner l'équation de la surface libre air--fluide $z(x)$. \\
    On pourra adimensionner l'équation avec la longueur capillaire $\ell_c$ dont on donnera l'expression et le sens physique.
    \question Simplifier l'équation précédente en estimant la valeur du nombre de Bond
$$\cal{B}o = \dfrac{L}{\ell_c},$$
$L$ étant la taille typique du système.
    \question Résoudre l'équation et déduire que l'ascension de fluide typique est
$$h = \ell_c \cot\theta.$$
\end{questions}

\plusloin

En chimie, la capillarité pose souvent des problèmes pour la mesure du volume. Modéliser les erreurs typiques de mesure de volumes en chimie. Faut-il lire le volume en haut ou en bas du ménisque ?

\paragraph{Données :} rayon de courbure d'une courbe 1D $z(x)$
$$\dfrac{1}{R_c} = \dfrac{z''(x)}{\qty(1 + z'(x)^2)^{3/2}}.$$
Dans le cas d'une surface 2D ayant une symétrie de révolution $z(r)$
$$\dfrac{1}{R_c} = \dfrac{z''(r) + \frac{1}{r}\qty(1 + z'(r)^2)z'(r)}{\qty(1 + z'(r)^2)^{3/2}}.$$
\end{exercise}