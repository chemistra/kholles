\maketitle

\frontmatter


\thispagestyle{empty}

Version du \today, \xxivtime.

\vfill

\paragraph{Contributeurices :}
\begin{itemize}
    \item Guillaume Bermudez (GB)
    \item Grégoire Le Lay (G2L)
    \item Léa Chocron (LC)
    \item Baptiste Corrège (BC)
\end{itemize}

\pagebreak

\subsection*{Programmes}

\newlist{programme}{enumerate}{3}
\setlist[programme,1]{label=\bfseries\qquad Thème \arabic*:,leftmargin=1em,itemindent=4.5em}
\setlist[programme,2]{label=\arabic{programmei}.\arabic*}
\setlist[programme,3]{label*=.\arabic*}

\subsubsection*{Première année PCSI, MPSI (2021)}
\begin{multicols}{2}
\begin{programme}
    \item Ondes et signaux
    \begin{programme}
        \item Formation des images
        \item Circuits électriques dans l'ARQS
        \item Circuit linéaire du premier ordre
        \item Oscillateurs libres et forcés
        \item Filtrage linéaire
        \item Propagation d'un signal
        \item Induction
        \begin{programme}
        \item Champ magnétique
        \item Actions d'un champ magnétique
        \item Lois de l’induction
        \item Circuits fixes
        \item Circuits mobiles
    \end{programme}
        \item Introduction à la physique quantique
    \end{programme}
    \item Mécanique
    \begin{programme}
        \item Cinématique
        \item Lois de Newton
        \item Approche énergétique
        \item Mouvement de particules chargées
        \item Loi du moment cinétique (TMC et TMCE)
        \item Forces centrales conservatives
        \item Mouvement d'un solide 
    \end{programme}
    \columnbreak
    \item Thermodynamique
    \begin{programme}
        \item Descriptions micro et  macro à l'équilibre
        \item Transformations et échanges d'énergie
        \item Premier principe. Bilans d'énergie
        \item Deuxième principe. Bilans d'entropie
        \item Machines thermiques
        \item Statique des fluides (PCSI)
    \end{programme}
    \item Chimie
    \begin{programme}
        \item Transformations de la matière
        \begin{programme}
            \item Etat final et thermo
            \item Cinétique
            \item Mécanismes réactionnels (PCSI)
        \end{programme}
        \item Structure
        \begin{programme}
            \item Structure des entités chimiques
            \item Interactions et solvants
        \end{programme}
        \item Structure des solides. Cristalograhphie
        \item Chimie des solutions
        \begin{programme}
            \item Acide-base
            \item Précipitation
            \item Oxydoréduction
            \item Complexes (PCSI)
        \end{programme}
        \item Chimie organique (PCSI)
        \begin{programme}
            \item Réactivité et SN
            \item Spectroscopies
            \item Oxydoréduction en chimie organique
            \item Activaction de groupes
            \item Protection et stratégies de synthèse
        \end{programme}
    \end{programme}
    
\end{programme}
\end{multicols}

\newpage

\subsubsection*{Deuxième année MP, PC, PSI (2021)}
\begin{multicols}{2}
\begin{programme}
    \item Optique ondulatoire (MP,PC)
    \begin{programme}
        \item Modèle scalaire des ondes lumineuses
        \item Superposition d’ondes lumineuses 
        \item Interférences par division du front d’onde
        \item Interférences par division d'amplitude
    \end{programme}
    \item Thermodynamique II
    \begin{programme}
        \item Systèmes ouverts en régime stationnaire
        \item Diffusion de particules (PC)
        \item Transport de charge (PC,PSI)
        \item Transferts thermiques. Diffusion thermique
        \item Rayonnement thermique (PC)
    \end{programme}
    \item Signal et électronique II
    \begin{programme}
        \item Analyse spectrale des signaux périodiques
        \item Stabilité des systèmes linéaires
        \item Oscillateurs et systèmes bouclés (PC,PSI)
        \item Amplificateur Linéaire Intégré (PSI)
        \item Échantillonage
        \item Électronique numérique (PSI)
        \item Modulation / Démodulation (PSI)
    \end{programme}
    \item Mécanique II
    \begin{programme}
        \item Référentiels  non galiléens
        \item Lois du frottement solide
    \end{programme}
    \item Mécanique des fluides (PC,PSI)
    \begin{programme}
        \item Cinématique et conservation de la masse
        \item Actions  de  contact  dans  un  fluide  en mouvement
        \item Equation de Navier--Stokes (PC)
        \item Ecoulement dans une conduite cylindrique
        \item Ecoulement autour d'un obstacle
        \item Bilans macroscopiques
        \item Tension superficielle (HP)
    \end{programme}
    \item \'Electromagnétisme
    \begin{programme}
        \item Bilan de charges, courants
        \item \'Electrostatique
        \item Magnétostatique
        \item Equations de Maxwell
        \item Electromagnétisme dans l'ARQS (PC,PSI)
        \item Millieux ferromagnétiques (PSI)
        \item Rayonnement et dipoles oscillants (MP)
    \end{programme}
    \item Physique des ondes (PC,PSI)
        \begin{programme}
            \item Propagation, absorption et dispersion 
            \item Ondes mécaniques (PC)
            \item Ondes acoustiques 
            \item Interfaces entre deux milieux
            \item Laser (PC)
        \end{programme}
    \item Conversion de puissance (PSI)
        \begin{programme}
            \item Puissance électrique en régime sinusoïdal
            \item Transformateur
            \item Conversion électro-magnéto-mécanique
            \item Conversion électronique statique
        \end{programme}
    \item Physique statistique (MP)
        \begin{programme}
            \item Facteur de Boltzmann
            \item Système à spectres discrets d’énergie
            \item Capacités thermiques classiques des gaz et des solides
        \end{programme}
    \item Mécanique quantique II (MP,PC)
         \begin{programme}
                \item Fonction d’onde, équation de Schrödinger
                \item Particule libre
                \item Potentiel constant par morceau
                \item \'Etats non stationnaires dans un puits infini (MP)
            \end{programme}
    \item Thermodynamique chimique
         \begin{programme}
                \item Premier principe chimique
                \item Second principe chimique
                \item Réacteurs ouverts (PC,PSI)
                \item Mélanges binaires (PC)
            \end{programme}
    
    \item Électrochimie
         \begin{programme}
                \item Thermodynamique des réactions redox
                \item Cinétique redox et courbes I-E
                \item Stockage et conversion d‘énergie chimique
                \item Corrosion humide et électrochimique
            \end{programme}
    \item Chimie quantique (PC)
        \begin{programme}
                \item Orbitales atomiques
                \item Orbitales moléculaires et réactivité
                \item Constitution et réactivité des complexes
        \end{programme}
    \item Chimie organique II (PC)
        \begin{programme}
                \item Conversion de groupes caractéristiques
                \item Création de liaisons carbone-carbone
        \end{programme}
\end{programme}
\end{multicols}

\newpage

\listofexercises

\newpage

\printindex

\mainmatter

\newpage