\begin{exercise}{Rails de Laplace inclinés}{2}{Sup}
{Induction}{lelay}

On considère un rail de Laplace (dans un champ $\vec{B} = B_0\ve_z$, une barre conductrice est placée en $x_0 > 0$ dans la direction $y$ perpendiculairement à une paire de rails dirigés selon $Ox$, écartés d'une distance $a$ et reliés en $x = 0$ par une résistance $R$).

\begin{questions}
    \questioncours Principe d'induction.
    \question On donne à la tige une vitesse initiale $v_0$. 
    \begin{parts}
        \part Sans calculs, quelle sera la vitesse de la barre lorsque $ t \rightarrow \infty$ ?
        \part Donner le temps caractéristique d'évolution de la vitesse de la barre.
    \end{parts}
    \question Cette fois on ne donne pas de vitesse initiale et on incline les rails d'un angle $\alpha$, de manière à ce que la barre subisse une force $g\sin(\alpha)$ la poussant vers l'avant.
    \begin{parts}
        \part Quelle sera la vitesse de la barre lorsque $ t \rightarrow \infty$ ?
        \part Donner le temps caractéristique d'évolution de la vitesse de la barre.
    \end{parts}
    \question Est-il raisonnable de supposer que les rails n'ont pas de résistance ? On suppose maintenant qu'ils sont chacun dotés d'une résistance linéique $\rho/2$.
    \begin{parts}
        \part Quelle différence par rapport à la question précédente ?
        \part Quelle sera la vitesse de la barre lorsque $ t \rightarrow \infty$ ?
        \part Donner le temps caractéristique d'évolution de la vitesse de la barre.
    \end{parts}
    \question Même question pour des rails divergents (au lieu d'être parallèles, ils sont écartés d'un angle $\alpha$).
\end{questions}

\end{exercise}