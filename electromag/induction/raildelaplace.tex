\begin{exercise}{Rail de Laplace}{1}{Sup}
{Induction}{lelay}

On considère un rail de Laplace (dans un champ $\vec{B} = B_0\ve_z$, une barre conductrice placée en $x_0 > 0$ dans la direction $y$ perpendiculairement à une paire de rails dirigés selon $Ox$, écartés d'une distance $a$ et reliés en $x = 0$ par une résistance $R$).

\begin{questions}
    \questioncours Loi de Lenz
    \question On donne à la tige une vitesse initiale $v_0$. 
    \begin{parts}
        \part Sans calcul, quelle sera la vitesse de la barre lorsque $ t \rightarrow \infty$ ?
        \part Donner la durée caractéristique d'évolution de la vitesse de la barre.
    \end{parts}
    \question Pour cette question, on ne donne pas de vitesse initiale mais un opérateur applique une force constante de norme $F_0$ sur la barre.
    \begin{parts}
        \part Quelle sera la vitesse de la barre lorsque $ t \rightarrow \infty$ ?
        \part Quelle est aux temps longs la puissance dissipée dans la résistance ?
    \end{parts}
    \question Pour cette question, on n'applique pas de force mais on remplace la résistance de la barre par une bobine d'inductance $L$. Quelle énergie faut-il pour déplacer la barre sur une distance $D$ ?
\end{questions}

\end{exercise}
