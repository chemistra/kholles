\begin{exercise}{Piège de Ioffe--Pritchard}{2}{Spé}
{Magnétostatique}{bermu, lelay}

\paragraph{Point méthode en électromagnétisme :}
\begin{itemize}
    \item à partir de la géométrie de problème, choisir le bon système de coordonnées ;
    \item utiliser les invariances du problème pour réduire le nombre de variables ;
    \item utiliser les symétries pour réduire le nombre de composantes ;
    \item identifier une surface fermée, un contour \emph{etc.} et appliquer le théorème intégral pertinent.
\end{itemize}


\begin{questions}
    \questioncours Exprimer le champ magnétique $\vB(\vr)$ d'un cylindre de rayon $R$ parcouru par un courant $I$, réparti volumiquement. On présentera les résultats d'électromagnétisme utiles.
    \question Justifier rapidement que l'on puisse écrire $\vB(\vr) = \rot(A(r)\ve_z)$, et trouver l'expression de $A(r)$ pour le champ précédent.
    \uplevel{On considère désormais le dispositif suivant : quatre fils conducteurs sont disposés aux sommets d'un carré de côté $2a$ et sont parcourus par des courants $+I$ et $-I$, alternativement.}
    \question Quelle est l'expression du potentiel vecteur $A(x, y)$ en un point $M$ d'un plan $(O, x, y)$ dont l'origine et les axes auront été judicieusement choisis ?
    \question Exprimer $A$ à l'ordre le plus bas pour $x$ et $y$ proches de l'origine.
    \question En déduire le champ magnétique $\vec{B}$ pour un point proche de l'origine. Son expression est-elle cohérente avec les invariances et les symétries de cette nouvelle configuration ? 
    \uplevel{On place dans ce champ un neutron de moment magnétique $\vec{m}$. À l'aide de techniques de pompage optique, il est possible de maintenir le moment magnétique du neutron dans la direction opposée de celle du champ.}
    
    \question Par analogie avec un dipôle électrique, donner l'énergie magnétique $E_m$ d'un tel neutron placé dans un champ $\vB$.
    
    \question Tracer la courbe $E_m(\rho)$ avec $\rho = \sqrt{x^2+y^2}$ pour le système étudié et commenter l'appellation "piège" de Ioffe pour ce dispositif.
    % \question On place dans ce champ un neutron de moment magnétique $\vec{m}$. Justifiez que le moment magnétique du neutron s'aligne avec celui du champ magnétique.
    % \question Quelle est la force appliquée par le champ magnétique sur le neutron ? En déduire la dynamique du neutron. Commenter l'appellation "piège" de Ioffe pour ce dispositif.
\end{questions}

\paragraph{Données :} 
% Rotationnel en coordonées cartésiennes
% $$\rot\vA = \qty(\pdv{A_z}{y} - \pdv{A_y}{z}) \ve_x
% + \qty(\pdv{A_x}{z} - \pdv{A_z}{x}) \ve_y
% + \qty(\pdv{A_y}{x} - \pdv{A_x}{y}) \ve_z$$ 
Rotationnel en coordonnées cylindrique
$$\rot\vA = \qty(\dfrac{1}{r}\pdv{A_z}{\theta} - \pdv{A_\theta}{z}) \ve_r
+ \qty(\pdv{A_r}{z} - \pdv{A_z}{r}) \ve_\theta
+ \qty(\dfrac{1}{r}\pdv{}{r} (r A_\theta) - \dfrac{1}{r}\pdv{A_r}{\theta}) \ve_z$$ 

\end{exercise}

\begin{solution}

\begin{questions}
    \questioncours $\vB(\vr) = \mu_o I/2\pi r \ve_\theta$
    \question On a $\vB(\vr) = \rot(A(r)\ve_z) = -\pdv{A}{r}\ve_\theta$ soit ici $A(r) = \mu_0 \frac{I}{2\pi}\ln(\frac{r}{R})$
    \uplevel{On considère désormais le dispositif suivant : quatre fils conducteurs sont disposés aux sommets d'un carré de côté $2a$ et sont parcourus par des courants $+I$ et $-I$, alternativement.}
    \question Les axes passent par les fils et le centre est au centre, les quatre distances aux fils sont $r_1 = \sqrt{x^2 + (a+y)^2}$, $r_2 = \sqrt{(a+x)^2 + y^2}$, $r_3 = \sqrt{x^2 + (a-y)^2}$, $r_4 = \sqrt{(a-x)^2 + y^2}$ et on a $$ A(x,y) =  \mu_0 \frac{I}{2\pi}\qty( \ln(\frac{r_1}{R}) - \ln(\frac{r_2}{R}) + \ln(\frac{r_3}{R}) - \ln(\frac{r_4}{R}) ) $$
    \question Attention il faut garder les termes d'ordre 2 c'est un peu degueulasse.
    \begin{align*}
        r_{1,3} & = a\sqrt{1 \pm 2\frac{y}{a} + \frac{x^2}{a^2} + \frac{y^2}{a^2}} \\
        &\approx a\qty(1 + \frac12\qty(\pm 2\frac{y}{a} + \frac{x^2}{a^2} + \frac{y^2}{a^2}) - \frac18\qty(2\frac{y}{a})^2) \\
        & = a\qty(1 \pm \frac{y}{a} + \frac12\frac{x^2}{a^2}) \\
        \ln(r_{1,3}) &= \ln\frac{a}{R} + \ln(1 \pm \frac{y}{a} + \frac12\frac{x^2}{a^2}) \\
        &\approx \ln\frac{a}{R} \pm \frac{y}{a} + \frac12\frac{x^2}{a^2} - \frac12\qty(\frac{y}{a})^2 \\
        &= \ln\frac{a}{R} \pm \frac{y}{a} + \frac12\frac{x^2-y^2}{a^2}
    \end{align*}
    et ainsi $r_{2,4}$ en échangeant $x$ par $y$ et vice versa.
    
    D'où
    \begin{align*}
        A(x,y) &=  \mu_0 \frac{I}{2\pi} \bigg(&& \\
        & &&+      \ln\frac{a}{R} + \frac{y}{a} + \frac12\frac{x^2-y^2}{a^2}\\ 
        & &&- \qty(\ln\frac{a}{R} + \frac{x}{a} - \frac12\frac{y^2-x^2}{a^2}) \\
        & &&+      \ln\frac{a}{R} - \frac{y}{a} + \frac12\frac{x^2-y^2}{a^2} \\
        & &&- \qty(\ln\frac{a}{R} - \frac{x}{a} + \frac12\frac{y^2-x^2}{a^2}) \\
        & &&\bigg)\\
        &=  \mu_0 \frac{I}{2\pi}\bigg(&& 2\frac{x^2-y^2}{a^2} \bigg) 
    \end{align*}
    
    \question On a $A(r) = \mu_0 \frac{I}{\pi}\frac{x^2-y^2}{a^2} $ et $B = (\partial_y A , -\partial_x A , 0) = -\mu_0 \frac{2I}{\pi a^2}(y, x, 0)$, $B$ ne dépend toujours pas de $z$.
    \uplevel{On place dans ce champ un neutron de moment magnétique $\vec{m}$. À l'aide de techniques de pompage optiques utilisant des faisceaux laser, il est possible de maintenir le moment magnétique du neutron dans la direction opposée de celle du champ.}
    
    \question $E_m = -\vec{m}\cdot \vB = m \abs{B}$ dans le cas où ils sont antiparallèle.
    
    \question On a $\abs{B} \propto \sqrt{x^2 +y^2}$ d'où $Em \propto \rho$ : on a bien un piège.
\end{questions}
\end{solution}
