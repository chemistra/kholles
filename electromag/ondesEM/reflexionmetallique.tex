\begin{exercise}{Réflexion sur un métal}{2}{Spé}
{Ondes électromagnétiques}{lelay}

On considère venant de $-\infty$ une onde plane, monochromatique de pulsation $\omega$ se propageant dans la direction des $z$ croissant et dont le champ électrique s'exprime $\vec{E} = E_0\cos(\omega t - kz)\vec{u}_x$.

Cette onde arrive sur un doncucteur parfait occupant le demi-espace $z > 0$.

On rappelle les relations de passage des champs électriques et magnétiques à l'interface entre deux milieux :
\begin{align*}
\vec{E}_2-\vE_1 &= \frac{\sigma}{\epsilon_0}\vec{n}_{12} \\
\vec{B}_2-\vB_1 &= \mu_0\vec{j}_S\cross\vec{n}_{12}
\end{align*}
Où $\sigma$ est la charge surfaçique et $\vec{j}_S$ le courant surfaçique.
\begin{questions}
    \questioncours Exprimer le champ magnétique associé à cette onde, puis son vecteur de Poynting.
    \question Montrer que lorsque l'onde arrive sur le conducteur, une onde réfléchie est nécessairement produite et donner son expression.
    \question Déterminer le champ électromagnétique résultant de l'onde incidente et de l'onde réfléchie dans le demi-espace $x < 0$. Interpréter. Que vaut alors la moyenne du vecteur de Poynting ?
    \question En déduire l'expression de $\sigma(x, y)$ et $\vec{j}_S(x,y)$.
    \uplevel{L'arrivée d'une onde électromagnétique sur un conducteur crée un effort mécanique sur le conducteur, qui prend la forme d'une force par unité de surface valant $\dd{\vec{F}} = \frac12(\sigma \vE + \vec{j}_S\cross\vec{B})\dd{S}$.}
    \question Quelle est l'origine microscopique de cette force ? Interpréter l'expression donnée.
    \question On parle de \textbf{pression de radiation}. Pourquoi ? 
    \question Calculer la valeur moyenne de la pression de radiation. Comment se compare-t-elle à la densité d'énergie moyenne de l'onde incidente (relation découverte par Maxwell) ?
    
    
    \question Exprimer pour l'onde incidente l'éclairement $\mathcal{E}$, c'est-à-dire la puissance surfacique moyenne de l'onde dans la direction perpendiculaire à la propagation, en fonction de $E_0$, $c$ et $E_0$. 

    \uplevel{D'après le principe de la dualité onde-particule de la physique quantique, on peut aussi considérer cette onde comme un flux de photons individuels.}
    
    \question Quelle serait alors l'énergie de chacun de ces photons ?
    \question En déduire le flux surfacique de photons $\phi$ (nombre de photons traversant une surface du plan $x,y$ par unité de temps et de surface) correspondant à cette onde.
    \question Quel est la quantité de mouvement porté par chacun des photons ? En déduire le flux de quantité de mouvement surfacique de l'onde.
    \question Retrouver l'expression de la pression de radiation.
    \question Par analogie, donner l'expression de la pression de radiation dans les cas suivants :
    \begin{parts}
        \part Au lieu d'être métallique, la surface $z=0$ est noire (absorbant parfaitement le rayonnement électromagnétique)
        \part La surface est blanche
    \end{parts}
\end{questions}

\end{exercise}

\begin{solution}

\begin{questions}
    \questioncours $B_0 = E_0/c$, $\Pi =E_0 B_)0/\mu_0 \cos^2(\omega t - kz) = E_0^2 c \epsilon_0 \cos^2(\omega t - kz)$
    
    \question Nullité du champ dans le conducteur et relation de passage.
    \question Déterminer le champ électromagnétique résultant de l'onde incidente et de l'onde réfléchie dans le demi-espace $x < 0$. Interpréter. Que vaut alors la moyenne du vecteur de Poynting ?
    \question $\sigma = 0$ , $j_S = 2\epsilon_0cE_0e^{i\omega t}\vec{e}_y$
    \uplevel{L'arrivée d'une onde électromagnétique sur un conducteur crée un effort mécanique sur le conducteur, qui prend la forme d'une force par unité de surface valant $\dd{\vec{F}} = \frac12(\sigma \vE + \vec{j}_S\cross\vec{B})\dd{S}$.}
    \question force de Lorentz, 12 pour pas prendre en compte champs induits.
    \question C'est une presion
    \question $P$=$E$ volumique (relation de Mxl) = $\epsilon E^2$
    
    
    \question vecteur de poynting en fait


    
    \uplevel{D'après le principe de la dualité onde-particule de la physique quantique, on peut aussi considérer cette onde comme un flux de photons individuels.}
    
    \question $E = h\nu$
    \question $ E\phi = \mathcal{E} $ d'où $\phi = c\epsilon_0 E_0^2/2h\nu$
    \question $p = \hbar k = h\nu/c$ d'où le flux $\phi p = \epsilon_0 E_0^2 / 2 = \Pi/c = \mathcal{E}/2c$
    
    \uplevel{On suppose que cette onde arrive sur un objet de masse $m$ présentant une surface $S$, orthogonale à la direction de propagation.}
    
    \question ON FAIT SOIT AVEC UN BILAN DE PARTICULES SOIT JUSTE AVEC L'ONDE
    \begin{parts}
        \part Noire : energie absorbée $\mathcal{E} S$, qté de mvt absorbée $\mathcal{E}S/2c$
        \part Miroir : Energie réféchie, qté de mvt absorbée double donc $\mathcal{E}S/c$
        \part Blanche : Energie reflechie, qté de mouvement = $\mathcal{E}S/2c(1 +$ qté de mvt diffusée de manière isotrope dans un angle solide de $2\pi$ (demi-sphère) soit
        $$
        \frac{1}{2\pi}\int_0^{2\pi}\dd{\varphi} \int_0^{\pi/2}\sin(\theta)\dd{\theta} = (1 - \sqrt{2}/2)
        $$
        ) donc voilà.
    \end{parts}
\end{questions}
\end{solution}