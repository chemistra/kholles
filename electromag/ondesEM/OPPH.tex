\begin{exercise}{Onde électromagnétique}{0}{Spé}
{Ondes électromagnétiques,Vide}{lelay}

On étudie la propagation d'une onde électromagnétique dans le vide dont le champ électrique correspondant est de la forme $\vec{E} = E_0\cos(\omega t - kz)\vec{u}_x$.
\begin{questions}
    \questioncours Rappeler les équations aux dérivées partielles auxquelles sont soumis les champs électriques et magnétiques dans le vide.
    \question Quelle est la direction, le sens et la vitesse de propagation de cette onde ? Quel type d'onde est-ce ? 
    \question Exprimer le vecteur d'ondes $\vec{k}$, le champ magnétique $\vec{B}$ et le vecteur de Poynting associés à l'onde.
    \question On suppose que cette onde rayonne à travers une surface effective $S = 4$ mm$^2$ une puissance moyenne $P = 10$ mW (valeur typique pour un laser de classe III.b). Calculer les amplitudes $E_0$ ey $B_0$ des champs électriques et magnétiques.
    \question La longueur d'onde de ce rayonnement est de 632 nm dans l'air. Dans quel domaine du spectre électromagnétique se situe cette onde ?
    \question Donner la fréquence correspondante en THz.
\end{questions}

\end{exercise}

\begin{solution}

\begin{questions}
    \questioncours Eqn de Mxl sans les sources
    \question direction $z$, sens propageant ($+z$), vitesse de propagation $c$ (on est dans le vide). OPPH (onde propageante plane harmonique) polarisée rectilignement
    \question $\vec{k} = \frac{\omega}{c}\vec{u}_z$, $\vec{B} = \frac{E_0}{c}\cos(\omega t - kz) \vec{u}_y$, vecteur de Poyting $\vec{\Pi} = EB/\mu_0 = E_0^2/(c\mu_0) \cos^2(\omega t - kz)$
    \question $\ev{\vec{\Pi}} = \frac12 \frac{E_0^2}{c\mu_0} = 2.5$ mW/mm$^2$. D'où $E_0$ et $B_0  =E_0 / c$
    \question Visible (rouge)
    \question $\nu = v_\phi / \lambda$, $v_\phi = c/n \approx c$ car $n_\text{air} \approx 1$
\end{questions}
\end{solution}