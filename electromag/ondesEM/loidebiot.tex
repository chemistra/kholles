\begin{exercise}{Loi de Biot}{3}{Spé}
{Ondes électromagnétiques,Polarisation rotatoire}{lelay}

On considère la propagation d'une OPPH électromagnétique dans un milieu optiquement actif occupant l'espace $z > 0$. Dans ce milieu, les ondes se propagent différemment en fonction de leur polarisation, ici les ondes polarisées circulairement gauches et droites se propagent avec des indices $n_g$ et $n_d$. On considère une OPPH se propageant dans la direction $Oz$, polarisée rectilignement selon $\ve_x$ en $z=0$.
\begin{questions}
    \questioncours Équation de propagation de la lumière, polarisation.
    \question Montrer que l'onde incidente, dans le vide ($z < 0$) peut s'écrire comme la superposition de deux ondes polarisées circulairement gauche et droite.
    \question Quelle est la polarisation de l'onde en $z$ quelconque, $z \geq 0$ ?
    \question Une substance est dite \textit{dextrogyre} si une onde initialement polarisée vers le haut (pour un observateur en face d'elle) tourne `vers la droite' d'un angle $\theta >0$, si elle tourne dans l'autre sens on dit que la substance est \textit{lévogyre}. Comment se traduit la différence entre ces deux milieux en fonction du signe de $\Delta n$ ?
    \question En s'inspirant de la loi de Beer-Lambert, donner la forme de la loi de Biot qui donne l'angle de rotation du plan de polarisation d'une onde lumineuse à travers une solution en fonction de la longueur de la cuve $\ell$, de la concentration de la solution $c$ et du pouvoir rotatoire spécifique du composé $[\alpha]$, dont on précisera l'unité.
\end{questions}

\end{exercise}