\begin{exercise}{Lame quart d'onde}{2}{Spé}
{Ondes électromagnétiques,Biréfringence}{lelay}

On considère une lame biréfringente d'épaisseur $e$ et de normale $\ve_z$. Biréfringente signifie que la propagation selon l'axe $Ox$ est caractérisée par un indice $n_x$ et la propagation selon l'axe $Oy$ par un indice $n_y$. La lame est transparente et non absorbante. Les axes $Ox$ et $Oy$ sont appelés \emph{lignes neutres} de la lame.
\begin{questions}
    \questioncours Équation de propagation de la lumière, notion d'indice optique.
    \question On suppose $n_x < n_y$. Expliquer pourquoi la ligne neutre $Ox$ est appelée \textit{axe rapide} alors que $Oy$ est appelée \textit{axe lent}.
    \question À l'entrée de la lame, en $z = 0$, on considère une onde électromagnétique incidente sous la forme d'une OPPH de longueur d'onde $\lambda_0$ dans le vide dont le champ électrique s'écrit en $z=0$
    $$
    \vec{E} = \vec{E}_0\cos(\omega t)
    $$
    Une fois dans la lame, quelle est l'équation différentielle vérifiée par chacune des composantes du champ électrique ?
    \question Donner la forme de $\vec{E}$ en sortie de la lame, en utilisant la notation $\bar{n} = n_y+n_x$ et $\Delta n = n_y-n_x$.
    \question Exprimer $\vec{E}_0$ si l'onde incidente est polarisée selon $\ve_x$. Caractériser l'état de polarisation en sortie de la lame.
    \uplevel{On se place maintenant dans le cas où l'épaisseur $e$ de la lame vérifie $\frac{\omega}{c}(n_y-n_x)e = \frac{\pi}{2}$.}
    \question Exprimer $e$ en fonction de $\lambda_0$ et $\Delta n$. Justifier l'appellation \textit{lame quart d'onde} pour ce type d'objets.
    \question On considère le cas $\vec{E}_0 = E_0 \qty(\frac{\ve_x}{\sqrt{2}}+\frac{\ve_y}{\sqrt{2}})$. Quelle est l'état de polarisation de la lame en entrée ? Faire un schéma. Que devient cet état en sortie de lame ?
    \question L'onde incidente est maintenant polarisée circulairement. Comment est alors $\vec{E}_0$ ? Que devient la polarisation en sortie de lame ?
    \question On place, sur un miroir plan l'ensemble constitué d'un polarisateur rectiligne P et d'une lame quart d'onde Q. L'axe du polariseur est  selon une bissectrice des lignes neutres de Q. Qu'observe-t-on à travers l'ensemble P+Q+miroir ?
\end{questions}

\end{exercise}