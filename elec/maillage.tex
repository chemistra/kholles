% Niveau :      PCSI - PC
% Discipline :  Elec
% Mots clés :   Elec, Ordre 2

\begin{exercise}{Résistance d'un maillage}{2}{Sup,Spé}
{\'Electrocinétique, Circuits d'ordre 2}{schlosser}

On s'intéresse à la résistance d'un maillage pris entre deux points.

\begin{questions}
    \questioncours Dipôles actifs et passifs : caractéristique, résistance et conductance. Association de dipôles.
    \question Quelle est la résistance d'une maille simple (Circuit \arabic{exercise}.1) ?
    \question Quelle est la résistance d'une maille double (Circuit \arabic{exercise}.2) ?
\begin{EnvUplevel}
    Dans les figures ci-dessous, toutes les résistances valent $R$.
\begin{multicols}{2}
\begin{circuit}[Maille simple]
    \draw (0,0)
    to [R] (2,0)
    to [R] (2,2)
    to [R] (0,2)
    to [R] (0,0) ;
    \draw (-.4,2.4) to [short, *-*] (0,2);
    \draw (2,0) to [short, *-*] (2.4,-.4);
\end{circuit}
\begin{circuit}[Maille double]
    \draw (0,0)
    to [R] (2,0)
    to [R] (4,0)
    to [R] (4,2)
    to [R] (4,4)
    to [R] (2,4)
    to [R] (0,4)
    to [R] (0,2)
    to [R] (0,0);
    \draw (2,4)
    to [R, *-*] (2,2)
    to [R, *-*] (2,0) ;
    \draw (4,2)
    to [R, *-*] (2,2)
    to [R, *-*] (0,2) ;
    \draw (-.4,4.4) to [short, *-*] (0,4);
    \draw (4,0) to [short, *-*] (4.4,-.4);
\end{circuit}
\end{multicols}

Et maintenant l'astuce !
\end{EnvUplevel}
    \question Monter que les courts-circuits coupant perpendiculairement les axes de symétrie et d'antisymétrie de la répartition des courants peuvent être supprimés sans modifier la répartition des courants.
    
    \question Retrouver par cette  méthode le résultat de la question 3.
    
    \question Donner le résultat pour la maille triple :
\end{questions}
    
\begin{circuit}[Maille triple]
    \draw (0,0)
    to [R] (2,0)
    to [R] (4,0)
    to [R] (6,0)
    to [R] (6,2)
    to [R] (6,4)
    to [R] (6,6)
    to [R] (4,6)
    to [R] (2,6)
    to [R] (0,6)
    to [R] (0,4)
    to [R] (0,2)
    to [R] (0,0);
    \draw (2,0)
    to [R, *-*] (2,2)
    to [R, *-*] (2,4)
    to [R, *-*] (2,6) ;
    \draw (4,0)
    to [R, *-*] (4,2)
    to [R, *-*] (4,4)
    to [R, *-*] (4,6) ;
    \draw (0,2)
    to [R, *-*] (2,2)
    to [R, *-*] (4,2)
    to [R, *-*] (6,2) ;
    \draw (0,4)
    to [R, *-*] (2,4)
    to [R, *-*] (4,4)
    to [R, *-*] (6,4) ;
    \draw (-.4,6.4) to [short, *-*] (0,6);
    \draw (6,0) to [short, *-*] (6.4,-.4);
\end{circuit}

C'est plus rapide comme ça, non ?

%\plusloin Quid du cas de la $n$-maille ?

\end{exercise}

\begin{solution}
    \url{https://fr.wikiversity.org/wiki/Signaux\_physiques\_(PCSI)/Exercices/Circuits\_\%C3\%A9lectriques\_dans\_l\%27ARQS\_:\_associations\_de\_conducteurs\_ohmiques}
\end{solution}