% Niveau :      PCSI - PC
% Discipline :  Elec
% Mots clés :   Elec, Ordre 2

\begin{exercise}{Philtre inconnu}{2}{Sup,Spé}
{\'Electrocinétique, Circuits d'ordre 2}{bermudez}

On cherche à décrire le circuit suivant :

\begin{circuit}
      \draw (0,0)
      to [open, v^=$e$, -o] (0,2)
      to [short,o-*] (1,2)
      to [R, l=$R$, *-*] (3,2)
      to [R, l=$R$, *-*] (5,2) 
      to [short,*-o] (6,2);
      \draw 
      (3,2) to [C, l=$C$,-*] (3,0)
      to [short] (0,0)
      to [short] (6,0)
      to [open, v_=$s$] (6,2) ;
      \draw
      (1,2) to [short] (1,3)
      to [C, l=$C$] (5,3)
      to [short] (5,2);
\end{circuit}

\begin{questions}
    \questioncours Pour chaque composant du circuit, donner la loi associée et donner le comportement asymptotique de ce filtre.
    \question Donner la fonction de transfert de ce filtre et tracer le diagramme de Bode associé. On introduira un temps pertinent $\tau$ pour adimensionner la fréquence $x = \omega \tau$
\end{questions}
\end{exercise}

\begin{solution}
    C'est un passe rien $\tau=RC$
    $$H(\omega) = \dfrac{1 + 2jx-x^2}{1 + 3jx -x^2}$$
\end{solution}