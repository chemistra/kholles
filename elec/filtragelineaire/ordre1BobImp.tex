% Niveau :      PCSI
% Discipline :  Elec
% Mots clés :   Elec, Ordre 1

\begin{exercise}{Bobine réelle}{1}{Sup}
{\'Electrocinétique, Circuits d'ordre 1}{lelay}

\begin{questions}
    \questioncours Donne la loi reliant la tension aux bornes d'une bobine au courant qui la traverse, sous forme différentielle et complexe. Unité et ordre de grandeur de l'inductance d'une bobine.
    \question La bobine "idéale" existe-t-elle dans la vraie vie ? Justifie qu'on représente une bobine "réelle" par une bobine idéale et un résistance $r$ en série. Comment doit être $r$ afin de se rapproche le plus possible de la bobine idéale ?
    \question On cherche maintenant à trouver l'inductance d'une certaine bobine, en prenant en compte sa résistance interne. On considère le circuit ci-dessous, où $R$ vaut 100 $\Omega$ :
    \begin{circuit}
          \draw
          (0,0) to [open, v_=$\underline{e}$, *-o] (0,2)
          (0,2) to [R, l=$r$] (2,2) 
                to [L, l=$L$] (4,2)
                to [R, l=$R$] (4,0)
          (4,2) to [short] (6,2)
          (6,0) to [open, v_=$\underline{s}$, *-o] (6,2)
          (0,0) to [short] (6,0);
    \end{circuit}
    Donner sa fonction de transfert $\underline{H} = \frac{\underline{s}}{\underline{e}}$
    \question Tracer l'allure du graphe de $\abs{\underline{H}}$ et $\arg(\underline{H})$ en fonction de $\frac{\omega}{\omega_0}$ où $\omega_0$ est une pulsation caractéristique du système que l'on précisera.
    \question Si on se place à très basse fréquence et que le signal d'entrée $\underline{e}$ a une amplitude de 6 V, le signal de sortie présente une amplitude de 5 V. En déduire la valeur de $r$.
    \question On constate qu'à $f = 2.4$ kHz, les deux signaux $\underline{s}$ et $\underline{e}$ sont en quadrature de phase. En déduire la valeur de l'inductance $L$.
\end{questions}
\end{exercise}


\begin{solution}


\begin{questions}
    \questioncours $U = L\dv{i}{t}$, $L\sim 10$ mH
    \question Impossible ideale, une bobine est faite de fil long donc au bout d'un moment on doit avoir de la résistance. idéale : résistance nul du fil, donc $r=0$.
    \question On a un diviseur de tension : 
    $$ \underline{s} = \frac{R}{R\,+\,r + jL\omega}\underline{e}$$
    soit sous forme canonique
    $$ H(\omega) = \frac{H_0}{1 + j\frac{\omega}{\omega_0}}$$
    avec $H_0 = R/(R+r)$ et $\omega_0 = (R+r)/L$
    \question On a un passe bas d'ordre 1, avec une phase qui va de 0 à $-\pi/2$
    \question À très basse fréquence on $H = H_0 = 5/6$ d'où $r = R/5 = 20$ $\Omega$
    \question quadrature de phase : on a $\omega = \omega_0$. On en déduit $L = (R+r)/(2\pi f)$ soit 8 mH.
\end{questions}

\end{solution}