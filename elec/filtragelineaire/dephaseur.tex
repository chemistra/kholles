% Niveau :      PCSI - PC
% Discipline :  Elec
% Mots clés :   Elec, Ordre 2

\begin{exercise}{Filtre déphaseur}{2}{Sup,Spé}
{\'Electrocinétique,Filtrage linéaire, Troisième ordre +}{lelay, X}

\begin{questions}
    \questioncours Rappeler le montage suiveur avec un A.O. et discuter de son intérêt pour construire des filtres d'ordre $n$.
    \question Donner la fonction de transfert du filtre ci dessous. Pourquoi l'appelle-t-on filtre déphaseur ?
\end{questions}

\begin{circuit}
      \draw
      %node [ground] at (4,0) {}
      (0,0) to [open, v_=$e$, *-o] (0,2)
      (0,2) to [R, l=$R$] (2,2) 
      to [C, l^=$C$, *-*] (2,0)
      (2,2) to [R, l=$R$] (4,2)
      to [C, l^=$C$, *-*] (4,0)
      (4,2) to [R, l=$R$] (6,2)
      to [C, l^=$C$, *-*] (6,0)
      (6,2) to [short] (8,2)
      (8,0) to [open, v_=$s$, *-o] (8,2)
      (0,0) to [short] (8,0);
\end{circuit}
\end{exercise}

\begin{solution}

$$H = \frac{1}{q(2+p)-(1+p)} \qq{avec} q=(2+p)(1+p)-1$$

\end{solution}