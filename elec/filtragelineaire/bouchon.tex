% Niveau :      PCSI - PC
% Discipline :  Elec
% Mots clés :   Elec, Ordre 2

\begin{exercise}{Filtre bouchon}{1}{Sup,Spé}
{\'Electrocinétique,Filtrage linéaire, Second ordre}{lelay}

On cherche à décrire le circuit suivant :

\begin{circuit}
      \draw
      (0,0) to [short] (5,0)
      to [open, v_=$s$, *-o] (5,2) 
      to [short] (2,2)
      (0,0) to [open, v^=$e$, *-o] (0,2)
      to [R, l=$R$] (2,2)
      to [L, l=$L$, *-*] (2,0) 
      (3.5,0) to [C, l_=$C$, *-*] (3.5,2);
\end{circuit}

\begin{questions}
    \questioncours Pour chaque composant du circuit, donner la loi associée et le comportement asymptotique.
    \question Donner la fonction de transfert de ce filtre et tracer le diagramme de Bode associé.
    \question Quel est le comportement de ce filtre ?
    \question \'Etudier les limites $R\longrightarrow 0$ et $R\longrightarrow +\infty$. Pourquoi observe-t-on un tel comportement ?
\end{questions}
\end{exercise}

\begin{solution}
    $$H(\omega) = \dfrac{1}{1 + j Q \qty(\dfrac{\omega}{\omega_0} - \dfrac{\omega_0}{\omega})}, Q=R\sqrt{\dfrac{C}{L}}, \omega_0 = \dfrac{1}{\sqrt{LC}}$$
\end{solution}