% Niveau :      PCSI - PC
% Discipline :  Elec
% Mots clés :   Elec, Ordre 2

\begin{exercise}{Cavité laser}{1}{Sup,Spé}
{\'Electrocinétique, Circuits d'ordre 2}{lelay,X}

On cherche à décrire le circuit suivant :

% \begin{circuit}
%     \node [op amp](A1){} (0,0);
%     \draw (A1.+) to [short] ++(0,-1) coordinate(tmp)
%     to [R, l^=$R_2$, *-*] ++(-2.5,0) node[ground]{} coordinate(tmp2)
%     to [R, l^=$R$, *-*] (A1.- -| tmp2)
%     node[anchor=east]{$V_\textsc{a}$}
%     to [short] (A1.-)
%     (tmp) to [R, l_=$R_1$] (tmp -| A1.out)
%     to [short, -*] (A1.out) node[anchor=west]{$V_\textsc{b}$}
%     to [short] ++(0,2) coordinate (tmp3)
%     to [C, l_=$C$] (tmp3 -| A1.-)
%     to [L, l_=$L$] (tmp3 -| tmp2)
%     to [short] (A1.- -| tmp2) ;
%     \draw [red, dashed] (-3.2,-2.25) rectangle(1.8,1.3) ;
%     \draw [blue, dashed] (-3.3,-1.2) to (-3.27,1.4) to (1.8,1.4) to (1.8,3.1) to (-4.4,3.1) to (-4.4,-1.2) to (-3.3,-1.2) ;
%     \node [blue] at (-4,2.8) {\textsfbf{(A)}};
%     \node [red] at (-2.8,0.9) {\textsfbf{(B)}};
% \end{circuit}

\begin{circuit}
    \node [op amp](A1){} (0,0);
    \draw (A1.+) to [short] ++(0,-1) coordinate(tmp)
    to [R, l^=$R_2$, *-*] ++(-2.5,0) node[ground]{} coordinate(gnd)
    to [R, l^=$R$, *-*] (A1.- -| gnd) node[anchor=south]{$V_\textsc{a}$} coordinate(Va)
    to [short] (A1.-);
    
    \draw (tmp) to [R, l_=$R_1$] (tmp -| A1.out)
    to [short, -*] (A1.out) node[anchor=west]{$V_\textsc{b}$}
    to [short] ++(0,1.5) coordinate (tmp3);

    
    \draw (Va)  to [short] ++(-0.5,0)
    to [L, l_=$L$] ++(-1.5,0)
    to [C, l^=$C$] ++ (-1.5,0) coordinate(entreefiltre)
    to [short] (tmp3 -| entreefiltre)
    to [short] (tmp3 -| A1.out);

    
    \draw [red, dashed] (-3.2,-2.25) rectangle(1.8,1.3) ;
    \draw [blue, dashed] (-3.3,-1.2) rectangle(-7,1.3) ;
    % \draw [blue, dashed] (-3.3,-1.2) to (-3.27,1.4) to (1.8,1.4) to (1.8,3.1) to (-4.4,3.1) to (-4.4,-1.2) to (-3.3,-1.2) ;
    \node [blue] at (-6.6,-0.8) {\textsfbf{(A)}};
    \node [red] at (-2.8,0.9) {\textsfbf{(B)}};
\end{circuit}

\begin{questions}
    \questioncours Caractéristique de l'amplificateur opérationnel (AO ou ALI) en régime linéaire
    \question Identifier la partie \textsfbf{(A)} du circuit. Donner rapidement sa fonction de transfert et l'allure de son diagramme de Bode. Comment choisir $R$, $L$ et $C$ pour sélectionner un fréquence $\omega_0$ le plus finement possible ?
    \question Identifier la partie \textsfbf{(B)} du circuit. Quel rôle remplit-elle ?
    \question Remarquer que les deux parties \textsfbf{(A)} et \textsfbf{(B)} sont reliées de manière circulaire. Que va-t-il se passer ? Ce phénomène vous rappelle-t-il un dispositif optique bien connu ?
\end{questions}
\end{exercise}

\begin{solution}

\begin{questions}
    \questioncours La baz
    \question C UN PASSE BANDE
    \question C UN AMPLI
    \question C UN LASER
\end{questions}

\end{solution}
