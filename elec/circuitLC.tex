% Niveau :      PCSI
% Discipline :  Elec
% Mots clés :   Elec, Ordre 1

\begin{exercise}{Oscillateur électronique}{1}{Sup}
{\'Electrocinétique, Oscillateur harmonique}{lelay}

Considère le circuit suivant :
\begin{circuit}
      \draw
      node [ground] at (0,0) {}
      to [vsource, v^>=$E$] (0,2)
      to [R, l^=$R$] (2,2)
      to [nos, l=$K$] (4,2)
      to [R, l^=$R$] (6,2)
      to [vsource, v^<=$E$] (6,0)
      node [ground]{}
      (2,2) to [R, l_=$R$] (2,0)
      node [ground]{}
      (4,2) to [C, l^=$C$] (4,0)
      node [ground]{};
\end{circuit}
\begin{circuit}
    \draw (0,0) node[spdt, rotate=90] (switch) {};
    \draw (switch.out 1) 
    to [short] ++(-1,0)
    to [vsource, v^>=$E$] ++(0,-3)
    to [short] ++(0.8,0)
    to [short] (switch.in);
    
    % \draw
    % (0,0) node[cute spdt up, rotate=90]{};% (switch) {}
    % (switch.in) node[left] {in}%
    % (switch.out 1) node[right] {out 1}
    % (switch.out 2) node[right] {out 2}
;
\end{circuit}

À $t < 0$    , l'interrupteur est ouvert et le condensateur est chargé, la tension à ses bornes est $u_0$. À $t = 0$, on ferme l'interrupteur.

\begin{questions}
    \questioncours Donne la loi reliant la tension aux bornes d'un condensateur à la charge qu'il contient. Unité et ordre de grandeur de la capacité d'un condensateur.
    \question Quelle est la tension aux bornes du condensateur tant que l'interrupteur reste ouvert ? Le courant qui le traverse ?
    \question À $t = 0$, on ferme l'interrupteur $K$. Donner le circuit équivalent à $t = 0^+$ et à $t = \infty$.
    \question Donner l'équation différentielle vérifiée par la tension $u$ aux bornes du condensateur.
    \question Quel est le temps caractéristique d'évolution de $u(t)$ ? Quel nom proposez-vous de lui donner ?
    \question Écrire $u(t)$ et donner l'allure de sa courbe.
\end{questions}
\end{exercise}


\begin{solution}

\begin{questions}
    \questioncours ok
    \question $E$
    \question $t = 0$, le condensateur est un fil, $t=\infty$, le condensateur est un interrupteur ouvert
    \question $RC \dv{u}{t} + 3 u = 2e$
    \question $\tau = RC/3$
    \question Une exponentielle saturante pépère.
\end{questions}

\end{solution}