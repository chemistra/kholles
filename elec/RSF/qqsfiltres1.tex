\begin{exercise}{Quelques (pré) filtres (1)}{1}{Sup}
{\'Electrocinétique}{lelay}

\begin{questions}
    \questioncours Notion de déphasage, d'avance et de retard de phase
    \question Donner le déphasage entre $\underline{s}$ et $\underline{e}$ dans le circuit suivant en fonction de la fréquence $\omega$ de $\underline{e}$
\begin{circuit}
      \draw
      (0,0) to [open, v_=$\underline{e}$, *-o] (0,2)
      (0,2) to [R, l=$R$] (2,2) 
      to [C, l^=$C$, *-*] (2,0)
      (2,2) to [short] (4,2)
      (4,0) to [open, v_=$\underline{s}$, *-o] (4,2)
      (0,0) to [short] (4,0);
\end{circuit}
    \question Pour quelle valeur de $\omega$ a-t-on un déphasage de $-\pi/3$ ?
    \question Tracer la courbe $\Delta\varphi = f(\omega)$
    \question Donner le déphasage entre $\underline{s}$ et $\underline{e}$ dans le circuit suivant en fonction de la fréquence $\omega$ de $\underline{e}$
\begin{circuit}
      \draw
      (0,0) to [open, v_=$\underline{e}$, *-o] (0,2)
      (0,2) to [R, l=$R$] (2,2) 
      to [L, l^=$L$, *-*] (2,0)
      (2,2) to [short] (4,2)
      (4,0) to [open, v_=$\underline{s}$, *-o] (4,2)
      (0,0) to [short] (4,0);
\end{circuit}
    \question Pour quelle valeur de $\omega$ a-t-on un déphasage de $\pi/3$ ?
    \question Tracer la courbe $\Delta\varphi = f(\omega)$
    \question Que remarquez-vous ?
\end{questions}

\end{exercise}
