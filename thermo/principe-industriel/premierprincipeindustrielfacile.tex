\begin{exercise}{Refroidisseur à eau}{1}{Sup, spé}
{Thermodynamique, Premier principe industriel}{lelay,centrale}

\begin{questions}
    \questioncours Principes de la thermodynamique en régime ouvert
    \uplevel{On considère un refroidisseur à eau, un système dans lequel circule de l'air et de l'eau. L'air entre dans le dispositif à la température $T_1 = 500$ K et en sort à la température $T_2 = 300$ K avec un débit de sortie de 65 L/s. En entrée et en sortie l'air est à pression atmosphérique. L'eau rentre à la température $\Theta_e = 12$ degrés Celsius avec un débit de 1 L/s et sort à la température $\Theta_s$.}
    \question Calculer $\Theta_s$.
\end{questions}

\end{exercise}

\begin{solution}
C'est juste le premier principe industriel. Il faut savoir retrouver la masse molaire de l'air (29 g/mol).
\end{solution}