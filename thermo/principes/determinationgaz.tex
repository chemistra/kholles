\begin{exercise}{Détermination des caractéristiques d'un gaz}{2}{Sup, Spé}
{Thermodynamique}{lelay}

On considère une conduite de section $S$. On place entre son extrémité et un piston une masse $m$ d'un gaz inconnu. On cherche à déterminer les propriétés de ce gaz.

\begin{questions}
    \questioncours Gaz parfaits : équation d'état, énergie interne, capacités thermiques et indice adiabatique (coefficient de Laplace).
    \uplevel{On place dans le conduit de manière transversale un fil de résistance $R$. Initialement, le gaz est à l'équilibre à la température $T_0$}
    \question En partant de la situation d'équilibre et en maintenant le piston en place, on fait passer un courant $I$ dans le fil pendant un temps $\tau$. On constate que la température est passé de $T_0$ à $T_1$. Quelle est l'énergie fournie au gaz ?
    \question On opère la même opération, cette fois-ci en laissant le piston libre de se déplacer. La température passe alors de $T_0$ à $T_2$. S'attend-t-on à avoir $T_1 = T_2$ ?
    \question Déterminer les caractéristiques du gaz.
\end{questions}

\end{exercise}