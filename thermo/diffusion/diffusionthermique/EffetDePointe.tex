\begin{exercise}{Effet de pointe}{2}{Spé}
{Diffusion thermique}{lelay}

On dispose d'une plaque de surface $S_0$ à la température $T_0$ par laquelle on peut faire passer un flux de chaleur $\phi$ arbitraire (penser à une plaque de cuisson par exemple). On pose sur cette plaque un conducteur thermique possédant une symétrie axiale. Une extrémité de ce conducteur thermique a la surface $S_0$, que l'on pose sur la plaque.

\begin{questions}
    \questioncours Notion de flux de chaleur, équation de la chaleur.
    \uplevel{On suppose dans un premier temps que le conducteur thermique est calorifugé (hormis la surface de contact $S_0$). On suppose que la température $T$ ne dépend que la distance $x$ à la plaque.}
    \question Donner la température $T(x)$ de la plaque à une abcisse $x$ quelconque en fonction de $T_0$, $\lambda$, $\phi$ et une intégrale $I(x)$ que l'on précisera.
    \question Que se passe-t-il si le rayon du conducteur thermique $r(x)$ décroît ? Croît ?
    \question Que vaut la température $T$ à l'extrémité du matériau si celui-ci forme une pointe, par exemple si c'est un cône ? Est-ce physiquement acceptable ?
    \uplevel{On considère maintenant que le matériau n'est plus calorifugé. Il échange une quantité d'énergie surfacique $\phi_{S, ext}(x)$ avec son milieu environnant, ce qui permet de le maintenir à température constante $T(x) = T_0$.}
    \question Faire un bilan d'énergie sur une tranche du conducteur de longueur $\dd{x}$.
    \question En déduire $\phi_{S, ext}(x)$.
    \question Que devient $\phi_{S, ext}(x)$ lorsque $r(x)$ tend vers 0 ? Pourquoi appelle-t-on cet effet ``l'effet de pointe'' ?
\end{questions}

\end{exercise}
