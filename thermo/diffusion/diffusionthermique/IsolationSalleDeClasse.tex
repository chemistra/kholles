\begin{exercise}{Isolation d'un salle de classe}{2}{Spé}
{Diffusion thermique}{lelay, mines}

On considère une salle de classe contenant $30$ élèves émettant chacun une puissance thermique de $60$ W. Cette salle est supposée calorifugée à l'exception d'un mur de la salle de surface $S_\text{m} = 30$ m$^2$, d'épaisseur $e_\text{m} = 20$ cm et de conductivité thermique $\lambda_\text{m} = 1.5$ W.m$^{-1}$.K$^{-1}$. Dans ce mur sont percées 5 fenêtres de conductivité thermique $\lambda_\text{f} = 1.0$ W.m$^{-1}$.K$^{-1}$ et d'épaisseur $e_\text{f} = 5$ mm, ayant chacune une surface $S_\text{f} = 1$ m$^2$. L'air extérieur est à $T_\text{e} = 10^\circ$C. On appelle $T_\text{i}$ la température à l'intérieur de la pièce.

\begin{questions}
    \questioncours Équation de la chaleur, notion de résistance thermique
    \question En supposant que la paroi extérieure des murs et des fenêtres soient à la température $T_\text{e}$, Quel est le flux de chaleur sortant par le mur ? Par les fenêtres ? Quelle est la principale cause des pertes thermiques ?
    \question En déduire la température d'équilibre $T_\text{i}$ de la pièce (pour laquelle la puissance thermique émise par les élèves compense celle des pertes thermiques).
    \question Quelle doit être la puissance des radiateurs de la salle pour que celle-ci soit maintenue à 20 degrés ?
    \uplevel{On considère maintenant que la paroi extérieure des fenêtres n'est pas à $T_\text{e}$ mais à $T_\text{s}$, et que le fenêtre échange à tout instant une puissance $P_\text{ech} = h S_\text{f} (T_\text{s}-T_\text{e})$ avec l'extérieur (modèle du flux conducto-convectif). On prendra $h = 20$ S.I.}
    \question D'où vient l'expression de $P_\text{ech}$ ? Quelle est l'unité de $h$ ?
    \question Qu'est-ce que cela change ? Avec ce modèle, $T_\text{i}$ sera-t-il a priori plus ou moins haut ?
    \question Calculer, avec ce modèle, $T_\text{i}$.
    \question Quelle doit être la puissance des radiateurs de la salle pour que celle-ci soit maintenue à 20 degrés ?
\end{questions}

\end{exercise}
