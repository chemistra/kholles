\begin{exercise}{Double vitrage}{1}{Spé}
{Diffusion thermique}{lelay, mines}

On souhaite poser une fenêtre en verre ($\lambda_v = 1.0$ W.m$^{-1}$.K$^{-1}$) d'un mètre carré dans une pièce devant être maintenue à 20 degrés Celsius. L'air extérieur à est 0 degrés Celsius.

\begin{questions}
    \questioncours Notion de résistance thermique
    \question Une vitre standard a comme épaisseur $e_v = 2$ mm. Quelle puissance faudra-t-il utiliser pour chauffer la pièce si on installe une vitre de ce type ? Cela vous parait-il raisonnable ?
    \question Quelle épaisseur doit faire la vitre si on veut ramener cette puissance à 400 W ? À 200 W ? Cela vous semble-t-il pratique ?
    \question On décide plutôt d'installer une fenêtre en double vitrage, composée de deux vitres épaisses de $e_v$ entre lesquelles il y a une épaisseur $e_a$ d'air ($\lambda_a = 0.02$ W.m$^{-1}$.K$^{-1}$). Quelle épaisseur $e_a$ d'air faut-il avoir pour n'avoir besoin que de 400 W pour chauffer la pièce ? Quelle est alors l'épaisseur totale de la fenêtre ?
    \question On souhaite réduire la puissance de chauffe à 200 W. Quelle épaisseur $e_a'$ d'air faut-il alors ? 
    \question En fait, on ne trouve pas de double vitrage avec une épaisseur $e_a'$ dans le commerce, seulement du triple vitrage constitué de trois plaques de verre entre lesquelles il y a deux couches d'air d'épaisseur $e_a'/2$ chacune. Pourquoi est-ce mieux ?
\end{questions}

\end{exercise}
