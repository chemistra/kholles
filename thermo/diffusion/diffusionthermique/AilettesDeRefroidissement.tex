\begin{exercise}{Ailette de refroidissement}{3}{Spé}
{Diffusion thermique}{lelay}

On veut refroidir un composant à la température $T_i$ à l'aide de l'air extérieur à la température $T_e$. Pour cela, on installe sur une paroi extérieure de ce composant une ailette de refroidissement consistant en un cylindre de métal ($c = 9\cdot10^2$~J.kg$^{-1}$.K$^{-1}$, $\lambda = 80$ W.m$^{-1}$.K$^{-1}$) de rayon $r$ et de longueur $L$. On considère que les échanges thermiques entre l'air et le cylindre se font par flux conducto-convectif, c'est à dire que le flux de chaleur passant du cylindre à l'air ambiant à travers une surface $\delta S$ est donné par la loi de Newton $\phi = h \delta S (T - T_e)$ où $T$ est la température locale de l'ailette, avec $h = 20$ W.m$^{-2}$.K$^{-1}$. On se place en régime permanent et on suppose que la température locale de l'ailette ne dépend que de $x$ la distance à la paroi.

\begin{questions}
    \questioncours Flux conducto-convectif : modèle, limites.
    \question Proposer une justification à la dernière hypothèse formulée dans l'énoncé.
    \question Effectuer un bilan d'énergie sur une tranche du cylindre de longueur $\dd{x}$.
    \question En déduire l'équation différentielle vérifiée par $T$. Préciser quelles sont les conditions aux limites du problème
    \question Donner $T(x)$ en supposant la longueur $L$ de l'ailette infinie.
    \question Quel est le flux de chaleur en $x=0$ ? Que serait-il si il n'y avait pas d'ailette ? En déduire l'efficacité de l'ailette, qui est le rapport du flux en $x = 0$ avec et sans ailette.
    \question Donner la valeur de l'efficacité pour $r = 1$ mm.
    \question À quelle condition l'approximation $L \approx \infty$ justifiée ? Est-elle légitime si on prend $L = 20$ cm ?
\end{questions}


\end{exercise}

\begin{solution}

\begin{questions}
    \questioncours $h = \lambda/e$
    \question Rayon faible devant la longueur, pas de variation radiale de température.
    \question Il faut faire le raisonnement du cours pour retrouver l'équation de la diffusion en 1D avec cette fois ci le terme d'échange avec l'extérieur. Bilan d'énergie
    \begin{itemize}
        \item Variation d'énergie interne pendant $\dd{t}$ (volume constant) : $\dd{U} = \rho \, \delta V \, c \dd{T}$
        \item Apport d'énergie depuis la gauche : $\lambda j_x \pi r^2 \dd{t}$
        \item Apport d'énergie depuis la droite : $-\lambda j_{x+\dd{x}}\pi r^2 \dd{t}$
        \item Perte d'énergie sur les cotés : $- h \dd{x} 2 \pi r (T-T_e) \dd{t}$
    \end{itemize}
    d'où
    \begin{align*}
    \pi r^2 \dd{x} \rho c \pdv{T}{t} &= \lambda j_x \pi r^2 - \lambda j_{x+\dd{x}}\pi r^2 - h \dd{x} 2 \pi r (T-T_e) \\
    \pdv{T}{t} &= \frac{\lambda}{\rho c} \pdv[2]{T}{x} - \frac{2h}{\rho c r}(T-T_e)
    \end{align*}
    
    \question  On trouve 
    $$\pdv{T}{t} = \frac{\lambda}{\rho c} \pdv[2]{T}{x} - \frac{2h}{\rho c r}(T-T_e)$$
    On peut se mettre en régime permanent (en combien de temps est-il atteint ? $\tau \sim L^2/D_\text{th}$) et on a alors
    $$ \pdv[2]{T}{x} - \frac{2h}{\lambda r}(T-T_e) = 0$$
    dont les solutions sont de la forme
    $$
    T(x) = T_e + A e^{-x/\ell} + B e^{x/\ell}
    $$
    avec $\ell = \sqrt{\lambda r/2h}$.
    
    Les conditions aux limites sont $$T(x = 0) = T_i \qqtext{(continuité de la température)}$$  et $$\pdv{T}{x}(x = L) = \frac{h}{\lambda}(T(x=L) - T_e) \qqtext{(continuité du flux thermique)}$$  (attention, on n'a PAS $T(x = L) = T_e$)
    
    \question Pour des raisons physiques on a $B = 0$ et alors $A = T_i - T_e$ soit $$
    T(x) = T_e + (T_i - T_e) e^{-x/\ell}
    $$
    
    \question En x = 0, le jlux est 
    $$ \pi r^2 \lambda \pdv{T}{x}(x = 0) = \pi r^2 \lambda\frac{1}{\ell} (T-T_e)$$
    et si il n'y avait pas la barre ce serait simplement 
    $$\pi r^2 h (T-T_e)$$
    
    \question $$ \eta = \frac{\lambda}{\ell h} = \sqrt{\lambda}{h r}$$
    on trouve 88.
    
    
    \question $L \gg \ell$. On a $\ell \sim 4$ cm et donc bof.
\end{questions}
\end{solution}
