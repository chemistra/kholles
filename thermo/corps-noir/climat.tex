
% Niveau :      PC
% Discipline :  Thermo
%Mots clés :    Rayonnement du corps noir

\begin{exercise}{Effet de serre}{2}{Spé}
{Thermodynamique, Rayonnement du corps noir}{bermu}

\begin{questions}
    \questioncours Qu'est-ce qu'un corps noir ? Rappeler les lois du déplacement de Wien et la loi de Stefan--Boltzmann.
    \question Quelle est la température effective $T_\text{ef}$ du soleil ? Évaluez le flux énergétique $I_0$ de ce dernier.
\uplevel{99,9\% du bilan énergétique de la Terre étant dû au rayonnement du Soleil, nous allons établir un modèle thermodynamique de la terre basé sur les échanges Soleil--Terre.}    
    \question Justifier que l'on puisse considérer la Terre comme un corps noir à condition de corriger le flux lumineux incident $I_0$ par $I_0 (1-A)$, $0<A<1$ étant l'albédo de la Terre. \\
    Tracer schématiquement le spectre en émission de la Terre.
    \question Quelle est la température d'équilibre $T_s$ en surface de la Terre ? Cette valeur vous paraît-elle cohérente ?
\uplevel{En fait, afin d'évaluer correctement la température de la terre il nous faut considérer le rôle majeur de l'atmosphère.}
    \question Rappeler ce qu'est l'effet de serre et le modéliser pour la Terre. \\
    On justifiera que l'atmosphère laisse passer intégralement les rayons du Soleil et absorbe les infrarouge.
    \question Calculer la température d'équilibre de la Terre pour $T\simeq 0$. Cela vous paraît-il  plus cohérent ?
    \question Qu'en est-il si on considère que l'atmosphère est un corps gris qui absorbe $\epsilon$ ($0<\epsilon<1$) dans l'IR et $\alpha$ ($0<\alpha<1$) dans le visible et que l'atmosphère et la surface de la Terre ont un albédo dans le visible $A_a$ et $A_s$ ? \\
    \'Etudier les cas limites.
    \question De quels facteurs environnementaux dépendent ces paramètres ?
\uplevel{L'effet de serre à un rôle majeur dans le réchauffement global ; si vous en doutez encore, le chimiste Svante Arhénius l'avait déjà prédit en 1895, mettez-vous à la page.}
    \question Faire un schéma des différentes boucles de rétroaction entraînant le réchauffement global.

%%% Mon bébé : c'est là où tu mets Fauve à l'honneur !

\end{questions}

\paragraph{Données :}
\begin{itemize}
    \item constante de Wien $b \simeq \dfrac{h c}{5 k_\textsc{b}} = 2,898\times 10^{-3}$ m$\cdot$K,
    \item constante de Stefan--Boltzmann $\sigma = 5,670\times 10^{-8}$ 
    $\mathrm{W\cdot m^{-2}\cdot K^{-4}}$,
    \item albédo de la Terre $A = 0,29$.
\end{itemize}
\end{exercise}