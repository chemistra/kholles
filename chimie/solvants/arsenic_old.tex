% Niveau :      PCSI *
% Discipline :  Chimie Orga I
% Mots clés :   Spectrométrie UV-visible, Réactions acidobasiques

\begin{exercise}{\emph{Arsenic et vieilles dentelles}}{2}{PCSI}
{Atomistique,Classification périodique, Structure électronique}{bermu}

La famille des pnictogènes est la colonne du tableau périodique dans laquelle se trouve notamment l'arsenic (As) et l'antimoine (Sb).

\begin{questions}
    \questioncours Liaisons covalentes, liaisons hydrogène, liaisons Van der Waals. Analogies et différences. Comparaison des énergies associées.
    
    \question L'arsenic, de numéro atomique 33, n'a qu'un isotope stable, de nombre de masse 75.
    \begin{parts}
        \part Donner le nombre de protons et le nombre de neutrons de cet isotope et le représenter sous la forme $^{A}_{Z}$As.
    
        \part Quelle est sa structure électronique à l'état fondamental ? En déduire la position de l'arsenic dans le tableau périodique.
    
        \part Justifier les principaux nombres d'oxydation +III et +V.
    \end{parts}
    
    \question Quel sont les pnictogènes $X$ et $Y$ situés dans les périodes 2 et 3, respectivement ?
    
    \question L'antimoine, se situe dans la cinquième période de la colone des pnictogènes. Il se trouve principalement sous forme de deux isotopes $^{121}$Sb et $^{123}$Sb.
    \begin{parts}
        \part Quel est le numéro atomique de Sb ?
        \part Sachant que la masse atomique de l'antimoine est de 121,8 g$\cdot$mol$^{-1}$, quelle est l'abondance relative de chaque isotope ?
    \end{parts}

\uplevel{On étudie les hydrures des éléments pnictogènes.}
    \question On s'intéresse tout d'abord à l'hydrure d'arsenic, ou arsine, qui s'écrit de manière générique AsH$_n$.
    \begin{parts}
        \part Au regard des questions précédentes, suggérer deux  formules et structures de Lewis probables pour et leur géométrie attendue selon la théorie VSEPR.
    
        \part L'angle mesuré entre les deux liaisons As--H est de 92,1$^\circ$. Conclure. 
        
        \part Les autres hydrures de pnictogènes ont également cette structure. Justifier rapidement ?
        
        \part (\emph{Question facultative}) Comment s'appellent les composés $X$H$_n$ et $Y$H$_n$ ?
\end{parts}

\question On donne dans la table ci-dessous quelques propriétés chimiques de ces hydrures.
\begin{table}[H]
    \centering
    \begin{tabularx}{.7\linewidth}{r|CCCC}
        Période & 2 ($X$) & 3 ($Y$) & 4 (As) & 5 (Sb) \\ \hline\hline
        Température d'ébulition ($^\circ$C) & $-33$ & $-88$ & $-63$ & $-17$ \\ 
        Solubilité (vol. / vol.) & 826 & 0,26 & 0,20 & 0,19 \\ \hline
    \end{tabularx}
    \caption{Comparaisons de propriétés chimiques des hydrures de pnictogènes (CTNP).}
\end{table}

\begin{parts}
    \part Justifier l'évolution de la température d'ébullition des hydrures de pnictogènes observée dans la table ci-dessus.

    \part Les atomes d’arsenic et d’hydrogène ont des électronégativités voisines. Comparer la polarité des molécules $X$H$_n$ et AsH$_n$. Préciser sur un schéma clair l’orientation  du moment dipolaire.
    
    \part En déduire une explication de l'évolution des solubilités des des hydrures de pnictogènes observée dans la table ci-dessus.
    
\end{parts}

\end{questions}

\end{exercise}