% Niveau :      PCSI *
% Discipline :  Chimie Orga I
% Mots clés :   Spectrométrie UV-visible, Réactions acidobasiques

\begin{exercise}{L'azote dans tout ses états}{2}{PCSI}
{Atomistique,Classification périodique, Structure électronique}{bermu}


\begin{questions}
    \questioncours Limites et extension du modèle de Lewis : liaison délocalisées et hybrides de résonance.
    
\uplevel{Dans cet exercice, on étudie différentes espèces de l'azote.}

    \question On s'intéresse tout d'abord aux composés chlorés de l'azote.
    \begin{parts}
        \part Rappeler rapidement le numéro atomique, la structure électronique, la position dans la classification périodique et les propriétés chimiques de l'azote.
        
        \part Les composés NC$\ell_3$, PC$\ell_3$, et PC$\ell_5$ existent, mais pas NC$\ell_5$. Expliquer ces quatre faits.
    
        \part En réalité, PC$\ell_5$ est un mélange de PC$\ell_4^{+}$ et de PC$\ell_6^{-}$. Donner la représentation de Lewis ainsi que la géométrie de ces ions.
    \end{parts}
    
    \question L'azote existe sous plusieurs formes oxydées : \\
    \begin{tabular}{lll}
        -- le monoxyde d'azote NO ; & -- le dioxyde d'azote NO$_2$ ; & -- l'ion nitronium NO$_2^+$ \\
        & -- l'ion nitrite NO$_2^-$ ;  & -- l'ion nitrate NO$_3^-$.
    \end{tabular}
    
    (il sera recommandé d'organiser sa restitution sous forme de tableau.)
    \begin{parts}
        \part Justifier les nombres d'oxydation observés.
        
        \part Donner les structures de Lewis correspondantes.
        
        \part Donner la géométrie de ces molécules dans la théorie VSEPR et comparer avec les résultats expérimentaux (ci-dessous).
        
        \part Lesquelles de ces molécules ont un moment dipolaire non nul ?
    \end{parts}
    
    \begin{table}[H]
    \centering
    \begin{tabularx}{.7\linewidth}{r|CCCCC}
        Molécule & NO & NO$_2$ & NO$_2^+$ & NO$_2^-$ & NO$_3^{-}$\\ \hline\hline
        Angle O--N--O ($^\circ$) & --- & 134 & 180 & 115 & 120 \\ 
        Longeur N--O (pm) & 115 & 120 & 114 & 124 & 127 \\ \hline
    \end{tabularx}
    \caption{Paramètres géométriques .}
\end{table}
    
    \question Il existe aussi des composés plus complexes issus de dimérisation ou trimérisation des oxydes d'azote, parmis eux :
    
    \begin{tabular}{lll}
        -- le protoxyde d'azote N$_2$O ; & -- le sesquioxyde d'azote N$_2$O$_3$ ; & -- le péroxyde d'azote N$_2$O$_4$ ;
    \end{tabular}

    \begin{parts}
        \part Donner les structures de Lewis de ces molécules ;
        \part Donner l'équation de réaction de formation de ces oxydes à partir d'oxydes simples.
    \end{parts}

\end{questions}

\end{exercise}