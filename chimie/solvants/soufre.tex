% Niveau :      PCSI *
% Discipline :  Chimie Orga I
% Mots clés :   Spectrométrie UV-visible, Réactions acidobasiques

\begin{exercise}{Soufre ou souffre ?}{1}{PCSI}
{Atomistique,Classification périodique, Structure électronique}{bermu}


\begin{questions}
    \questioncours Règles du duet et de l'octet. Limites. Schéma de Lewis.
    
\uplevel{Dans cet exercice, on étudie différentes espèces oxydées du soufre.}
    
    \question Le soufre existe sous plusieurs formes oxydées : \\
    \begin{tabular}{lll}
        -- le monoxyde de soufre SO ; & -- le dioxyde de soufre SO$_2$ ; & -- le trioxyde de soufre SO$_3$ ; \\
       & -- l'ion sulfite SO$_3^{2-}$ ; & -- l'ion sulfate SO$_4^{2-}$.
    \end{tabular}
    
    (il sera recommandé d'organiser sa restitution sous forme de tableau.)
    \begin{parts}
        
        \part Donner les structures de Lewis correspondantes.
        
        \part Donner la géométrie de ces molécules dans la théorie VSEPR et comparer avec les résultats expérimentaux (ci-dessous).
        
        \part \`A partir des résultats expérimentaux de la table ci-dessus, commenter les moments dipolaires de ces molécules.
    \end{parts}
    
    \begin{table}[H]
    \centering
    \begin{tabularx}{.7\linewidth}{r|CCCCC}
        Molécule & SO & SO$_2$ & SO$_3$ & SO$_3^{2-}$ & SO$_4^{2-}$ \\ \hline\hline
        Angle O--S--O ($^\circ$) & --- & 118 & 120 & 106 & 109.5 \\ 
        Longeur S--O (pm) & 148 & 143 & 142 & 151 & 149  \\
       Moment dipolaire (D) & 1,63 & 1,65 & $< 0,1$ & 1,90 & $< 0,1$ \\ \hline
    \end{tabularx}
    \caption{Paramètres géométriques .}
\end{table}

\end{questions}

\end{exercise}