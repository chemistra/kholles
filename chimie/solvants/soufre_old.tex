% Niveau :      PCSI *
% Discipline :  Chimie Orga I
% Mots clés :   Spectrométrie UV-visible, Réactions acidobasiques

\begin{exercise}{Soufre ou souffre ?}{2}{PCSI}
{Atomistique,Classification périodique, Structure électronique}{bermu}


\begin{questions}
    \questioncours Après avoir rappelé le principe de la représentation de Lewis et les règles du duet et de l'octet, justifier ces dernières par l'approche contemporaine de la configuration électronique des atomes. On abordera aussi le cas des exceptions de remplissage.
    
\uplevel{Dans cet exercice, on étudie différentes espèces du soufre.}

    \question Le soufre moléculaire se trouve sous forme cyclique S$_8$ (cycle à 8 atomes).
    \begin{parts}
        \part Rappeler rapidement le numéro atomique, la structure électronique, la position dans la classification périodique et les propriétés chimiques du soufre.
        
        \part Donner la structure de Lewis de S$_8$.
    
        \part Dans cette structure, les angles S--S--S valent 107$^\circ$.
    \end{parts}
    
    \question Le soufre existe sous plusieurs formes oxydées : \\
    \begin{tabular}{lll}
        -- le monoxyde de soufre SO ; & -- le dioxyde de soufre SO$_2$ ; & -- le trioxyde de soufre SO$_3$ ; \\
       & -- l'ion sulfite SO$_3^{2-}$ ; & -- l'ion sulfate SO$_4^{2-}$.
    \end{tabular}
    
    (il sera recommandé d'organiser sa restitution sous forme de tableau.)
    \begin{parts}
        \part Justifier les nombres d'oxydation observés.
        
        \part Donner les structures de Lewis correspondantes.
        
        \part Donner la géométrie de ces molécules dans la théorie VSEPR et comparer avec les résultats expérimentaux (ci-dessous).
        
        \part \`A partir des résultats expérimentaux de la table ci-dessus, retrouver les moments dipolaires de ces molécules. \uplevel{\textsfbf{Aide :} l'angle entre l'axe de symétrie de SO$_3^{2-}$ et liaisons S--O est de 67$^\circ$.}
    \end{parts}
    
    \begin{table}[H]
    \centering
    \begin{tabularx}{.7\linewidth}{r|CCCCC}
        Molécule & SO & SO$_2$ & SO$_3$ & SO$_3^{2-}$ & SO$_4^{2-}$ \\ \hline\hline
        Angle O--S--O ($^\circ$) & --- & 118 & 120 & 106 & 109.5 \\ 
        Longeur S--O (pm) & 148 & 143 & 142 & 151 & 149  \\
       Moment dipolaire (D) & 1,63 & 1,65 & $< 0,1$ & 1,90 & $< 0,1$ \\ \hline
    \end{tabularx}
    \caption{Paramètres géométriques .}
\end{table}
    
    \question Lewis et Langmuir ont proposé une formule qui permet de calculer les charges partielles $\delta$ portées par les atomes d’une molécule à partir de la structure de Lewis de la molécule :
    $$\delta_\textsc{a} = - 2 \sum_{\textsc{b} \text{ lié à } \textsc{a}} \dfrac{\chi_\textsc{a} - \chi_\textsc{b}}{\chi_\textsc{a} + \chi_\textsc{b}},$$
    où $\chi_\textsc{a}$ et $\chi_\textsc{b}$ sont les électronégativités des atomes A et B.

    \begin{parts}
        \part Justifier qualitativement cette formule.
        
        \part Retrouver le moment dipolaire de SO.
        
        \uplevel{\textsfbf{Données :} électronégativités dans l'échelle de \textsc{Pauling} $\chi_\textsc{o} = 3,44$, $\chi_\textsc{s} = 2,58$.}
    \end{parts}

\end{questions}

\end{exercise}