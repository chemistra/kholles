\begin{exercise}{Solubilité de l'arséniate de cuivre}{1}{PCSI}
{solubilite,Equilibres chimiques}{lelay}

L'arséniate de cuivre (II) Cu$_3$(AsO$_4$)$_2$ lorsqu'il est plongé dans l'eau libère des ions cuivre (II) Cu$^{2+}$ et des ions arséniates AsO$_4^{3-}$. Sa solubilité dans l'eau pure à 25$^o$C est de 1.74~g.L$^{-1}$. 

\begin{questions}

    \question En déduire sa solubilité molaire et son $pK_s$.
    
    \uplevel{On mélange un volume $V_1 = 10$~mL de solution de sulfate de cuivre (II) à $c_1 = 1.6\cdot10^{-2}$~mol.L$^{-1}$ et un volume $V_2 = 40$~mL de solution d'arséniate de sodium à $c_2 = 2.0\cdot10^{-2}$~mol.L$^{-1}$.}

    \question Observe-t-on l'apparition d'un précipité ?

    \question Même question avec $c_1 = 8.0\cdot 10^{-2}$~mol.L$^{-1}$.
    
\end{questions}
Données : $M(\text{Cu}) = 63.5$~g.mol$^{-1}$, $M(\text{As}) = 75$~g.mol$^{-1}$ et $M(\text{O}) = 16$~g.mol$^{-1}$
\end{exercise}

\begin{solution}
\begin{questions}

    \question Masse molaire totale 468.5 g/mol d'où solubilité molaire 3.71~mmol/L
    
    $n$ moles dans un litre d'eau pure donnent $3s$ moles de Cu$^{2+}$ et $2s$ moles de AsO$_4^{3-}$ d'où $K_s = (3s)^3(2s)^2  =108 s^5$ avec $s = 3.71$~mmol/L d'où $K_s = 7.59\cdot 10^{-11}$ et $pK_s = -\log K_s = 10.1$
    
    \question Concentration en cuivre : $3.2$~mmol/L ; en ions arséniates $16$~mmol/L ; quotient réactionnel $Q = 8.4\time 10^{-12}$.
    
    $Q < K_s$ \textbf{donc il n'y a pas précipitation}
    
    \question Concentration en cuivre : $16$~mmol/L ; quotient réactionnel $Q = 1.0\time 10^{-9}$.
    
    $Q > K_s$ \textbf{donc il y a précipitation}
    
\end{questions}
\end{solution}

\begin{exercise}{Effet d'ion commun}{1}{PCSI}
{solubilite,Equilibres chimiques}{lelay}

Le chlorure d'argent est un solide dont la solubilité dans l'eau pure est de 1.92~mg/L.

\begin{questions}

    \question En déduire sa solubilité molaire et son $pK_s$.
    
    \uplevel{On verse maintenant du chlorure d'argent non pas dans l'eau pure mais dans une solution de chlorure de potassium de concentration $c$.}

    \question Sans calculs, que dire de la solubilité du chlorure d'argent dans ce cas ?

    \question Trouve la solubilité du chlorure d'argent pour $c = 10$~mmol/L.
    
\end{questions}
Données : $M(\text{Ag}) = 107.87$~g.mol$^{-1}$, $M(\text{Cl}) = 35.45$~g.mol$^{-1}$
\end{exercise}

\begin{solution}
\begin{questions}

    \question Masse molaire totale 143.32 g/mol d'où solubilité molaire $1.341\times 10^{-5}$~mol/L
    
    $n$ moles dans un litre d'eau pure donnent $s$ moles de Ag$^{+}$ et $s$ moles de Cl${-}$ d'où $K_s = s^2$ avec $s = 1.341\times 10^{-5}$~mol/L d'où $pK_s = -\log K_s = 9.752$
    
    \question Il y a déjà du chlore ionique dans la solution, la solubilité sera donc moindre.
    
    \question On a cette fois $Ks = s (c + s)$. Si on est malin on voit que nécessairement $s \ll c$, sinon il faut résoudre une équation du second degré. À la fin, $s = 1.8\times 10^{-8}$ mol/L
    
\end{questions}
\end{solution}

\begin{exercise}{Mélange d'halogénure d'argent}{1}{PCSI}
{solubilite,Equilibres chimiques}{lelay}

On dispose dans un bécher d’un volume $V_0 = 100$~mL d’une solution aqueuse de chlorure de sodium $C_1 = 100$~mmol/L et de bromure de sodium $C_2 = 200$~mmol/L. Les deux composés sont supposés parfaitement dissous. On dispose d’autre part d’une solution de nitrate d’argent de concentration $C = 1.00$~mol/L dans une burette de 50 mL.

\begin{questions}

    \question Tracer les domaines d'existence des précipités AgCl et AgBr en fonction de $p\text{Ag}$ ($p\text{Ag}=10^{-[Ag^+]/C^0}$)
    
    \uplevel{On introduit maintenant progressivement le nitrate d'argent de la burette dans le becher.}

    \question Sans calculs, que va-t-il se passer ? Peut-on récupérer un précipité absolument pur ? Lequel ?

    \question Quel est la quantité maximale théorique de ce précipité pur qu'il est possible de récupérer ? Calculer le rendement de cette opération.
    
\end{questions}
Données : $pK_s(\text{AgCl}) = 9.8$, $pK_s(\text{AgBr}) = 12.3$.
\end{exercise}

\begin{solution}
\begin{questions}

    \question Masse molaire totale 143.32 g/mol d'où solubilité molaire $1.341\times 10^{-5}$~mol/L
    
    $n$ moles dans un litre d'eau pure donnent $s$ moles de Ag$^{+}$ et $s$ moles de Cl${-}$ d'où $K_s = s^2$ avec $s = 1.341\times 10^{-5}$~mol/L d'où $pK_s = -\log K_s = 9.752$
    
    \question Quand $p\text{Ag}$ diminue, càd quand $[\text{Ag}^+]$ augmente, AgBr se forme, puis AgCl à partir du moment où $p\text{Ag}$ passe en dessous de 8.8. On peut donc obtenir AgBr pur.
    
    \question Le cas limite correspond à arrêter de verser lorsque $p\text{Ag} = 8.8$ i.e. $[Ag^+] = 10^{-8.8}$ M. 

    On a alors $[Br^-] = \frac{K_s}{[Ag^+]} = 10^{-3.5}$ M.

    Soit $V$ le volume versé et $n$ la quantité de solide formée. Alors on a 
    \begin{align*}
        C_2 V_0 &= [Br](V_0 + V) + n \\
        C V &= [Ag](V_0 + V) + n \\
    \end{align*}
    On en déduit $V(C+[Br]-[Ag]) = V_0(C_2 + [Ag]-[Br])$. Puisque $[Ag], [Br] \ll C_2, C$ alors $V\approx V_0 C_2/C = 20$~mL.
    D'où $n = C_2 V_0 - [Br] (V_0+V) = 0.200*0.1 - 10^{-3.5} (0.1 + 0.02) = 19.96$ mmol. Au début on en avait 20.0, d'où un rendement de 99.8 \%.
\end{questions}
\end{solution}