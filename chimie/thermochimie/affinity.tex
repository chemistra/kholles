% Niveau :      PCSI *
% Discipline :  Chimie Orga I
% Mots clés :   Spectrométrie UV-visible, Réactions acidobasiques

\begin{exercise}{Raffinage du nickel par le procédé Mond}{2}{PC,MP}
{Thermochimie, Affinité, Déplacement d'équilibre}{bermu}


    Du nickel de très haute pureté peut être obtenu par l’intermédiaire du nickel carbonyle (tétracarbonylenickel) Ni(CO)$_4$. Ce complexe se forme à température modérée et pression ordinaire par action du monoxyde de carbone gazeux CO$_\text{(g)}$ sur des pastilles de nickel Ni$_\text{(s)}$.
    Aucun autre métal n’est susceptible de réagir dans les mêmes conditions.
    
    Après séparation, le nickel carbonyle est décomposé selon la réaction inverse pour donner du métal de haute pureté.

\begin{questions}
    \question \'Ecrire l'équation de réaction, avec une stoechiométrie 1 pour le nickel.
    \question À l’aide des données thermodynamiques fournies, établir, en fonction de la température $T$, les expressions de l’enthalpie libre standard de la réaction, dans les domaines de température 0--43$^\circ$C et 43--200$^\circ$.
    
Pouvait-on prévoir le signe du coefficient de $T$ dans l’expression de l’enthalpie libre standard de réaction ?

    \question Tracer le graphe de l’enthalpie libre standard de réaction $\Delta_\text{r}G^\circ$ en fonction de la
température $T$.

Quelle est la température d’inversion de l’équilibre ?

    \question Quelle est la variance d’un système à l’équilibre constitué de nickel solide, de monoxyde de carbone et de nickel carbonyle gazeux ? Quel est l’effet sur ce système d’une augmentation de pression à température constante ?
    
    \question Quelle est la variance d’un système à l’équilibre constitué de nickel solide, de monoxyde de carbone et de nickel carbonyle liquide ? Quel est l’effet sur ce système d’une augmentation de température à pression constante ?
    
    \question Pour une pression totale de 1 bar, à quelle température sera-t-il judicieux de se placer pour former le nickel carbonyle et le séparer facilement des impuretés, peu volatiles, contenues dans le métal ? On choisira entre 20 $^\circ$C, 40 $^\circ$C, 50 $^\circ$C, 100 $^\circ$C, 150 $^\circ$C ou 200 $^\circ$C.
    
    \question Quelle température conviendra le mieux pour décomposer le nickel carbonyle sous une pression totale de 1 bar ? On choisira entre 20 $^\circ$C, 40 $^\circ$C, 50 $^\circ$C, 100 $^\circ$C, 150$^\circ$C.
    
    \paragraph{Données :} dans les conditions standard.
    
    \'Ebullition de Ni(CO)$_4$ : $T_\text{vap} = 43$ $^\circ$C, $\Delta_\text{vap}H^\circ = 30$ kJ$\cdot$mol$^{-1}$.
    
    Dans les CNTP :
    \begin{center}
\begin{table}[H]
    \qquad\begin{tabular}{r|cccc}
        Espèce & $\mathrm{Ni_{(s)}}$ & $\mathrm{CO_{(g)}}$ & $\mathrm{Ni(CO)_{4 (\ell)}}$ \\ \hline\hline
        $\Delta_\text{f}H^\circ$ (kJ$\cdot$mol$^{-1}$) & --- & $-111$ & $-632$ \\
        $S_m^\circ$ (J$\cdot$mol$^{-1}\cdot$K$^{-1}$) & $30$ & $198$ & $320$ \\ \hline
    \end{tabular}
\end{table}
    \end{center}
\end{questions}

\end{exercise}


% Niveau :      PCSI *
% Discipline :  Chimie Orga I
% Mots clés :   Spectrométrie UV-visible, Réactions acidobasiques

\begin{exercise}{Oxydation du soufre}{2}{PC,MP}
{Thermochimie, Affinité, Déplacement d'équilibre}{bermu}

    Nous allons nous intéresser au passage du dioxyde de soufre SO$_{2 \text{(g)}}$ au trioxyde de soufre SO$_{3 \text{(g)}}$ par l'action de l'oxygène O$_{2 \text{(g)}}$. Ce passage se fait essentiellement au contact d’un catalyseur spécifique, le pentaoxyde de vanadium V2O5.

\begin{questions}
    \question \'Ecrire l'équation de réaction, avec une stoechiométrie 1 pour le dioxygène.
    
    \question Calculer son enthalpie standard de réaction et son entropie standard de réaction à $T = 300$ K et en déduire l'expression de l’enthalpie libre standard de réaction pour toute température $T$.
    
    \question Quelle est la température d’inversion de l’équilibre ? Préciser l’expression numérique de $\ln K^\circ(T)$ pour toute température ($K^\circ$ désigne la constante d’équilibre).
    
    \uplevel{Les industriels travaillent vers $T = 430$ $^\circ$C sous $p = P^\circ = 1$ bar avec un léger excès de dioxygène
provenant de l’air par rapport à la quantité stoechiométrique 2 SO$_2$ pour 1 O$_2$. Nous allons interpréter ces choix.}

    \question Partons de $\lambda$ moles de dioxygène pur et de $1 - \lambda$ moles de dioxyde de soufre. Dresser un tableau d’avancement et donner la relation liant à l’équilibre le paramètre $\lambda$, l’avancement $\xi$, la constante d’équilibre $K^\circ$ et la pression totale $p$.
    
    \question À $T$ et $p$ fixées, pour quelle valeur de $\lambda$ a-t-on un avancement $\xi$ maximal ?
    
    \uplevel{Nous supposons que nous partons désormais des proportions stoechiométriques : 2 mol de dioxyde de soufre pour 1 mol de dioxygène et que l’équilibre est atteint.}
    
    \question Quelle est l’influence d’un ajout de diazote à $T$ et $p$ constantes sur l’état d’équilibre ?

    \question Conclure sur la meilleure composition théorique du mélange initial.
    
    \question Comment interpréter le choix de la pression atmosphérique par les industriels ?
    
    \paragraph{Données :} dans les CNTP
    \begin{center}
\begin{table}[H]
    \qquad\begin{tabular}{r|cccc}
        Espèce & $\mathrm{SO_{2 (g)}}$ & $\mathrm{SO_{3 (g)}}$ & $\mathrm{O_{2 (g)}}$ \\ \hline\hline
        $\Delta_\text{f}H^\circ$ (kJ$\cdot$mol$^{-1}$) & $-297$ & $-396$ & --- \\
        $S_m^\circ$ (J$\cdot$mol$^{-1}\cdot$K$^{-1}$) & $248$ & $257$ & $205$ \\ \hline
    \end{tabular}
\end{table}
    \end{center}
\end{questions}

\end{exercise}

% Niveau :      PCSI *
% Discipline :  Chimie Orga I
% Mots clés :   Spectrométrie UV-visible, Réactions acidobasiques

\begin{exercise}{Déplacements d'équilibre}{2}{PC,MP}
{Thermochimie, Affinité, Déplacement d'équilibre}{bermu}


\begin{questions}
    \questioncours Loi de Van't Hoff : interprétation et démonstration.
    
\begin{EnvUplevel}
    Dans cet exercice, on étudie le critère d'évolution d'une réaction sous la forme générale
    \begin{equation}
        \sum_i\nu_i A_i = 0, \tag*{de constante $K^\circ(T)$}
    \end{equation}
    où les $\nu_i$ sont les coefficients stoechiométriques écrits algébriquement avec la convention
    $$\left\lbrace\begin{array}{ll}
        \nu_i > 0 & \text{pour les espèces produites, \emph{i.e.} les produits ;}  \\
        \nu_i < 0 & \text{pour les espèces consommées, \emph{i.e.} les réactifs ;}  \\
        \nu_i = 0 & \text{pour les espèces inertes.}  \\ 
    \end{array}\right.$$
    
    On rappelle la définition de l'opérateur de Lewis $$\Delta_\text{r}X = \qty(\pdv{X}{\xi})_{P,T},$$
    $\xi$ étant l'avancement de la réaction et $X$ une grandeur quelconque ($G$, $H$, $S$, $V$...)
\end{EnvUplevel}
    \question \textsf{Quelques rappels}
    \begin{parts}
        \part Rappeler les relations entre $\Delta_\text{r}G^\circ$ et $K^\circ(T)$ ; entre $\Delta_\text{r}G$, $K^\circ$ et $Q$.
        \part Rappeler le critère d'évolution d'une réaction vers l'équilibre avec $Q/K^\circ$ d'une part et $\Delta_\text{r}G$ d'autre part, et montrer que ces deux critères sont équivalents.
    \end{parts}
    
    \uplevel{La réaction est constituée d'une phase aqueuse en contact avec une phase gazeuse. On partage d'un coté les $\nu_{i,\text{g}}$ gazeux, les $\nu_{i,\text{aq}}$ aqueux et les $\nu_{i,\ast}$ constitué des corps purs et du solvant.}
    
    \question Montrer qu'on peut décomposer $\Delta_\text{r}G$ de la manière suivante
    \begin{equation}
        \Delta_\text{r}G = \Delta_\text{r}G^\circ + RT\ln Q_\text{g} + RT\ln Q_\text{aq} + RT\ln Q_\ast. \tag{$\star$}
    \end{equation}
        On précisera les expressions de $Q_\text{g}$ en fonction de $x_{i,\text{g}} = n_{i,\text{g}}/n_\text{tot,g}$, $\nu_{i,\text{g}}$ et $P$ ; de $Q_\text{aq}$ en fonction de $c_{i,\text{aq}}$, $\nu_{i,\text{aq}}$ et enfin de $Q_\ast$.
    
    \question Partant de l'équilibre, on change la température $T \mapsto T + \dd{T}$ à pression constante.
    \begin{parts}
        \part Que vaut $\Delta_\text{r}G$ à l'équilibre ?
        \part Montrer que seule la variation de $\Delta_\text{r}G^\circ(T)$ contribue à changer $\Delta_\text{r}G(T)$ dans la relation $(\star)$.
        \part En déduire
        $$\qty(\pdv{\Delta_\text{r}G}{T})_{P,n_i,\text{éq}} = -\dfrac{\Delta_\text{r}H^\circ}{T},$$
        interpréter le sens chimique de cette relation.
    \end{parts}
    
    \question Partant de l'équilibre, on change la pression $P \mapsto P + \dd{P}$ à température constante.
    \begin{parts}
        \part Quel est le seul terme qui varie dans la relation $(\star)$ ?
        \part En déduire
        $$\qty(\pdv{\Delta_\text{r}G}{P})_{T,n_i,\text{éq}} = \dfrac{\Delta_\text{r}n_\text{tot,g}RT}{P} = \Delta_\text{r}V,$$
        où on a noté $\Delta_\text{r}n_\text{tot,g} = \sum_{i,\text{g}} \nu_i$ (justifier cette notation), interpréter le sens chimique de cette relation.
        
        Ce principe est connu sous le nom de Le Châtelier.
    \end{parts}
    
    \question Partant le l'équilibre, on augmente le volume de la solution $V_\text{sol} \mapsto V_\text{sol} + \dd{V_\text{sol}}.$
    \begin{parts}
        \part Réécrire $Q_\text{aq}$ en fonction des $n_i$, de $V_\text{sol}$, $n_{i,\text{aq}}$ et $\nu_{i,\text{aq}}$.
        \part En déduire
        $$\qty(\pdv{\Delta_\text{r}G}{V_\text{sol}})_{T,P,n_i,\text{éq}} = \dfrac{\Delta_\text{r}n_\text{tot,aq}RT}{V_\text{sol}},$$
        où on a noté $\Delta_\text{r}n_\text{tot,aq} = \sum_{i,\text{aq}} \nu_i$ (justifier cette notation), interpréter le sens chimique de cette relation.
    \end{parts}
        
    \question Conclure quant à un principe modérateur générique en chimie.
    
\end{questions}

\end{exercise}
