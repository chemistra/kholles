\documentclass[
 %nopagenumber, exercisesheet, %% L'un
 answers,                     %% Ou l'autre 
 %noexercisenumber,
]
{kholles}


\title{Le fascicule de khôlles}
\author{Guillaume Bermudez, Baptiste Corrège}
\date{}
\mail{guillaume.bermudez@ens.fr, \qquad baptiste.correge@ens.fr}

% \title{}
% \author{}
% \mail{}

%Refs
\bibresource{misc/sources}

% \usepackage{pdfpages}
% \usepackage{circuitikz}

%%%%% EXERCICES %%%%%
\begin{document}

\setcounter{part}{28} %\part{Mécanique}
\setcounter{section}{4} %\section{Mécanique RNG}
\setcounter{exercise}{0} %numexo

             %%%%%%%%%%%%%%%%%%%%%%%%%%%%%%%%%%%%%%%%%%%%%%%%%%%%%%%%%%%%%%%%%%%%%%%%%%%%%%
             %%%%  ATTENTION DE BIEN METTRE LES NOUVEAUX EXOS DANS MAIN.TEX EGALEMENT  %%%%
             %%%%%%%%%%%%%%%%%%%%%%%%%%%%%%%%%%%%%%%%%%%%%%%%%%%%%%%%%%%%%%%%%%%%%%%%%%%%%%


\section{Filtrage linéaire --- Second ordre et plus...}
\input{elec/filtragelineaire/filtres_2nd_ordre}
% Niveau :      PCSI - PC
% Discipline :  Elec
% Mots clés :   Elec, Ordre 2

\begin{exercise}{Construction d'un passe bas}{1}{Sup,Spé}
{\'Electrocinétique,Filtrage linéaire, Second ordre}{lelay}

\begin{questions}
    \questioncours Présentez brièvement les différents types de filtres au programme
    \question À l'aide des composants électroniques usuels, construire un passe-bas d'ordre 2 simple. Vérifier son comportement par une analyse asymptotique.
    \question Donner la fonction de transfert de ce filtre. Tracer le diagramme de Bode asymptotique associé. 
    \question Comment choisir R, L et C pour avoir une fréquence de coupure de 1 kHz et une résonnance d'amplitude 20 dB ? 
\end{questions}
\end{exercise}

\begin{solution}
    \begin{circuit}
      \draw
      (0,0) to [short] (5,0)
      to [open, v_=$s$, *-o] (5,2) 
      to [short] (4,2)
      (0,0) to [open, v^=$e$,*-o] (0,2)
      to [R, l=$R$] (2,2)
      to [L, l=$L$, -*] (4,2) 
      to [C, l^=$C$] (4,0);
    \end{circuit}
    
    Ensuite, $\omega_0 = \frac1{\sqrt{LC}} = 10$ et $G_{dB} = 20\log{H_{max}} = 20$ donc $H_{max} \simeq Q = \frac1R \sqrt{\frac{L}{C}}= 10$
\end{solution}
% Niveau :      PCSI - PC
% Discipline :  Elec
% Mots clés :   Elec, Ordre 2

\begin{exercise}{Filtre bouchon}{1}{Sup,Spé}
{\'Electrocinétique,Filtrage linéaire, Second ordre}{lelay}

On cherche à décrire le circuit suivant :

\begin{circuit}
      \draw
      (0,0) to [short] (5,0)
      to [open, v_=$s$, *-o] (5,2) 
      to [short] (2,2)
      (0,0) to [open, v^=$e$, *-o] (0,2)
      to [R, l=$R$] (2,2)
      to [L, l=$L$, *-*] (2,0) 
      (3.5,0) to [C, l_=$C$, *-*] (3.5,2);
\end{circuit}

\begin{questions}
    \questioncours Pour chaque composant du circuit, donner la loi associée et le comportement asymptotique.
    \question Donner la fonction de transfert de ce filtre et tracer le diagramme de Bode associé.
    \question Quel est le comportement de ce filtre ?
    \question \'Etudier les limites $R\longrightarrow 0$ et $R\longrightarrow +\infty$. Pourquoi observe-t-on un tel comportement ?
\end{questions}
\end{exercise}

\begin{solution}
    $$H(\omega) = \dfrac{1}{1 + j Q \qty(\dfrac{\omega}{\omega_0} - \dfrac{\omega_0}{\omega})}, Q=R\sqrt{\dfrac{C}{L}}, \omega_0 = \dfrac{1}{\sqrt{LC}}$$
\end{solution}
\input{elec/filtragelineaire/passepas}
\input{elec/filtragelineaire/mutuelle_inductance}
\input{elec/filtragelineaire/laser}
% Niveau :      PCSI - PC
% Discipline :  Elec
% Mots clés :   Elec, Ordre 2

\begin{exercise}{Filtres en cascade}{3}{Sup,Spé}
{\'Electrocinétique,Filtrage linéaire, Troisième ordre +}{bermu}

On considère le circuit suivant :

\begin{circuit}
      \draw
      (0,0) to [short] (3,0)
      to [open, v_=$u_1$, *-o] (3,2) 
      to [short] (2,2)
      (0,0) to [open, v^=$u_0$, *-o] (0,2)
      to [R, l=$R$] (2,2) 
      to [C, l^=$C$, *-*] (2,0);
\end{circuit}

On note $Y_1 = jC\omega$.

\begin{questions}
    \questioncours Rappeler les lois de composition d'impédances ou d'admittance pour des dipôles en série et en parallèle.
    \question Donner la fonction de transfert $H_1(\omega)$ de ce filtre en fonction de $R$ et $Y_1$ puis de $\omega$ et tracer le diagramme de Bode associé. Quel est le comportement de ce filtre ?
\uplevel{On met désormais ce filtre en cascade avec un autre filtre identique.}
    \question Pourquoi la fonction de transfert de ce nouveau filtre n'est pas $H_2 = {H_1}^2$ ? \\
        Comment pourrait-on obtenir un filtre avec une telle fonction de transfert ?
    \question Montrer que la fonction de transfert $H_2(\omega)$ de ce filtre peut s'écrire
    $$H_2 = \dfrac{H_1}{1+R Y_2},$$
    avec $Y_2$, l'admittance d'un dipôle que l'on représentera et dont on donnera la  l'expression.
    \question Représenter le diagramme de bode de $H_2$.
\uplevel{On met désormais $n$ filtres de ce type en cascade.}
    \question Montrer qu'on peut écrire le circuit comme suit :
    \begin{circuit}
      \draw
      (0,0) to [R, l=$R$] (2,0)
      to [R, l=$Y_n$] (4,0) ;
      \draw (4,-.4) to [open, v^=$u_0$] (0,-.4) ;
      \node [ocirc] at (0,0) {} ;
      \node [circ] at (4,0) {} ;
\end{circuit}
et donner une relation de récurence entre $Y_n$, $Y_{n-1}$ et $Z_1$. \\ On ne demande pas de résoudre explicitement l'équation. La résolution de cette équation
    \question En déduire une relation de récurrence sur $H_n(\omega)$.
\uplevel{Prenons le cas $n=3$.}
    \question Quel est le comportement de ce filtre ? Pourquoi l'appelle-t-on un déphaseur.
\end{questions}
\end{exercise}

\begin{solution}
\begin{questions}
    \questioncours cf. cours
    \question
    $$H_1(\omega) = \dfrac{1}{1 + RY_1} = \dfrac{1}{1 + j\omega\tau_\textsc{rc}}, \qqtext{avec} \tau_\textsc{rc} = RC.$$
    \question Le diviseur de tension ne marche plus car on débite du courant, pour cela il faudrait un suiveur de tension entre les deux filtres.
    \question $Y_2$ est le composant
    \begin{circuit}
          \draw (-0.5,0) to [short, *-*] (0,0);
          \draw (0,-1) to [short] (0,1)
          to [R, l^=$R$] (2,1)
          to [R, l^=$Y_1$] (4,1)
          to [short] (4,-1)
          to [R, l^=$Y_1$] (0,-1);
          \draw (4,0) to [short, *-*] (4.5,0);
    \end{circuit}
    $$Y_2 = Y_1 + \dfrac{1}{R + 1/Y_1}$$
    $$H_2 = \dfrac{1}{(1 + RY_1)\qty(1 + RY_1 + R\dfrac{1}{1 + 1/Y_1})} = \dfrac{1}{1 + 3 RY_1 + R^2Y_1^2} = \dfrac{1}{1 + 3j\omega\tau - \omega^2\tau^2}$$
    \question 5
    \question Ainsi on peut représenter le circuit avec $Y_n$
    \begin{circuit}
          \draw (-0.5,0) to [short, *-*] (0,0);
          \draw (0,-1) to [short] (0,1)
          to [R, l^=$R$] (2,1)
          to [R, l^=$Y_{n-1}$] (4,1)
          to [short] (4,-1)
          to [R, l^=$Y_1$] (0,-1);
          \draw (4,0) to [short, *-*] (4.5,0);
    \end{circuit}
    $$Y_n = Y_1 + \dfrac{1}{R + 1/Y_{n-1}}$$
    \question Et on en déduit de proche en proche
    $$H_n = H_1\times H_2 \times \ldots \times H_{n-1} \times \dfrac{1}{1 + RY_n}$$
    \question 
\end{questions}

\paragraph{Annexe sans intéret}
$$\dfrac{1}{1 + RY_n} = \dfrac{\lambda + \lambda^{2n}}{1 + \lambda^{2n + 1}} \qqtext{avec $\lambda$ la solution de module < 1 de l'équation} \lambda (2 + RY_1 - \lambda) = 1$$

\end{solution}
% Niveau :      PCSI - PC
% Discipline :  Elec
% Mots clés :   Elec, Ordre 3

\begin{exercise}{Filtre de Butterworth}{2}{Sup}
{\'Electrocinétique,Filtrage linéaire, Troisième ordre +}{lelay}

On donne le filtre de Butterworth d'ordre trois :

\begin{circuit}
      \draw (0,0) 
      to [short, o-] (1,0)
      to [L, l=$L_1$] (3,0)
      to [L, l=$L_2$] (5,0)
      to [short, -*] (6,0);
      
      \draw (0,-2) 
      to [short, o-*] (6,-2);
      
      \draw (3,0) 
      to [C, l=$C$] (3,-2);
      
      \draw (5,0) 
      to [R, l=$R$] (5,-2);
\end{circuit}

tel que $L_1/L_2 = 8$ et $R^2 C/L_2 = 27/8$

\begin{questions}
    \questioncours Pour chaque composant du circuit, donner la loi associée et le comportement asymptotique.
    \question Qualitativement, donner le comportement asymptotique de ce filtre. Pourquoi selon-vous est-il appelé d'ordre trois ?
    \question Déterminer la fonction de transfert du filtre et tracer son gain en fonction de la fréquence. Donner la pulsation de coupure en fonction de $L_2$
    \question Quel est l'intérêt d'utiliser ce type de filtre ? On comparera aux filtres similaires vus en classes préparatoires.
\end{questions}
\end{exercise}

\begin{solution}
    % $$H(\omega) = \dfrac{1}{1 + j Q \qty(\dfrac{\omega}{\omega_0} - \dfrac{\omega_0}{\omega})}, Q=R\sqrt{\dfrac{C}{L}}, \omega_0 = \dfrac{1}{\sqrt{LC}}$$
    
    \begin{align*}
        H(\omega) &= \frac{1}{1 +\frac{L_1+L_2}{R} j \omega  + L_1 C (j\omega)^2 +\frac{L_1L_2C}{R}(j\omega)^3} \\ 
        &= \frac{1}{1 + 3\qty(\frac{3L_2}{R}j\omega) + 3 \qty(\frac{3L_2}{R}j\omega)^2 + \qty(\frac{3L_2}{R}j\omega)^3} \qqtext{d'apres l'enonce}\\
        &= \frac{1}{\qty(1 + j\frac{\omega}{\omega_0})^3} \qqtext{avec} \omega_0  = \qty(\frac{R}{L_1 L_2 C})^{1/3} = \frac{3L_2}{R}
    \end{align*}
    
    $$ G(\omega) = \frac{1}{\sqrt{1 + \qty(\frac{\omega}{\omega_0})^6}}$$
    
    % $$ \omega_0  = \qty(\frac{R}{L_1 L_2 C})^{1/3}$$
    
    Ce filtre est un passe bas qui présente une meilleure sélectivité que ceux étudiés en prépa (-60 dB / décade) et qui n'introduit pas de résonance (comme le filtre RLC d'ordre 2) qui pourrait déformer le signal. 
\end{solution}
% Niveau :      PCSI - PC
% Discipline :  Elec
% Mots clés :   Elec, Ordre 2

\begin{exercise}{Filtre déphaseur}{2}{Sup,Spé}
{\'Electrocinétique,Filtrage linéaire, Troisième ordre +}{lelay, X}

\begin{questions}
    \questioncours Rappeler le montage suiveur avec un A.O. et discuter de son intérêt pour construire des filtres d'ordre $n$.
    \question Donner la fonction de transfert du filtre ci dessous. Pourquoi l'appelle-t-on filtre déphaseur ?
\end{questions}

\begin{circuit}
      \draw
      %node [ground] at (4,0) {}
      (0,0) to [open, v_=$e$, *-o] (0,2)
      (0,2) to [R, l=$R$] (2,2) 
      to [C, l^=$C$, *-*] (2,0)
      (2,2) to [R, l=$R$] (4,2)
      to [C, l^=$C$, *-*] (4,0)
      (4,2) to [R, l=$R$] (6,2)
      to [C, l^=$C$, *-*] (6,0)
      (6,2) to [short] (8,2)
      (8,0) to [open, v_=$s$, *-o] (8,2)
      (0,0) to [short] (8,0);
\end{circuit}
\end{exercise}

\begin{solution}

$$H = \frac{1}{q(2+p)-(1+p)} \qq{avec} q=(2+p)(1+p)-1$$

\end{solution}
%\input{elec/filtragelineaire/filtres_1er_ordre} % AJOUTE
%\input{elec/filtragelineaire/filtres_2nd_ordre} % AJOUTE
%\input{elec/filtragelineaire/filtres_2nd_ordre_2} % AJOUTE
%\input{elec/filtragelineaire/ordre1BobImp}

%\input{electromag/magnetostat/breves_ampere} %MODIFIE

%\input{elec/usine} % MODIFIE

%\input{meca/mecarng/penduleRMN}% MODIFIE

%\input{meca/mecarng/deviation_est} % MODIFIE

%\input{meca/oscharm/sismo} % NOUVEAU


%\input{meca/mecapoint/balistique} % MODIFIE

% \section{\'Electrostatique}
% \input{electromag/electrostat/breves_gauss}
% \input{electromag/electrostat/cerceau}
% \input{electromag/electrostat/keesome}
% \input{electromag/electrostat/vache}

% \section{Magnétostatique}
% \input{electromag/magnetostat/breves_ampere}
% \input{electromag/magnetostat/ioffe}

% \section{Equations de Maxwell}
% \input{electromag/equationsMaxwell/divergence}
% \input{electromag/equationsMaxwell/schottky}

% \part{\'Electromagnétisme}
% \section{Induction}
% \input{electromag/induction/raildelaplace}
% \input{electromag/induction/raillaplace2}
% \input{electromag/induction/mesuredemutuelle}
% \input{electromag/induction/couplageinertiel}
% \input{electromag/induction/cadrequitombe}
% \input{electromag/induction/couplagevisqueux}

% \section{Ondes électromagnétiques}
% \input{electromag/ondesEM/OPPH}
% \input{electromag/ondesEM/vitphasegroupe}
% Rayonnement dipolaire (MP)
% \input{electromag/ondesEM/effet_de_peau}
% \input{electromag/ondesEM/microondes}

% % Guidé (MP)
% \input{electromag/ondesEM/microondes}

% % Plasma et rflexion métallique
% \input{electromag/ondesEM/effet_de_peau}
% \input{electromag/ondesEM/ondelongitudinale}

% % Milieux complexes
% \input{electromag/ondesEM/drude}
%\input{electromag/ondesEM/lamequartdonde}
% \input{electromag/ondesEM/loidebiot}

% %WTF
% \input{electromag/ondesEM/pressionderadiation}
% \input{electromag/ondesEM/reflexionmetallique}


\end{document}