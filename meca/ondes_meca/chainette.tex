% Niveau :      PC
% Discipline :  Méca
% Mots clés :   Chaînette

\begin{exercise}{Chaînette}{3}{Spé}
{Mécanique,Ondes mécaniques,Corde}{bermu}

Soit une chaînette homogène de masse linéique $\mu_\ell$, fixée en $\pm L/2$ à une altitude $H$.

À quelle condition la chaînette touche le sol à $z = 0$ ? Donner des systèmes physiques réels ayant un tel comportement et estimer les paramètres intervenants.

\paragraph{Données :} L’abscisse curviligne $s$ est telle qu’un élément de longueur de la corde $\dd{s}$ de la courbe vérifie
\begin{align*}
    \dd{s}^2 = \dd{x}^2 + \dd{z}^2 &= \qty(1+ z'(x)^2)\dd{x}^2. \\
    \mathrm{argsinh}'(x) &= \dfrac1{\sqrt{1+x^2}}.
\end{align*}
\end{exercise}