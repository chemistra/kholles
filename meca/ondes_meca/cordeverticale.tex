% Niveau :      PCSI
% Discipline :  Méca

\begin{exercise}{Corde verticale}{3}{Spé}
{Mécanique,Ondes mécaniques,Corde}{lelay}

Le but de cet exercice est de modéliser les mouvements d'une corde soumise à une excitation sinusoïdale. On commence par étudier les mouvements selon l'axe $y$ d'une corde de longueur $L$, tendue entre deux points placés sur l'axe $x$ avec une tension $T_0$. 
\begin{questions}
    \questioncours Équation de D'Alembert et notion d'impédance.
    \question Montrez que l'on peut au premier ordre modéliser les petits mouvements de la corde par une équation de D'Alembert. Quelle est la célérité associée ?
    \question Faire apparaître une "impédance mécanique", en considérant les variables $v_y$ et $T_y$.
    \uplevel{On étudie maintenant les mouvements transversaux d'une corde verticale accrochée un plafond ($z=0$), de masse $m$ et de longueur $L$.}
    \question Déterminer $T_0(z)$ la tension de la corde au repos.
    \question Déterminer l'équation de propagation dans la corde.
    \question L'extrémité de la corde en $z=L$ est libre de se mouvoir. On suppose que l'on impose un petit mouvement transversal à la corde $y(0,t) = y_0\cos(\omega t)$.
    \begin{parts}
        \part Justifier que l'on cherche les solutions sous la forme $y(z,t) = f(z)\cos(\omega t)$
        \part Établir l'équation différentielle vérifiée par $f$. L'adimensionner en utilisant comme longueur caractéristique $g/\omega^2$.
        \part La suite de cet exercice n'a pas encore été écrite, mais bravo si vous êtes arrivé jusque là !
    \end{parts}
\end{questions}
\end{exercise}