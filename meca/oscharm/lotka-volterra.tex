% Niveau :      PCSI
% Discipline :  Méca

\begin{exercise}{Dynamique proie-prédateur}{3}{Sup}
{Oscillateur harmonique}{lelay}

L'exercise utilise un des modèles développés par Lotka et Volterra en 1925 pour modéliser les systèmes avec des proies et des prédateurs en interaction.

On considère une île au large de la Bretagne sur laquelle vivent des sardines et des goélands :
\begin{itemize}
    \item les sardines se reproduisent entre elles,
    \item les goélands se battent entre eux,
    \item les goélands mangent les sardines et se reproduisent si ils se sont suffisamment bien nourris.
\end{itemize}
On note $s(t)$ la population de sardines et $g(t)$ la population de goélands, et on considère la modélisation suivante :
\begin{align*}
    \dv{s}{t} &= a\:s(t) - b\:s(t)\:g(t) \\
    \dv{g}{t} &= -c\:g(t) + d\:s(t)\:g(t)
\end{align*}
\begin{questions}
    \question Justifier chacun des termes de la modélisation ci-dessus.
\uplevel{Faire un rictus de douleur à la vue de ce système d'équations, car il est non-linéaire : il n'a pas de solution analytique ! Mais cela ne nous empêchera pas de l'étudier.}
    \question Montrer que ce système admet deux \emph{points fixes}.\\
    Des points fixes sont des points d'équilibre $(s_0, g_0)$ tels que si pour un $t_0$ donné on a $s(t_0) = s_0$ et $g(t_0) = g_0$, alors $\forall t > t_0$, $s(t) = s_0$ et $g(t) = g_0$.
    \question On se place au voisinage du point fixe intéressant et on prend $s(t) = s_0 + x(t)$ et $g(t) = g_0 + y(t)$. Écrire les équations au premier ordre en $x$ et en $y$ (\emph{i.e.} les termes comme $x^2$, $y^2$ et $xy$ sont négligés devant les autres).
    \question Montrer qu'alors $x$ et $y$ obéissent à une équation d'oscillateur harmonique dont on exprimera la pulsation $\omega$ en fonction des paramètres du problème.
    \question Les sardines et les goélands peuvent-elles coexister ?
\end{questions}
\end{exercise}