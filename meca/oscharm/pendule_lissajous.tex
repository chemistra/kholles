\begin{exercise}{Le pendule de Lissajous}{2}{Sup}
{Mécanique,Oscillateur harmonique,Pendule}{lelay}

\begin{questions}
    \questioncours Rappeler l'équation qui régit les mouvements d'un pendule de masse $m$ accroché à une corde de longueur $\ell$ et la résoudre pour des petits angles.
    
    \uplevel{On considère une corde de longueur $2L$ accrochée en haut de deux poteaux écartés de $D < 2L$. Au point le plus bas de la corde est accroché un pendule de masse $m$ et de longueur $R$.}
    \question Faire un schéma, et expliquer pourquoi on peut considérer les mouvements dans le plan des poteaux et dans le plan non horizontal qui lui est orthogonal comme indépendants.

    \question En considérant des petits angles, résoudre l'équation dans le plan des deux poteaux puis dans le plan orthogonal.
    \question Dessiner la trajectoire du mobile dans le plan horizontal pour $\omega_1 = \omega_2$
    \question Expliquer à quoi ressemblera (grossièrement) la trajectoire si $\frac{\omega_1}{\omega_2}$ est une fraction rationnelle
    \question A votre avis, Que si passe-t-il si $\frac{\omega_1}{\omega_2} $ est irrationnelle ?
\end{questions}
\end{exercise}