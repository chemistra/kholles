% Niveau :      PCSI
% Discipline :  Méca

\begin{exercise}{Ressorts en chaîne}{2}{Sup}
{Mécanique,Oscillateur harmonique,Ressort}{lelay}

Dans cet exercice, tous les ressorts ont une raideur $k$ et une longueur à vide $a$, et toutes les masses ont une masse $m$.
\begin{questions}
    \question Soit une masse entre deux murs, reliée à chacun d'eux par un ressort. Donner la position $x(t)$ de la masse en fonction du temps
    \question Soit le système mur - ressort - masse - ressort - masse - ressort - mur. Donner les équations qui régissent le système.
    \begin{parts}
        \part Donner les équations qui régissent le système.
        \part En utilisant le changement de variable $X = x_1+x_2$ et $Y = x_1 - x_2$, résoudre le système et donner $X(t)$ et $Y(t)$
        \part En déduire $x_1(t)$ et $x_2(t)$. Interpréter.
    \end{parts}
    \question Soit le même système qui précédemment avec $N$ masses. Donner l'équation qui régit la masse $i$.
    \question On s'intéresse au cas $N=3$. % Ils sont en prépa faut bien qu'ils s'habituent à calculer un jour quand même.
    \begin{parts}
        \part Donner les équations qui régissent le système.
        \part En utilisant le changement de variable $X = x_1-x_3$, $Y = x_1 + \sqrt2 x_2 + x_3$ et $Z = x_1 - \sqrt2 x_2 + x_3$, résoudre le système et donner $X(t)$, $Y(t)$ et $Z(t)$
        \part Pensez-vous qu'il soit possible de généraliser ce processus ?
    \end{parts}
\end{questions}
\end{exercise}
