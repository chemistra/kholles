% Niveau :      PCSI *
% Discipline :  Méca
% Mots clés :   Trajectoire particule chargée

\begin{exercise}{Arc électrique : c'est disruptif !}{2}{Sup, Spé}
{Mécanique,Mécanique particules chargées,Thermodynamique}{bermu}

\begin{questions}
\questioncours Quelle est la définition de l'électron-volt eV et dans quels cas l'utilise-t-on ?

\begin{EnvUplevel}
On se propose d'étudier une décharge électrique entre deux électrodes planes séparées d'une distance $d = 10$ mm et donc la différence de potentiel est $V_0 = 1$ kV. Le problème étant quasi 1D, on ne regardera que la variable d'espace $x$.
\end{EnvUplevel}

\question Quelle est l'expression du potentiel $V$ et du champ électrique $\vE$ entre les deux électrodes ?

\question Quelle est la trajectoire d'un seul électron en fonction du temps ? Cela vous paraît-il réaliste ?

\uplevel{La décharge électrique est créée par un phénomène d'\emph{avalanche électronique} : à chaque collision avec un atome d'air, un électron arrache un électron à l'atome qui lui-même arrache d'autres électrons \textit{etc}.}

\question Entre deux collisions, l'électron parcours une distance $\ell$ correspondant au libre parcours moyen dans le gaz. Supposant qu'il soit réémis suivant $Ox$, quelle est la vitesse moyenne $v_m$ de l'électron ?

%\question Et s'il était réémis dans des directions aléatoires ? (\emph{Vous pouvez seulement entamer la démarche sans faire les calculs.})

\question Sachant qu'il faut un travail $W_\text{ion}$ pour arracher un électron, donner un critère sur $V_0$ et le libre parcours moyen $\ell$ du gaz pour qu'il y ait une avalanche électronique. \\
Calculer $V_0$. Cela vous parait-il cohérent ?

\question Combien d'électrons auront été émis après un temps $t$ ? En déduire que la loi de courant des électrons est exponentielle
$$\vj_e = j_0 e^{\alpha x}\ve_x,$$
$j_0$ et $\alpha$ étant des paramètres que vous identifierez.

\uplevel{La décharge électrique étant aussi liée au courant des ions il faut également prendre en compte la dynamique des ions.}

\question Comment se comportent les ions ? Donner l'énergie cinétique des ions lorsqu'ils arrivent en $z=0$ et comparer à $W_\text{ion}$.

\uplevel{Cette vitesse étant très grande, cela signifie que les ions arrachent également des électrons à la cathode.}

\question Estimer le nombre $\gamma$ d'électrons arrachés par  un ion.

\question Justifier que
$$j(0) = \gamma\qty(j(0) - j(d)),$$
et en déduire la loi de Townsend
$$j(d) = j(0) \dfrac{e^{\alpha d}}{1 + \gamma(1 - e^{\alpha d})}.$$

\question Cette loi admet une singularité : donner la relation entre $\alpha$ et $\gamma$ à la singularité et interpréter physiquement la singularité.
\question En déduire la loi de Paschen :
$$V_\text{disruption} = U_\text{ion}\dfrac{d/\ell}{\ln{d/\ell} - \ln(1+\gamma^{-1})}.$$
Quelle est la tension de disruption du système ?
\end{questions}

\paragraph{Données :}
\begin{itemize}
    \item charge élémentaire $e = 1,6\times 10^{-19}$ C,
    \item masse le l'électron $m_e = 9,1\times 10^{-31}$ kg,
    \item masse d'un ion $m_i = 2,6\times 10^{-26}$ kg,
    \item dans l'air sec : $W_\text{ion} = 34$ V et $\ell = 70$ nm.
\end{itemize}
\end{exercise} 