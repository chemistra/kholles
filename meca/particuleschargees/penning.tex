\begin{exercise}{Piège de Penning}{3}{Sup}
{Particuleschargees}{lelay}

On s'intéresse au piège de Penning, qui est un exemple classique de piège électrostatique, c'est à dire un dispositif qui permet de circonscrire des particules chargées dans une région de l'espace.

\begin{questions}
    \questioncours Champ et potentiel électrique.
    \uplevel{On suppose qu'un arrive à réaliser à l'aide du système d'électrodes approprié le champ électrique $\vec{E}$ ci dessous}
    $$
    \vec{E} = \frac{U}{d}(x\vec{e}_x+y\vec{e}_y-2z\vec{e}_z)
    $$
    \question Quel doit être le signe de $U$ pour qu'une charge $q$ soumise à ce champ soit confinée autour du plan $z = 0$ ? Qu'en est-il du mouvement de la charge dans les directions $x$ et $y$ ?
    \question On ajoute à l'installation un dispositif générant un champ $\vec{B} = B_0 \ve_z$. Établir l'équation différentielle vérifiée par $u = x + iy$.
    \question Montrer que si le champ magnétique dépasse une valeur critique $B_c$, il est possible de confiner la particule.
    \question Déterminer $x(t)$ et $y(t)$ pour $B_0 \gg B_c$ en prenant comme condition initiale $\vec{OM} = x_0 \ve_x$ et $\vec{v}(0) = \vec{0}$. On fera apparaître deux pulsations caractéristiques.
    
\end{questions}

\end{exercise}
