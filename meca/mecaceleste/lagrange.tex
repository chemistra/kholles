\begin{exercise}{Points de Lagrange}{3}{Sup, Spé}
{Mécanique,Mécanique céleste,Mécanique en référentiel non galiléen}{lelay}

\begin{questions}
    \questioncours Position d'équilibre d'un satellite dans un champ gravitationnel et lois de Kepler.
\uplevel{Nous nous plaçons désormais dans le référentiel héliocentrique $(S ; r,\theta)$ en coordonnées polaires dans lequel la Terre $T$ est fixe en $r = R_\oplus = 1$ U.A. et $\theta=0$.}
\question Ce référentiel est-il Galiléen ?
\begin{EnvUplevel}
Dans ce cas, pour un corps de masse $m$, il faut prendre en compte la force centrifuge d'inertie
$$\vf_i = m\omega^2\vr,$$
dans le bilan des forces, avec $\omega$ la période orbitale de la Terre.
\end{EnvUplevel}
    \question Quelle est la valeur de $\omega$ ? Donner l'expression de $\omega$ à l'aide de la troisième loi de Kepler.
\uplevel{On appelle points de Lagrange, les positions d'équilibre d'un satellite de la terre dans le précédent référentiel.}
    \questionbonus Contributions scientifiques de Joseph-Louis Lagrange.
    \question Donnez l'équation vectorielle de ce point $\vr$ sans chercher à la résoudre pour le moment. \\
    Exprimer cette équation uniquement en fonction de
    $$Q = \dfrac{M_\oplus}{M_\odot} \qqtext{et} \vu = \dfrac{\vr}{R_\oplus}.$$
    \question Cette équation n'étant pas très simple, nous allons chercher plusieurs solutions particulières :
    \begin{parts}
        \part En projetant l'équation sur l'axe $\ve_y$, montrez que l'équation admet que le point de Lagrange se situe sur l'axe Terre--Lune. \\
        Nous ne regardons pas le second cas.
        \part Chercher les solutions de cette équation dans l'axe Terre -- Lune, proche de la Terre \emph{i.e.} $u \simeq 1$. \\
        Discuter de la pertinence de cette approximation.
        \part Comparer avec la position de la Lune ou d'un satellite en orbite géostationnaire et commenter.
        \part (\emph{Question facultative}) Ou chercheriez vous d'autres solutions sur l'axe des abscisses ?
    \end{parts}
\end{questions}

\plusloin[Les autres points de Lagrange]
Il y a 5 points de Lagrange, notés L$_1 \ldots$L$_5$. Comme nous l'avons vu en 6(b) L$_1$ et L$_2$ se situent proches de la Terre. L$_4$ et L$_5$ se situent au sommet des deux triangles équilatéraux dont un côté est le segment $ST$. On pourra montrer que ces deux points sont solution de l'équation au premier ordre en $Q$. 

\paragraph{Données :}
\begin{itemize}
    \item constante universelle de gravitation $G = 6,674 \times 10^{-11}$ SI, m$\cdot$s$^{-1}$,
    \item masse du Soleil $M_\odot = 1,989 \times 10^{30}$ kg,
    \item masse de la Terre $M_\oplus = 5,974 \times 10^{24}$ kg $= 3 \times 10^{-6} M_\odot$,
    \item distance Terre--Soleil $R_\oplus = 1$ U.A. $ = 1,496 \times 10^{11}$ m,
    \item distance Terre--Lune $d = 384 400$ km $= 2,6 \times 10^{-3}$ U.A.
\end{itemize}
\end{exercise}