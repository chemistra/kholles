% Niveau :      PCSI - PC
% Discipline :  Méca
% Mots clés :   Problème à 2 corps

\begin{exercise}{Le problème à deux corps}{3}{Sup, Spé}
{Mécanique,Mécanique céleste}{bermu,lelay}

On considère deux corps dans l'espace de masse $m_1$ et $m_2$, dont les positions par rapport à une origine placée en un point arbitraire sont $\vr_1$ et $\vr_2$.

\begin{questions}
    \question Donner la trajectoire des deux corps si $m_1 = m_2 = 0$
    \question Rappeler qualitativement la trajectoire de $m_1$ si $m_2 \gg m_1$
    \question On pose $\vR = \dfrac{m_1\vr_1 + m_2\vr_2}{m_1+m_2}$. Donner l'interprétation physique ce vecteur. Justifier puis démontrer qu'il n'est pas accéléré.
    \question Proposer un changement de variable $\vr = f(\vr_1, \vr_2)$ qui semble simplifier le problème.
    \question Exprimer $\vr_1$ et $\vr_2$ en fonction de $\vR$ et $\vr$
    \question Donner l'équation du mouvement de $\vr$. Cette équation vous est-elle familière ? Interprétez les différents termes et donner une signification physique à $\vr$.
    \question Donner une caractéristique du mouvement de $\vr$. Écrire l'équation différentielle vérifiée par $r = \| \vr \|$. Savez-vous la résoudre ?
    \question Ne vous sous-estimez pas. Réécrire l'équation avec le changement de variable $u = \frac1r$ et en notant par un point la dérivée par rapport à $\theta$ (on utilisera une identité bien connue du problème à deux corps pour faire le lien entre $\dd{t}$ et $\dd{\theta}$). Reconnaissez-vous l'équation obtenue ?
    \question Résoudre l'équation, exprimer la trajectoire $r(\theta)$ et discuter de la courbe obtenue en fonction d'un paramètre que l'on précisera.
\end{questions}
\textbf{Pour aller plus loin :} Effectuer le même raisonnement avec trois corps. Quid du cas à $N$-corps ?
\end{exercise}