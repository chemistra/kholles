% Niveau :      PC *
% Discipline :  Méca
% Mots clés :   Lune

\begin{exercise}{Autour de la Lune}{3}{Spé}
{Mécanique,Mécanique céleste, Mécanique en référentiel non galiléen}{bermu}

Considérons un satellite en orbite autour de la Terre, disons la Lune par exemple. Ce satellite est sous l'influence gravitaire de la Terre mais également du Soleil.

\begin{questions}
        \question (\emph{Culture sciences-physiques}) Que savez-vous du problème à 3 corps (c'est d'ailleurs le titre d'un roman de Cixin Liu) ?
        \question Sans calcul, montrez que si un satellite est trop éloigné de la Terre, il peut sortir de son orbite sous l'effet du Soleil.
        \question Soit un satellite de masse $\delta_m$ situé dans l'alignement Terre-Soleil. On se place dans le référentiel où le Soleil et la Terre sont fixes : ce référentiel est-il galiléen ? Si non, indiquer l'amplitude des forces d'inertie.
        \question On appelle $D$ la distance Soleil--Terre et $d$ la distance Terre--satellite, $M_\odot$ la masse du Soleil et $m_\oplus$ la masse de la terre. Quelles sont les forces s'exerçant sur $\delta_m$ ? 
        \question En supposant $\frac{d}{D} \ll \qty(\frac{m_\oplus}{M_\odot})^{\frac12}$, quelle est la valeur de $r$ pour laquelle ces forces sont à l'équilibre ? c'est la limite de Hill.
        \question Calculer la limite de Hill pour la Terre et le Soleil, et montrer que la Lune n'est pas en danger immédiat de sortir de son orbite.
\end{questions}

\paragraph{Données}
\begin{itemize}
    \item Masse de la Terre : $m_\oplus$ à retrouver
    \item Masse du Soleil : $M_\odot = 1,99\times 10^{30}$ kg.
\end{itemize}

\end{exercise}

\begin{solution}
\begin{questions}
    \question C UN ROMAN
    \question Zou
    \question Non, $F_i = \delta m r\Omega^2 \vb{e_r}$
    \question Si on prend le satellite entre le soleil et la Terre, $F_{\odot} = -\frac{GM_{\odot}\delta m}{(a-r)^2}\vb{e_r}$ et $F_{M} = \frac{GM\delta m}{r^2}\vb{e_r}$
    \question On doit trouver $r_H = a\qty(\frac13\frac{m}{M})^{1/3}$
    \question On trouve 1 500 000 km, la distance Terre-Lune est 380 000 km
\end{questions}
\end{solution}