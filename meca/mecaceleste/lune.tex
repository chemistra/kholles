% Niveau :      PC *
% Discipline :  Méca
% Mots clés :   Lune

\begin{exercise}{Autour de la Lune}{5}{Spé Ulm physique}
{Mécanique,Mécanique céleste}{bermu}

\begin{questions}
\question Soit un satellite autour de la Terre. \`A quelle conditions le satellite s’effondre-t-il sur lui-même ? La Lune est-elle donc stable ?
\question La période orbitale de la Lune est peu ou proue synchronisée avec celle de la terre, comment expliquer un tel phénomène ?
\end{questions}

\paragraph{Indices 1 : }
\begin{enumerate}
    \item Considérer les forces de marées et les forces de cohésion du satellite.
    \item Question très technique. Justifier que la forme de la Lune dans le champ de pesanteur de la terre ne peut être un sphère.
\end{enumerate}

\paragraph{Indices 2 : }
\begin{enumerate}
    \item Calculer le différentiel de forces gravitationelles entre les points du satellite le plus proche et le plus éloigné de la Terre, et comparez les à la force de gravité du satellite même. \\
Vous devriez trouver un rayon en deçà duquel le satellite s'effondre sur lui-même. C'est la limite de Roche.
    \item La Lune étant asymétrique par rapport à l'axe Lune terre, on peut par exemple la modéliser comme deux masses reliées par un bras rigide. Montrer que ce système décrit une équation d'oscillateur harmonique.
\end{enumerate}
\end{exercise}