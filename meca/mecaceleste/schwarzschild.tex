% Niveau :      PC *
% Discipline :  Méca
% Mots clés :   Lune

\begin{exercise}{Autour des trous noirs}{2}{Sup, Spé}
{Mécanique,Mécanique céleste}{bermu}

\begin{questions}
    \questioncours Vitesse de satellisation (\emph{i.e.} la vitesse nécessaire à un corps pour échapper à la gravité de l'astre sur lequel il se trouve).
    \question Définir brièvement ce qu'est un trou noir (et éventuellement l'horizon des évènements) et comment il se forme.
\begin{EnvUplevel}
Pour un trou noir classique, on peut estimer quelle est la taille critique $R_\textsc{s}$ en dessous de laquelle un objet de masse $M$ devient un trou noir en disant qualitativement que "même la lumière ne peut pas échapper à la gravité du trou noir".
\end{EnvUplevel}
    \question En utilisant les résultats de la question de cours, explicitez sous forme de formule mathématique la phrase précédente.
    \question En déduire le rayon de Schwarzschild $R_\textsc{s}$, en fonction de la masse de l'objet $M$ ainsi que du rayon $R_\odot$ et de la masse $M_\odot$ du soleil. Ce résultat est malgré toutes nos approximations exact !\\
    De combien doit-on réduire la taille du soleil pour qu'il devienne un trou noir. Interpréter au regard de la question 2.
\uplevel{\plusloin ~\vspace{-1em}}
    \question Discuter qualitativement des limites du modèle. Interpréter le terme de \emph{lentille gravitationelle}
\begin{EnvUplevel}
Dans le cas où le trou noir est en rotation à la vitesse $\Omega$ (trou noir de Kerr), comme par exemple le système binaire \textsf{GRS 1915+105}, l'horizon des évènements est donné par la relation
$$R_\textsc{k} = \dfrac{R_\textsc{s}}{2}\qty[1 + \sqrt{1 - \qty(\dfrac{2}{5}\dfrac{R^2 \Omega}{R_\textsc{s}\:c})^2}].$$
\end{EnvUplevel}
    \question Interpréter cette équation. Quid du soleil et de la terre ?
\end{questions}

\paragraph{Données :}
\begin{itemize}
    \item constante universelle de gravitation $G = \SI{6,674e-11}{SI}$,
    \item vitesse de la lumière dans le vide $c = \SI{2,998e8}{m.s^{-1}}$,
    \item masse du soleil $M_\odot = \SI{1,989e30}{kg}$,
    \item rayon du soleil $R_\odot = \SI{6,963e8}{m}$.
    \item période de rotation du soleil $T_\odot = 27$ j.
\end{itemize}
\end{exercise}

\begin{solution}  
    \begin{questions}
        \question $\En_p = \En_c$
    \end{questions}
\end{solution}