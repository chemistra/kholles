% Niveau :      PC *
% Discipline :  Méca
% Mots clés :   Lune

\begin{exercise}{Limite de Roche}{2}{Spé}
{Mécanique,Mécanique céleste}{bermu,lelay}

Considérons un satellite en orbite autours de la Terre, disons la Lune par exemple, tiraillé par la gravité terrestre. 

\begin{questions}

    \questioncours Effets de marées.
    
    \question Sans calcul, montrez que si un satellite est trop près de son corps de référence, il peut se disloquer sous l'effet du gradient gravitationnel. 
    
    \uplevel{Soit un satellite sphérique $m$ de rayon $r$ et de densité $\rho_m$ orbitant circulairement à une distance $d$ autour d'une planète sphérique $M$ de rayon $R$ et de densité $\rho_M$.
    
    On modélise la déformation du satellite en le découpant en deux sphères tangentes de masse $m/2$.}
    
    \question Quelle est la force de cohésion entre les deux sphères ?
    
    \question Quelle est la force de marée subie par les deux sphères ? \\ On rappelle que la force de marée est la différence entre la force appliquée en deux points de l'astre.
    
    \question Donner la distance $d$ pour laquelle ces forces sont égales. C'est la \textit{limite de Roche}. 
    
    \question Calculer la limite de Roche pour la Terre et la Lune, et montrer que la Lune n'est pas en danger de se disloquer. Calculer aussi cette limite pour Jupiter et la Terre (si vous vous demandez pourquoi : cf \textit{The Wandering Earth} (2019), film de Frant Gwo adapté de la nouvelle \textit{Terre errante} de Cixin Liu).
 \end{questions}

\paragraph{Données}
\begin{itemize}
    \item Masse de la Terre : à retrouver
    \item Rayon de la Terre : à retrouver
    \item Densité de la Lune : $3.34$ g/cm$^3$
    \item Densité de Jupiter : $1.33$ g/cm$^3$
    \item Rayon de Jupiter : $69\,911$ km
\end{itemize}

\end{exercise}

\begin{solution}
\begin{questions}
        \question C LÉ MARÉES
        \question Zou la dislocatioon 
        \question $ \frac{Gm\delta m}{r^2} $
        \question $ \frac{2GM \delta m r}{d^3} $
        \question $ d = r\qty( 2\frac{M}{m})^{1/3} = R\qty(2\frac{\rho_M}{\rho_m})^{1/3}$
        \question Rayon de Roche Terre-Lune : $\sim 9500$ km.
\end{questions}
\end{solution}