\begin{exercise}{Vitesse d'un satellite sur son orbite elliptique}{2}{Sup}
{Mécanique, Moment cinétique}{correge}

On considère un satellite de masse 1 tonne en orbite elliptique autour de la Terre. Le satellite se maintient sur cette trajectoire car il est soumis à la force gravitationnelle qu’exerce sur lui la Terre. Cette force est en permanence dirigée vers le centre de la Terre. On note :
\begin{itemize}
    \item $O$ le centre de la Terre qui est un des foyers de l’ellipse;
    \item $C$ le centre de l’ellipse;
    \item $A$ l’apogée du satellite (point de l’orbite le plus éloignée de la surface de la Terre);
    \item $P$ son périgée (point de l’orbite le plus proche de la surface de la Terre);
    \item $A'$ le point de la surface de la Terre qui fait face à A, l’apogée du satellite;
    \item $P'$ le point de la surface de la Terre qui fait face à P, le périgée du satellite;
    \item $S$ le point tel que $\vec{CS}$, vertical dirigé vers le haut, représente le demi-petit axe de l’ellipse.
\end{itemize}
On donne les grandeurs suivantes :
\begin{itemize}
    \item Distance $CS=16715$km;
    \item Distance $AA'=35000$km;
    \item Distance $PP'=350$km;
    \item Distance $OA'=OP'=RT=6400$km;
\end{itemize}
On considère que le satellite est en mouvement dans un référentiel galiléen dont le centre est situé au centre de la Terre.

\begin{questions}
    \questioncours Moment cinétique.
    \question Réaliser une figure comportant toutes les indications de l’énoncé. Représenter le satellite en $S$ avec sa vitesse $\vec v$ et les vecteurs polaires unitaires.
    \question La vitesse en $S$ est $v_S=\SI{14650}{km.h^{-1}}$. Calculer le moment cinétique du satellite en $S $ par rapport au point $O$ (un angle permettant de repérer le satellite peut être utile).
    \question En appliquant le théorème du moment cinétique, trouver la vitesse du satellite à son apogée $A$ et à son périgée $P$.
\end{questions}

\end{exercise}

\begin{solution}
\begin{questions}
    \questioncours 
    \question 
    \question $L_O(S)=mvCS=\SI{2.4e17}{kg.m^2.s^{-1}}$
    \question Avec le théorème du moment cinétique, on trouve un moment cinétique est constant. On sait qu'au périgée et à l'apogée, rayon vecteur et vitesse sont orthogonales. Donc $v_A=\SI{5.9e3}{km.h^{-1}}$, $v_P=\SI{3.6e4}{km.h^{-1}}$
\end{questions}

\end{solution}

