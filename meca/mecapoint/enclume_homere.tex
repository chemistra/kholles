\begin{exercise}{Distance au firmament}{2}{Sup, Spé}
{Mécanique,Mécanique du point}{lelay}

Xénophanes de Colophon, auteur grec de l'antiquité, affirmait que les étoiles étaient fixes et collées sur une grande surface semi-sphérique qu'il appelait firmament. Hésiode, à la même époque, précise la distance entre la surface de la Terre et le firmament en affirmant qu'un enclume lâchée depuis les étoiles mettrait neuf jours et autant de nuits à atteindre la Terre.  

\begin{questions}
    \questioncours Intégrale première du mouvement
    \question On suppose d'abord que l'accélération de la pesanteur est constante égale à $g$. Déterminer la distance entre les étoiles et la surface de la Terre d'après Hésiode. Cela vous paraît-il réaliste ?
    \question On suppose maintenant que la force de gravité s'exerçant sur l'enclume est la force générale de gravitation due à la Terre. Estimer la distance totale parcourue par l'enclume.
    \question Comparer aux autres distances du système solaire : Distance Terre Lune 384 000 km, distance Terre Mras 306 millions de km
\end{questions}

Données : On donne $\int \frac{\dd{x}}{\sqrt{\frac1x-1}} = -x\sqrt{\frac1x-1} - \arctan(\sqrt{\frac1x-1})$
\end{exercise}