% Niveau :      PCSI *
% Discipline :  Méca
% Mots clés :   Ballistique, Mécanique du point, PFD, Chute libre

\begin{exercise}{Variations sur le looping}{2}{Sup, Spé}
{Mécanique,Mécanique du point}{bedo,bermu}

\begin{questions}
\questioncours Lois du frottement solide.

\bigskip

\question\textsf{Question ouverte :} On considère une bille dans un champ de pesanteur lancée avec une certaine vitesse horizontale. La bille est contrainte d'adopter une trajectoire lui faisant faire un looping.

\'Etudier la différence entre les trois situations suivantes :
\begin{enumerate}[label={\bfseries \sffamily (a)}]
    \item la bille est une perle enfilée dans un circuit rigide en forme de cercle ;
    \item la bille roule dans une gouttière circulaire ;
    \item la bille est attaché à une corde non rigide et non élastique.
\end{enumerate}

\end{questions}

\paragraph{Conseils méthodologiques :}
\begin{itemize}
    \item \textsl{Schématiser ces situations et poser les grandeurs physiques qui vous paraissent importantes} ;
    \item \textsl{\'Etudier qualitativement les trois situations et identifier les différences} ;
    \item \textsl{Faire les calculs pour chaque situations} ;
    \item \textsl{Conclure}.
\end{itemize}

\end{exercise}
