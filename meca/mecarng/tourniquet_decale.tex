% Niveau :      Spé
% Discipline :  Méca RNG

\begin{exercise}{Bague sur un cerceau non centré}{2}{Spé}
{Mécanique,Mécanique en référentiel non galiléen}{lelay}

\begin{questions}
    \questioncours Forces d'inertie d'entraînement et de Coriolis.
\begin{EnvUplevel}
On considère une bague de masse $m$ enfilée sur un cerceau rigide, qui tourne à vitesse angulaire $\Omega$ autour d'un axe vertical lui étant tangent. Tous les frottements sont ici négligés.
\end{EnvUplevel}
    \question Donner l'équation du mouvement de la bague et interpréter qualitativement l'influence de chaque terme sur sa dynamique. On fera intervenir une pulsation caractéristique $\omega_0$ convenablement choisie.
    \question Combien de positions d'équilibre de la bague y a-t-il ?
    \question Donner la position d'équilibre de la bague en fonction du paramètre $\mu = \frac{\Omega^2}{\omega_0^2}$. 
    \question Montrer que toutes les forces considérées ici dérivent d'un potentiel et donner la forme de l'énergie potentielle totale $E_p$ de la bague.
    \question Tracer le profil de l'énergie potentielle et indiquer les positions d'équilibre stables et instables pour $\mu \ll 1$, $\mu \gg 1$ et $\mu$ quelconque.
    \question Imaginer une méthode graphique pour trouver les positions d'équilibre.
\end{questions}
\end{exercise}