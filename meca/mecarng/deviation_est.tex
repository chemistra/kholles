% Niveau :      Spé
% Discipline :  Méca
% Mots clés :   Chaine, caténaire

\begin{exercise}{Gravimétrie terrestre}{3}{Spé}
{Mécanique,Mécanique en référentiel non galiléen,Pesanteur,Gravité,Forces d'inertie}{bermu}

\begin{questions}
    \questioncours Caractère conservatif de la force d'inertie d’entraînement et champ de pesanteur dans un référentiel non galiléen. 
    
\begin{EnvUplevel}
On considère le référentiel terrestre qui se situe en surface de la Terre à la latitude $\lambda$.

\end{EnvUplevel}
    \question \textsf{Question ouverte :} le champ de pesanteur $g$ varie plus significativement si on se déplace de $h = \SI{1}{m}$ vers le nord, vers l'est, ou en altitude ? \`A quelle conditions peut-on estimer que $g$ est constant ?
\end{questions}
\paragraph{Notations :} 
On note $\Omega_\textsc{t}$ la pulsation associée à la rotation de la Terre, et $R_\textsc{t}$ son rayon.

\hspace{4.3em}
On note $\lambda$ la latitude qui est l'angle entre l'équateur et point le étudié ($\lambda = 49^\circ$ à Paris).
\end{exercise}

\begin{solution}
\begin{questions}
    \questioncours $$\vec{g} = \vec{g}_0 + \vec{a}_\text{ie} = -\dfrac{GM}{(R_\textsc{t}+z)^2}\vec{e}_z + \Omega^2_\textsc{t} R_\textsc{t} \cos\lambda \vec{e}_h = -\grad\qty(\dfrac{GM}{R_\textsc{t}+h} + \dfrac{1}{2}\Omega^2_\textsc{t} R_\textsc{t}^2 \cos\lambda^2),$$
$\vec{e}_z$ étant la verticale locale (depuis le centre de la Terre) et $\vec{e}_h$ le projeté orthogonal depuis l'axe de rotation de la Terre.
    \question
    \begin{itemize}
        \item \textsf{Variation selon $y$ :} nulle
        \item \textsf{Variation selon $z$ :}
        $$\delta g \simeq \delta\qty(-\dfrac{GM}{(R_\textsc{t}+h)^2}) = g_0 \dfrac{2 \delta z}{R_\textsc{t}} \qquad \dfrac{\delta  g}{g_0} \simeq \SI{3e-7}{}$$
        Raisonnable donc pour $\delta z \ll R_\textsc{t}$.
        \item \textsf{Variation selon $x$ :}
        $$\delta g \simeq \delta\qty(\Omega^2_\textsc{t} R_\textsc{t} \cos\lambda) = \Omega^2_\textsc{t} R_\textsc{t} \sin\lambda \delta\lambda = \Omega^2_\textsc{t} \sin\lambda \delta x \qquad \dfrac{\delta  g}{g_0} \simeq \SI{4e-10}{}$$
        Toujours raisonnable car $a_\text{ie}/g_0 \simeq \SI{5e-3}{}$ au max.
    \end{itemize}
\end{questions}
\end{solution}

% Niveau :      Spé
% Discipline :  Méca
% Mots clés :   Chaine, caténaire

\begin{exercise}{Expérience de Cassini}{3}{Spé}
{Mécanique,Mécanique en référentiel non galiléen,Chute libre,Déviation vers l'est}{bermu}

\begin{questions}
    \questioncours Caractère galiléen approché du référentiel terrestre. \\
    On appuiera la démonstration d'ordres de grandeur et on estimera la contribution des forces d'inertie dans un problème de dynamique terrestre.
    

\begin{EnvUplevel}
On se propose d'étudier l'expérience effectuée par Cassini en 1679 à l'observatoire de Paris sur les effets du caractère non galiléen du référentiel terrestre sur la chute libre d'un corps de masse $m$, supposé ponctuel, lâché sans vitesse dans un puits de profondeur $h$, et situé à la latitude $\lambda$.

On se placera dans le repère $(O,x,y,z)$ placé à la surface de la Terre, $Ox$ pointant vers le nord et $Oy$ vers l'ouest.

\end{EnvUplevel}
    \question Donner les équations de la dynamique de la particule, projetées sur chaque axe.
    \question Considérant que les effets non-galiléens son faibles, simplifier l'équation de sorte à ce qu'il n'y ait qu'un terme de force suivant chaque composante et résoudre l'équation.
    \question \`A l'observatoire de Paris (lattitude $\lambda = 49^\circ$), dans un puits de $h = \SI{158}{m}$ de profondeur,\linebreak Cassini a mesuré une déviation vers l'est de $y = 28$ mm vers l'est. Interpréter.
    \questionbonus Montrer également qu'on devrait observer une faible déviation vers le sud.
\end{questions}
\paragraph{Notations :} 
On note $\Omega_\textsc{t}$ la pulsation associée à la rotation de la Terre, et $R_\textsc{t}$ son rayon.

\hspace{4.3em}
On note $\lambda$ la latitude qui est l'angle entre l'équateur et point le étudié ($\lambda = 49^\circ$ à Paris).\end{exercise}

\begin{solution}
\begin{questions}
    \questioncours Dynamique inférieure à 24 heures.
    \question 
    $$\dv{\vv}{t} = -g\vec{e}_z - 2\vec{\Omega}_\textsc{t}\cross\vv = -g\mqty(0,0,1) - 2\Omega_\textsc{t} \mqty(\cos\lambda\\0\\-\sin\lambda)\cross\mqty(\dot{x}\\\dot{y}\\\dot{z}).$$

    $$\left\lbrace\begin{array}{l}
        \ddot{x} = 2\Omega_\textsc{t} \sin\lambda \dot{y}  \\
        \ddot{y} = 2\Omega_\textsc{t} (\sin\lambda \dot{x} + \cos\lambda \dot{z})  \\
        \ddot{z} = -g - 2\Omega_\textsc{t} \cos\lambda \dot{y}
    \end{array}\right.$$
    \question Or on a $\dot{x}$ et $\dot{y}$ négligeables, donc :
    $$\left\lbrace\begin{array}{l}
        \ddot{x} = 2\Omega_\textsc{t} \sin\lambda \dot{y}  \\
        \ddot{y} = 2\Omega_\textsc{t} \cos\lambda \dot{z}  \\
        \ddot{z} = -g
    \end{array}\right.$$
    d'où
    \begin{align*}
        \ddot{z} &= -g &
        \dot{z} &= -gt &
        z &= -\dfrac{1}{2}gt^2 \\
        \ddot{y} &= -2\Omega_\textsc{t} \cos\lambda g t &
        \dot{y} &= -\Omega_\textsc{t} \cos\lambda g t^2 &
        y &= - \dfrac{1}{3}\Omega_\textsc{t} \cos\lambda g t^3 \\
        \ddot{x} &= -2\Omega_\textsc{t}^2 \cos\lambda \sin\lambda g t^2 &
        \dot{x} &= -\dfrac{2}{3}\Omega_\textsc{t}^2 \cos\lambda \sin\lambda g t^3 &
        x &= -\dfrac{1}{6}\Omega_\textsc{t}^2 \cos\lambda \sin\lambda g t^4
    \end{align*}
    or $t = \sqrt{\dfrac{2h}{g}}$ d'où
    $$\left\lbrace\begin{array}{l}
        x =  -\dfrac{2}{3}\Omega_\textsc{t}^2 \cos\lambda \sin\lambda \dfrac{h^2}{g} \\
        y = - \dfrac{1}{3}\Omega_\textsc{t} \cos\lambda g \qty(\dfrac{2h}{g})^{3/2} \\
        z = - h
    \end{array}\right.$$
    \question On introduit une longueur typique :
    $$\ell = \dfrac{g}{\Omega_\textsc{t}^2} \simeq \SI{1,8e9}{m}$$
    $$\left\lbrace\begin{array}{l}
        x =  -\dfrac{2}{3} \cos\lambda \sin\lambda \dfrac{h^2}{\ell} \\
        y = - \dfrac{2\sqrt{2}}{3} \cos\lambda h \sqrt{\dfrac{h}{\ell}} \\
        z = - h
    \end{array}\right.$$
    On retrouve bien l'ODG.
    \question Pour $x$ on trouve une déviation vers le sud de $x = \SI{0,13}{um}$
\end{questions}
\end{solution}