% Niveau :      Spé
% Discipline :  Méca
% Mots clés :   Chaine, caténaire

\begin{exercise}{Pendule de Foucault}{3}{Spé}
{Mécanique,Mécanique en référentiel non galiléen,Pendule,Coriolis}{bermu}

\begin{questions}
    %\questioncours Caractère galiléen approché des référentiels héliocentrique, de Copernic, géocentrique et terrestre. \\
    %On appuiera la démonstration d'ordres de grandeur et on estimera la contribution des forces d'inertie dans un problème de dynamique terrestre.
    \questioncours Forces d'inertie d'entraînement et de Coriolis.
    %\question Montrer que sur Terre, à une lattitude $\lambda$ donnée, l'accélération d'inertie d'entraînement agit comme une correction sur $\vec{g}_0\mapsto\vec{g}$, dont on donnera l'ordre de grandeur. Quel est l'angle entre $\vec{g}_0$ et $\vec{g}$ ?

\begin{EnvUplevel}
On se propose d'étudier l'expérience effecutée par Léon Foucault en 1851 à Paris sur les effets du caractère non galiléen du référentiel terrestre sur la dynamique d'un pendule.

Soit un pendule simple, constitué d’une corde de très grande longueur $L$ de masse négligeable, suspendue en un point fixe $B$ du repère terrestre, et au bout de laquelle oscille un point matériel $M$ de masse~$m$. Soit $\lambda$ la latitude du lieu.

On étudie le mouvement du point $M$ dans le référentiel terrestre $\scr{R}$ lié au repère $(A ; x, y,z)$ centré sur la position d’équilibre $A$ du point $M$, $Az$ étant la verticale ascendante du lieu, $Ax$ dirigé vers le nord et $Ay$ vers l'est. On néglige tout frottement.
\end{EnvUplevel}
    \questionbonus Culture scientifique : contributions scientifiques de Léon Foucault.
    \question En précisant bien quels référentiels vous utilisez, effectuer un bilan des forces détaillé sur $M$.
\uplevel{Dans l’hypothèse de petites oscillations autour de la verticale locale $(Az)$, on a en première approximation que le mouvement s’effectue dans le plan $(A ; x, y)$ et que la composante suivant $Oz$ de la résultante des forces est nulle. On introduit donc $\Big.\vec{\xi} \deq x\ve_x + y\ve_y$, la projection de $M$ dans le plan $(A ; x, y)$.}
    \question Justifiez cette hypothèse et montrez qu'on a par conséquent que dans le plan $(A ; x, y)$
    \begin{align*}
        &\textsf{(a)}& \vec{T} & \propto \vec{AM} \simeq - m\omega_0^2 \vec{\xi}, &&\\
        &\textsf{(b)}& \vec{F}_\textsc{c} &\simeq -m f \ve_z \cross\dv{\vec{\xi}}{t},&&
    \end{align*}
    où $\vT$ est la tension de la corde, $\vec{F}_\textsc{c}$ la force de Coriolis, et $\omega_0$ et $f$ des quantités constantes (le justifier) dont on donnera les expressions, le sens physique et les ordres de grandeur.
    
    \question En déduire que l’équation vectorielle du mouvement projeté dans le plan horizontal $(A ; x, y)$ est
    $$\dv[2]{\vec{\xi}}{t} = -{\omega_0}^2\vec{\xi} - f \ve_z\cross\dv{\vec{\xi}}{t}$$
    et interpréter chaque terme.
    \question Montrez que si l'on se place dans un référentiel $\scr{R}'$ en rotation autour de l'axe $Oz$ par rapport à $\scr{R}$ à une pulsation $\omega_\textsc{c}$ bien choisie, on obtient l'équation d'un oscillateur harmonique.
    %$$\dv[2]{\vec{\xi}'}{t} = -{\omega_0}^2\vec{\xi}'.$$
    En déduire que le pendule oscille dans un plan tournant à une vitesse angulaire qu'on donnera.
\end{questions}
\paragraph{Données historiques :} pendule du Panthéon, $L = 67$ m et $\lambda = 49^\circ$.
On note $\omega_\textsc{t}$ la pulsation associée à la rotation de la Terre.

\noindent\textsf{Nota :} la latitude $\lambda$ est l'angle depuis l'équateur jusqu'au point étudié.
\end{exercise}


\begin{solution}

\begin{questions}
    \setcounter{question}{2}
    \question On se place dans $\scr{R}$, en translation circulaire autour
\end{questions}

\end{solution}