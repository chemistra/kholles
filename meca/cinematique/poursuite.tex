\begin{exercise}{La poursuite}{4}{Sup}
{Cinématique}{lelay}

On considère un étang circulaire de rayon $a$. A la circonférence de celui-ci se trouve un têtard, et au centre un poisson. À $t=0$ le poisson se rend compte de la présence du têtard et se met en tête de l'attraper : il se déplace vers le têtard à la vitesse $V$. Le têtard, paniqué, se met à nager le plus vite possible en faisant des tours de l'étang à la vitesse $a\omega$. On appelle $R(t)$ la distance entre le têtard et le poisson, et $\theta$ l'angle centre de l'étang / têtard / poisson.
\begin{questions}
    \question Faire un schéma.
    % En fait cet exo est d'une extrême difficulté, j'ai moi-même passé une heure dessus avec un succès mitigé et j'ai fini par choper la réponse dans un vieux livre de maths des années 60, c'est assez dégueu et ça n'a aucun intérêt donc je propose de laisser ça en plan et on le rédigera si il faut le filer a qqn qu'on déteste
    
\end{questions}

\end{exercise}