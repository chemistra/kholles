\begin{exercise}{Mouvement d'un satellite}{1}{Sup}
{Cinématique}{lelay}

On considère un satellite en mouvement circulaire uniforme au dessus de la Terre. Il ressent une accélération 
$$a = g \qty(\dfrac{R}{r})^2,$$
avec $R$ le rayon de la Terre et $r$ le rayon de l'orbite. Ce satellite est dit géostationnaire, c'est-à-dire que la période de révolution du satellite est égale à la période de rotation propre de la Terre.


\begin{questions}
    \questioncours Rappeler les caractéristiques (position, vitesse, accélération) du mouvement circulaire uniforme. Qu'est-ce qui change si le mouvement n'est plus uniforme ?
    
    \question Déterminer l'altitude du satellite.
    
    \question Calculer la norme de la vitesse du satellite.
    
    \question Quel est, selon vous, l'intérêt d'un tel satellite ? Les inconvénients ?
\end{questions}

\end{exercise}

\begin{solution}

\begin{questions}
    \questioncours Rappeler les caractéristiques (position, vitesse, accélération) du mouvement circulaire uniforme. Qu'est-ce qui change si le mouvement n'est plus uniforme ?
    
    \question $a = r\omega^2$, 36 000 km (42 000 avec le rayon de la Terre)
    
    \question $v = a\omega$
    
    \question Intéret : ca bouge pas donc ca couvre la moitie de la surface de la Terre en continu. Inconvenient : aux bords de la zone couverte ça recoit mal. Or geostationnaire ce n'est possible qu'au dessus de l'équateur, donc les pôles l'ont dans l'os. 
    
    On utilise alors des orbites géosynchrones, qui ont une certaine inclinaison. Pb : La moitié du temps on couvre un pole, l'autre moitié l'autre (alors qu'on s'en fout).
    
    Solution : On utilise des orbites excentriques pour passer la plupart du temps dans la zone intéressante. Voir par ex l'orbite de Molnya (période 12h).
\end{questions}

\end{solution}
