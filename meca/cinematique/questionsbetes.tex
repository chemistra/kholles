\begin{exercise}{Questions de cinématique}{1}{Sup}
{Cinématique}{lelay}


\begin{questions}
    \questioncours Rappeler les caractéristiques des coordonnées cartésiennes, polaires et sphériques.
    \question Quel est le système de coordonnées le plus simple a priori ? Exprimez la position, la vitesse et l'accélération d'un point dans ce système de coordonnées.
    \question On s'intéresse dans un premier temps aux coordonnées polaires 
    \begin{parts}
        \part Montrez qu'on peut toujours trouver un repère cartésien $\ve_x, \ve_y, \ve_z$ tel que $\ve_r$ et $\ve_\theta$ peuvent s'exprimer seulement en fonction de $r, \theta, \ve_x$ et $\ve_y$
        \part Quelle est alors l'expression du vecteur position $\vec{OM}$ en coordonnées cylindriques ?
        \part Montrez qu'on a $\dv{\ve_r}{t} = \ve_\theta$. Comment interpréter cela graphiquement ? En déduire $\dv{\ve_\theta}{t}$.
        \part Démontrez la formule de la vitesse en cylindrique
        \part Démontrez la formule de l'accélération en cylindrique
    \end{parts}
    \question On s'intéresse ensuite aux coordonnées sphérique 
    \begin{parts}
        \part Le problème de la conversion cartésien -- sphérique peut-il toujours se ramener à un problème plan ?
        \part Quelle est l'expression du vecteur $\vec{OM}$ en coordonnées sphériques ?
        \part Montrez que le volume élémentaire en sphérique est donnée par $\dd{\tau} = \dd{r} \times r \dd{\theta} \times r \sin\theta \dd{\varphi}$
        \part Démontrez la formule de la vitesse en sphérique
        \part Démontrez la formule de l'accélération en sphérique
    \end{parts}
\end{questions}

\end{exercise}

\begin{solution}
J'avoue que l'accélération en sphérique je la sais pas par coeur lol
\end{solution}
